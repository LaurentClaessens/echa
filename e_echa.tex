% This is part of Un soupçon de physique, sans être agressif pour autant
% Copyright (C) 2006-2010
%   Laurent Claessens
% See the file fdl-1.3.txt for copying conditions.

\documentclass[a4paper,12pt]{book}
%\documentclass[a4paper,12pt,draft]{book}

\usepackage{latexsym}
\usepackage{amsfonts}
\usepackage{amsmath}
\usepackage{amsthm}
\usepackage{amssymb}
\usepackage{bbm}
\usepackage{cases}
\usepackage{soul}

%\usepackage[notcite,notref]{showkeys}

\usepackage{eso-pic}
\usepackage{pstricks}
\usepackage{pst-node}
\usepackage{pst-eucl}
\usepackage{pst-plot}
\usepackage{pst-math}
\usepackage{pstricks-add}
\usepackage{subfigure}


\usepackage{multido}
\usepackage{fp}
\usepackage{ifthen}
\usepackage{graphicx}

\usepackage{makeidx}
\usepackage{isotope}
\usepackage[cdot,thinqspace,amssymb]{SIunits} 
 % L'option amssymb sert à éviter un conflit avec la commande \square de amssymb. Note qu'elle n'est plus accessible. Si tu en as besoin, faudra RTFM
%ftp://ftp.belnet.be/packages/ctan/macros/latex/contrib/SIunits/SIunits.pdf

\usepackage{verbatim}
\usepackage[ps2pdf]{hyperref} 		%Doit être appelé en dernier.
\hypersetup{colorlinks=true,linkcolor=blue}


%%%%%%%%%%%%%%%%%%%%%%%%%%
%
%   Trucs mathématiques
%
%%%%%%%%%%%%%%%%%%%%%%%%

\newcommand{\eR}{\mathbbm{R}}
\newcommand{\eZ}{\mathbbm{Z}}
\newcommand{\eQ}{\mathbbm{Q}}
\newcommand{\eN}{\mathbbm{N}}
\newcommand{\efrac}[2]{\frac{ \displaystyle #1 }{\displaystyle #2 }}

\DeclareMathOperator{\dom}{dom}

%%%%%%%%%%%%%%%%%%%%%%%%%%
%
%   Numérotations en tout genre
%
%%%%%%%%%%%%%%%%%%%%%%%%

\setcounter{tocdepth}{3}		% Profondeur de la table des matières
\setcounter{secnumdepth}{2}		% Profondeur dans le texte

%%%%%%%%%%%%%%%%%%%%%%%%%%
%
%   Les lignes magiques pour le texte en français.
%
%%%%%%%%%%%%%%%%%%%%%%%%

\usepackage[utf8]{inputenc}
\usepackage[T1]{fontenc}
\usepackage{textcomp}
\usepackage{lmodern}
\usepackage[a4paper]{geometry} 
\usepackage[english,frenchb]{babel}

\usepackage[fr]{exocorr}

\newcounter{bidon}
\newcommand{\Vbidon}{klmkm}


%%%%%%%%%%%%%%%%%%%%%%%%%%%%%%%%%%%%%
%
% Des choses de géométrie et de dessins
%
%%%%%%%%%%%%%%%%%%%%%%%%%%%%%%%%%%%%%%

%%%%%%%%% Résumé des principales commandes :

%	\pstMarquePoint[<+style+>]{<+Point à marquer+>}{<+position relative+>}{<+À noter+>}

%	\pstInterLL{<+point1+>}{<+point2+>}{<+repoint1+>}{<+repoint2+>}{<+intersection+>}

%	\pstTranslation{<+debut segment+>}{<+fin segment+>}{<+point accroche+>}[<+point final+>]
%	\pstRotation[RotAngle=<+angle+>]{<+debut segment+>}{<+bout segment+>}[<+point final+>]
%	\pstHomO[HomCoef=<+Coef d'omothetie+>]{<+centre+>}{<+point initial+>}[<+point final+>]
%	\pstTransHom{<+debut segment+>}{<+fin segment+>}{<+point accroche+>}{<+coef homo+>}{<+point final+>}
%	\pstTransRot{<+debut segment+>}{<+fin segment+>}{<+point accroche+>}{<+angle rotation+>}{<+point final+>}
%	\pstProjectionOrth{<+Premier point axe+>}{<+second point axe+>}{<+point a traiter+>}{<+point final+>}

%	\psaxes(<+là où les axes se croisent+>)(<+bas gauche+>)(<+haut droi+>)

%%%%%%%%%%% Fin du résumé %%%%%%%%%%%%%

% La macro pstLigneLongueur
% #1	Optionnel : ce sont les parametres de tracé
% #2	Premier point de la ligne
% #3	Second point de la ligne
% #4	Longueur de la ligne souhaitée
%
%   Cela trace une droite de longueur spécifiée qui passe par les deux points donnés. 
\newcommand{\pstLigneLongueur}[4][]{%
   \pstMiddleAB{#2}{#3}{interC}						% Trouver le milieu du segment donné
   \pstInterLC[Diameter=\pstDistVal{#4}]{#2}{#3}{interC}{}{interP}{interQ}		% Points d'intersection entre la droite et le cercle de diamètre #4
   \pstLineAB[#1]{interP}{interQ}
}


% La macro ValeurFonction
% #1	est le nom de la fonction à calculer. Cette dernière doit être une macro à un argument du genre 3*#1+5/#1
% #2	est la valeur de la variable en laquelle on veut calculer la fonction
% #3	est le nom de la variable FP dans laquelle le résultat sera écrit.
\newcommand{\ValeurFonction}[3]{\FPeval{#3}{#1{#2}}}

% La macro pstPointSurCourbe
% #1	est le nom de la fonction
% #2	est le x où l'on veut le point
% #3	est le nom du point pstricks qui sera en x sur la courbe.
\newcommand{\pstPointSurCourbe}[3]{%
\ValeurFonction{#1}{#2}{inter}
\pstGeonode(#2,\inter){#3}
}

\newcommand{\pstTortue}[2]{%
\pstTranslation{O}{#1}{TorPos}[#2]
\pstGeonode(#2){TorPos}
}

% La macro markpoint
% #1	(optionel) est le style dans lequel le point va être marqué
% #2,#3	les coordonées du point
% #4	le nom du point
%
% Le tout crée un point aux coordonnées données et trace des lignes en pointillé du point vers les axes.
\newcommand{\markpoint}[4][]{%
		\pstGeonode[#1](#2,#3){#4}
		\psline[linestyle=dashed](0,#3)(#2,#3)(#2,0)
			}


\newcommand{\zigzag}{\psline(0,0)(0.25,0.3)(0.5,-0.3)(0.75,0)}
\newcommand{\psPtRel}[3]{ \rput(#1){\pstGeonode(#2)}{#3} }


\newcommand{\pstBoite}[5]{%
\rput(#1){\pstGeonode(#2,0){ibdm}}
\pstHomO[HomCoef=0.5]{#1}{ibdm}[ihd]
\rput(ihd){\pstGeonode(-#2,-#3){ibg}}
\psframe(ihd)(ibg)
\pstMiddleAB{ibg}{ihd}{#4}
\pstHomO[HomCoef=2]{#1}{#4}[#5]
}

%%%%%%%%%%%%%%%%%%%%%%%%%%
%
%   Des constructions géométriques
%
%%%%%%%%%%%%%%%%%%%%%%%%

% Pour la commande \pstTransHom
%1 Début du segment
%2 fin du segment
%3 Le point d'accroche
%3 Le coef d'homotéthie
%5 Le nom du point obtenu

\newcommand{\pstTransHom}[5]{%		
\pstTranslation{#1}{#2}{#3}[inter]
\pstHomO[HomCoef=#4]{#3}{inter}[#5]
}

\newcommand{\pstTransRot}[5]{%
\pstTranslation{#1}{#2}{#3}[inter]
\pstRotation[RotAngle=#4]{#3}{inter}[#5]
}

% La commande suivante donne les points de tangence passant par un point donné à un cercle donné.
% #1 : centre du cercle
% #2 : un point du cercle
% #3 : le point d'où doit partir les tangentes
% #4 et #5 : les points de tangence
\newcommand{\pstTangenteOPM}[5]{%
\pstMiddleAB{#1}{#3}{inter}
\pstInterCC{#1}{#2}{inter}{#3}{#4}{#5}
}

%%%%%%%%%%%%%%%%%%%%%%%%%%
%
%   Mécanique et forces
%
%%%%%%%%%%%%%%%%%%%%%%%%

% La macro \pstMarqueForce prend les arguments suivants :
% 1 est le départ
% 2 est l'arrivée
% 3 est l'endroit par rapport à l'arrivée où on va mettre le label
% 4 est le label en question
\newcommand{\pstMarqueForce}[4]{%
\psline[arrows=->](#1)(#2)
\rput(#2){\rput(#3){#4}}
}

% La macro \pstMarquePoint prend les arguments suivants :
% - 1 est optionnel : il donne le style de point qui va arriver
% - 2 est le point (au sens pstrick) qui doit être marqué
% - 3 est la position où va se trouver la marque par rapport au point pstrick. Typiquement c'est (0.3;90) en coordonnées polaires
% - 4 est ce qu'on va y noter
\newcommand{\pstMarquePoint}[4][PointSymbol=none]{%
\rput(#2){\rput(#3){#4}}				% Mettre le symbole du point là où il doit être
\pstGeonode[#1](#2){mpinter}				% Cette ligne fait apparaître un point à l'endroit que l'on marque.
							%   étant donné le \psset{PointSymbol=none} dans lequel toutes les figures se trouvent,
							%   ceci ne fait pas grand chose si on ne donne pas à \pstMarquePoint l'argument optionnel
							%   du type PointSymbol=* par exemple.
}

% La macro suivante prend comme arguments :
% - début et fin d'une force
%math - début et fin d'un premier vecteur
% - début et fin d'un second vecteur
% Puis place les points qui décomposent la force dans les noms des deux derniers arguments

\newcommand{\pstDecompForce}[8]{%
  \psset{PointSymbol=none, PointName=none}
\pstTranslation{#3}{#4}{#1}[ibBu]
\pstTranslation{#5}{#6}{#1}[ibBd]

\pstTranslation{#3}{#4}{#2}[jbBu]
\pstTranslation{#5}{#6}{#2}[jbBd]

\pstInterLL{#2}{jbBd}{#1}{ibBu}{#7}
\pstInterLL{#2}{jbBu}{#1}{ibBd}{#8}
}

% Pour pstProjectionOrth
% 1 et 2 définissent une droite
% 3 est un point
% 4 est la projection orthogonale de ce point sur le droite
\newcommand{\pstProjectionOrth}[4]{%
  \psset{PointSymbol=none, PointName=none}
	\pstRotation[RotAngle=90]{#1}{#2}[PrInter]
	\pstTranslation{#1}{PrInter}{#3}[PrInter2]
	\pstInterLL{#1}{#2}{#3}{PrInter2}{#4}
}

\newcommand{\tabconvun}[4]{%
\begin{tabular}{lcl}
en \kilo\meter\per\hour		&:& \unit{#1}{\kilo\meter\per\hour}\\
en \meter\per\second		&:& \unit{#2}{\meter\per\second}\\
en \kilo\meter\per\second	&:& \unit{#3}{\kilo\meter\per\second}\\
en \milli\meter\per\second	&:& \unit{#4}{\milli\meter\per\second}
\end{tabular}
}

\newcommand{\tabconvunmh}[4]{%
\begin{tabular}{lcl}
en \kilo\meter\per\hour		&:& \unit{#1}{\kilo\meter\per\hour}\\
en \meter\per\second		&:& \unit{#2}{\meter\per\second}\\
en \kilo\meter\per\second	&:& \unit{#3}{\kilo\meter\per\second}\\
en \meter\per\hour		&:& \unit{#4}{\meter\per\hour}
\end{tabular}
}

%%%%%%%%%%%%%%%%%%%%%%%%%%
%
%   Optique géométrique
%
%%%%%%%%%%%%%%%%%%%%%%%%


\newcommand{\pstRayon}[3][linecolor=yellow]{%
  \psset{PointSymbol=none, PointName=none}
\pstMiddleAB{#2}{#3}{inter}
{
\psset{linecolor=red}
\psset{#1}
%\psline[arrows=->](#2)(inter)  % Ce qui serait bien c'est d'avoir la longueur du segment, de la diviser par deux et de mettre la flèche bien au milieu.
%\psline(#2)(#3)
\psline[ArrowInside=->](#2)(#3)
} % fin du psset
}

% Pour pstDioptre
% 2 et 3 sont deux points qui définissent la droite du dioptre. En particulier, 1 est celui où aboutit le rayon incident
% 4 est le point d'où vient le rayon incident
% 5 et 6 sont les deux indices de réfraction
% 7 et 8 sont les points du réfléchi et du réfracté avec distance égale à la distance 1-3. Le réfracté est en (0,0) si il n'existe pas.
%
%   En ce qui concerne la réflexion totale, je crois que ce qui se passe c'est que dans ce cas il n'y a pas d'intersection entre le cercle et la ligne
% dans ma construction. Du coup elle est mise au centre du cercle et donc le rayon se réduit à un point.
%
%  Attention : il y a un argument optionnel. En fait le rayon réfracté est construit comme l'intersection entre un cercle et une droite. Or il y a deux intersections. Dans les cas usuels (de haut vers le bas), #8 est le bon. Mais si c'est le mauvais, alors on récupère l'autre dans l'argument optionnel.
\newcommand{\pstDioptre}[8][Diinter]{%
  \psset{PointSymbol=none, PointName=none}
	\pstGeonode(#2){DiO}(#3){DiP}(#4){DiRi}
	\pstRotation[RotAngle=90]{DiO}{DiP}[DiQ]				% Le segment OQ désigne la normale au plan OP.
   	\pstDecompForce{DiO}{DiRi}{DiO}{DiP}{DiO}{DiQ}{DiRix}{DiRiy}		% Décompose le rayon incident en parallèle et normal au plan du dioptre
  \FPeval{Dicoefrel}{#5/#6}
	\pstTransHom{DiRi}{DiRiy}{DiO}{\Dicoefrel}{DiRex}			% La composante x du rayon entrant est n1/n2 fois celle du rayon incident

   	\pstTranslation{DiO}{DiRex}{DiQ}[DiR]
   	\pstInterLC{DiRex}{DiR}{DiO}{DiRi}{#8}{#1}				% Voila qui conclu le rayon réfracté. Tu notes que le #8 est celui qu'on utilise par défaut tandis que #1 est 
										%   l'argument optionnel au cas o on considère un rayon qui va du bas vers le haut.

	\pstTranslation{DiRi}{DiRiy}{DiRiy}[#7]					% Place le rayon réfléchi
}

% Pour pstMiroir
% 1 et 2 désignent la ligne du miroir
% 3 est le point lumineux
% 4 est un point du rayon qui s'échappe de 3
% 5 est l'image
% 6 est le point de tombée du rayon sur le miroir
% 7 est un point du rayon réfléchi (celui qui est ausi loin de 6 que 5)
%
\newcommand{\pstMiroir}[7]{%
\pstRotation[RotAngle=90]{#1}{#2}[MiPerp]
\pstDecompForce{#1}{#3}{#1}{#2}{#1}{MiPerp}{MiNor}{MiC}		% Calcule le projeté perpendiculaire au miroir de l'image
\pstTranslation{#3}{MiNor}{MiNor}[#5]				% Place l'image
\pstInterLL{#1}{#2}{#3}{#4}{#6}					% Place le point de percée du rayon lumineux sur le miroir
\pstTranslation{#5}{#6}{#6}[#7]					% Place le fameux point du rayon réfléchi
}

% Pour pstMiroirSph
% 1 est le centre
% 2 est le point de la sphère qui correspond à l'axe optique
% 3 est le point dont il faut calculer l'image
% 4 est le foyer (calculé)
% 5 et 6 sont les points d'intersection du rayon parallèle avec le miroir et un point du prolongement du côté opposé au foyer
% 7 et 8 sont la même chose, mais pour le rayon passant par le centre
% 9 est enfin le point image !
\newcommand{\pstMiroirSph}[9]{%
\pstMiddleAB{#1}{#2}{#4}		% Voila qui place déjà le foyer
\pstTranslation{#1}{#2}{#3}[MiAl]
\pstInterLC{#3}{MiAl}{#1}{#2}{Miautr}{#5}
\pstInterLC{#3}{#1}{#1}{#2}{Miautr}{#7}
\pstInterLL{#5}{#4}{#3}{#1}{#9}		% Place l'image
\pstTranslation{#9}{#5}{#5}[#6]		% Place un point de prolongement pour le rayon réfléchit qui venait parallèlement à l'axe optique. Il est à la distance entre le point de tombée 
					%   sur le miroir et l'image.
\pstTranslation{#9}{#7}{#7}[#8]
}

% Pour pstTraceMiroirSph
% 1 est le centre
% 2 est le point de la sphère qui correspond à l'axe optique
% 3 est le foyer
% 4 L'angle sous lequel on va tracer le miroir
% 5 Le premier point du bord du miroir
% 6 Le second point du bord du miroir
\newcommand{\pstTraceMiroirSph}[6]{%
\pstRotation[RotAngle=#4]{#1}{#2}[#6]
\pstRotation[RotAngle=-#4]{#1}{#2}[#5]
\pstMiddleAB{#1}{#2}{#3}
}

% Pour pstLentille.
% 1 centre optique
% 2 foyer
% 3 le haut de l'objet, càd le point A
% 4 intersection du rayon passant par le foyer avec la lentille
% 5 image de A
% 6 le point B (projection de A sur l'axe optique)
% 7 image de B (projection de A' sur ce même axe)résume %%%%%%%%%%%%%

\newcommand{\repere}{%
\psset{ticksize=.7pt,linewidth=.7\pslinewidth,labelsep=2.5pt}
\pstGeonode[PosAngle=-135]{O}
\pcline{->}(0,0)(1,0)\rput(0.5,-0.4){$\vec i$}
\pcline{->}(0,0)(0,1)\rput(-0.5,0.5){$\vec j$}
		    }


\newcommand{\pstLentille}[7]{%
 \pstRotation[RotAngle=90]{#1}{#2}[leninterP]		% Pour se donner la direction de la lentille
 \pstInterLL{#3}{#2}{#1}{leninterP}{#4}			% L'endroit où le rayon passant par le foyer intersecte la lentille
 \pstTranslation{#2}{#1}{#4}[leninterQ]			% se donne la droite sur laquelle repart le rayon parallèlement
 \pstInterLL{#3}{#1}{#4}{leninterQ}{#5}			% Trouve l'image de A
 \pstProjection{#2}{#1}{#3}[#6]
 \pstProjection{#2}{#1}{#5}[#7]
}


%%%%%%%%%%%%%%%%%%%%%%%%%%
%
%   Les théorèmes et choses attenantes
%
%%%%%%%%%%%%%%%%%%%%%%%%


\newcounter{numtho}
\newcounter{numloiphyz}

\newtheoremstyle{mes_tho}%
		{9pt}{9pt}%
		{\itshape}%
		{}%
		{\bfseries}{.}%
		{\newline}%
		{}%

\newtheoremstyle{mes_exemples}%
		{9pt}{9pt}%
		{}%
		{}%
		{\bfseries}{.}%
		{\newline}%
		{}%


\newtheoremstyle{mes_exclamations}%
		{9pt}{9pt}%
		{}%
		{}%
		{\bfseries}{}%
		{\newline}%
		{}%

\theoremstyle{mes_exemples}	\newtheorem{exemple}[numtho]{Exemple}
				\newtheorem{remark}[numtho]{Remarque}
				\newtheorem{amusement}[numtho]{Amusement}
				\newtheorem{erreur}[numtho]{Error}
				%\newtheorem{exercice}{Exercice}			% Les exercices ne se numérotent pas avec les autres, pour que les références soient plus faciles à suivre dans la partie corrigée.


\theoremstyle{mes_tho}
			\newtheorem{theoreme}[numtho]{Théoreme}
			\newtheorem{lemma}[numtho]{Lemme}
			\newtheorem{proposition}[numtho]{Proposition}
			\newtheorem{corollary}[numtho]{Corollaire}
			\newtheorem{theorem}[numtho]{Théorème}
			\newtheorem{definition}[numtho]{Définition}
			\newtheorem{loiphyz}[numloiphyz]{Loi numéro}

\newenvironment{enplus}{\bgroup\footnotesize\begin{amusement}}{\end{amusement}\egroup}


% Le fichier DefsExclamations contient les définitions d'environements du type
%	 \newenvironment{idee}{\bgroup\green \begin{pour_idee}}{\end{pour_idee}\egroup}
% ce fichier est créé par le scripte externe DefsExclamations.py. Taper
%
% ./DefsExclamations.py > DefsExclamations.tex

\newcounter{contBla}
\renewcommand{\thecontBla}{}
\newtheoremstyle{exclamations}{9pt}{9pt}{}{}{\bfseries}{}{\newline}{}
\theoremstyle{exclamations}
\newtheorem{pour_idee}[contBla]{Idée !}
\newtheorem{pour_pourquoidonc}[contBla]{Comment ? Pourquoi ?}
\newtheorem{pour_probleme}[contBla]{Y'a une faute ?}
\newenvironment{idee}{\bgroup \green \begin{pour_idee}}{\end{pour_idee}\egroup}
\newenvironment{pourquoidonc}{\bgroup \blue \begin{pour_pourquoidonc}}{\end{pour_pourquoidonc}\egroup}
\newenvironment{probleme}{\bgroup \red \begin{pour_probleme}}{\end{pour_probleme}\egroup}



% Il y a une astuce pour définir plein de théorèmes en un coup dans la discussion suivante
% http://groups.google.fr/group/fr.comp.text.tex/browse_thread/thread/ca9175f9043466fb?hl=fr#

% Un bogue à propos de l'affichage de couleur en haut de page dans okular est posté ici :
% https://bugs.kde.org/show_bug.cgi?id=202612

\renewcommand{\theenumi}{\alph{enumi}}

\newcommand{\defe}[2]{\textbf{#1}\index{#2}}
%\newcommand{\abstracty}{Résumé de ce qui t'attend}
\makeatletter
  \newenvironment{abstract}{%
      \if@twocolumn
        \section*{\abstractname}%
      \else
        \small
        \begin{center}%
          {\bfseries \abstractname \vspace{-.5em}\vspace{\z@}}%
        \end{center}%
        \quotation
      \fi}
      {\if@twocolumn\else\endquotation\fi}
\makeatother
%%%%%%%%%%%%%%%%%%%%%%%%%%
%
%   Les macros qui font des choses
%
%%%%%%%%%%%%%%%%%%%%%%%%

\newcommand{\tq}{\mid}
%\newcommand{\emu}{^{-1}}
\newcommand{\bF}{\mathbf{F}}
\newcommand{\bDx}{\mathbf{\Delta x}}
\newcommand{\fF}{\overrightarrow{F}}
\newcommand{\fP}{\overrightarrow{P}}
\newcommand{\fG}{\overrightarrow{G}}
\newcommand{\fR}{\overrightarrow{R}}
\newcommand{\fAB}{\overrightarrow{AB}}
\newcommand{\fh}{\overrightarrow{h}}
\newcommand{\fv}{\overrightarrow{v}}

\newcommand{\mA}{\mathcal{A}}
\newcommand{\mO}{\mathcal{O}}

