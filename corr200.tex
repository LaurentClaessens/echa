% This is part of Un soupçon de physique, sans être agressif pour autant
% Copyright (C) 2006-2009
%   Laurent Claessens
% See the file fdl-1.3.txt for copying conditions.



% Ce fichier est généré automatiquement par le script ran_exo.py
  \begin{corrige}{200}

\begin{align*}
S_{1}&=\left\{-3,-5\right\}&S_{2}&=\left\{0,-2\right\}\\
S_{3}&=\left\{10,-6\right\}&S_{4}&=\left\{0,9\right\}\\
S_{5}&=\left\{-3,1\right\}&S_{6}&=\left\{-7,-3\right\}\\
S_{7}&=\left\{3,10\right\}&S_{8}&=\left\{-3,8\right\}\\
S_{9}&=\left\{-6,6\right\}&S_{10}&=\left\{-1,7\right\}
\end{align*}
 Affin d'avoir deux solutions entières, un trinome doit s'écrire sous la forme $a(x-x_1)(x-x_2)$ où $x_1$ et $x_2$ sont les deux racines entières. Le trinôme aura donc toujours la forme $ax^2-a(x_1+x_2)+ax_1x_2$. C'est pour cela que \emph{tous} les trinômes de cet exercice peuvent commencer par simplifier le coefficient de $x^2$.
\end{corrige}