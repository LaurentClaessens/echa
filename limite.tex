\subsection{Limites en des nombres}
%----------------------------------

Si tu regardes la fonction $f(x)=5x+3$, tu ne serais pas étonnée si je te disais par exemple que 
\begin{align}
\lim_{x\to 10}f(x)&=53&\text{et}&\lim_{x\to 0}f(x)=3.
\end{align}
En effet, plus $x$ est proche de $10$, plus $f(x)$ est proche de $53$ et plus $x$ est proche de $0$, plus $f(x)$ est proche de $3$. Pas grand chose de neuf sous le Soleil.

 Oui, mais l'intérêt d'introduire le concept de limite dans le cas de l'infini était qu'on ne peut pas bêtement calculer $f(\infty)$. Il fallait donc une astuce pour parler du comportement de $f$ quand on s'approche de l'infini.

Ici, $f(10)=53$ et $f(0)=3$, donc on ne voit pas très bien pourquoi il faudrait s'inquiéter et introduire une notion de limite. Malheureusement, si tu prenais la peine de regarder encore une fois la figure \ref{FigUnSurx}, tu verrais qu'on a quand même besoin d'une astuce pour décrire ce qu'on voudrait dire  : la fonction $1/x$ tend vers l'infini quand $x$ tend vers zéro. Je sais que c'est ce que tu veux dire, mais nous allons voir qu'en réalité c'est faux\footnote{Surtout si ce que tu voulais dire est quelque chose du genre \og je m'en fous !\fg.}. Cela va faire l'objet d'une subtilité très marrante. Mais faisons comme si de rien n'était et posons la définition suivante :
\begin{definition}		\label{DefInfNombre}
Lorsque $a\in\eR$, on dit que la fonction $f$ \defe{tend vers l'infini quand $x$ tend vers $a$}{} si
\[ 
  \forall M\in\eR,\exists \delta\tq (| x-a |\leq \delta )\Rightarrow f(x)\geq M\text{ quand $x\in\dom f$}.
\]
\end{definition}
Cela signifie que l'on demande que dès que $x$ est assez proche de $a$ (c'est à dire dès que $| x-a |\leq\delta$), alors $f(x)$ est plus grand que $M$, et que l'on peut trouver un $\delta$ qui fait ça pour n'importe quel $M$. Une autre façon de le dire est que pour toute hauteur $M$, on peut trouver un intervalle de largeur $\delta$ autour de $a$\footnote{C'est à dire un intervalle de la forme $[a-\delta,a+\delta]$.} tel que sur cet intervalle, la fonction $f$ est toujours plus grande que $M$.

Montrons sur un dessin pourquoi je disais que la fonction $x\to 1/x$ n'est pas de ce type. Regarde donc la figure \ref{FigUnsurxContreEx}.
\begin{figure}[ht]
\centering
\psset{xunit=0.6,yunit=0.6}
\begin{pspicture}(-11,-5)(11,5)
  \psaxes[dotsep=1pt,Dy=2,Dx=2]{->}(0,0)(-10.9,-4.9)(11,5)
	\psset{PointSymbol=none, PointName=none}
	\def\Fn{1 x div}
	\psplot[linecolor=red]{0.2}{10.3}{\Fn}
	\psplot[linecolor=red]{-10.3}{-0.2}{\Fn}
\end{pspicture}
\caption{La fonction $f(x)=1/x$. Le problème est que ça descend d'un côté et que ça monte de l'autre.}\label{FigUnsurxContreEx}
\end{figure}
Le problème est qu'il n'existe par exemple aucun intervalle autour de $0$ sur lequel $f$ serait toujours plus grande que $10$. En effet n'importe quel intervalle autour de $0$ contient au moins un nombre négatif. Or quand $x$ est négatif, $f$ n'est certainement pas plus grande que $10$. Nous y reviendrons.

Pour l'instant, montrons que la fonction $f(x)=1/x^2$ de la figure \ref{FigUnSurxCarr} est une fonction qui vérifie la définition \ref{DefInfNombre}. 
\begin{figure}[ht]
\centering
\psset{xunit=0.6,yunit=0.6,Dx=2,Dy=2}
\begin{pspicture}(-11,-1)(11,11)
  \psaxes[dotsep=1pt]{->}(0,0)(-10.9,0)(11,10)
	%\psframe[linecolor=cyan](-11,-1)(11,11)
	\psset{PointSymbol=none, PointName=none}
	\def\Fn{1 x 2 exp div}
	\psplot[linecolor=red]{0.30151}{10}{\Fn}
	\psplot[linecolor=red]{-10}{-0.30151}{\Fn}
\end{pspicture}
\caption{La fonction $f(x)=1/x^2$. Elle monte bien vers l'infini quand $x$ tend vers zéro; tant du côté des négatifs que du côté des positifs.}\label{FigUnSurxCarr}
\end{figure}
Avant de prendre n'importe quel $M$, prenons par exemple $100$. Nous avons besoin d'un intervalle autour de zéro sur lequel $f$ est toujours plus grande que $100$. C'est vite vu que $f(0.1)=f(-0.1)=100$, donc l'intervalle $[-\frac{ 1 }{ 10 },\frac{1}{ 10 }]$ est le bon. Partout dans cet intervalle, $f$ est plus grande que $100$. Partout ? Ben non : en $x=0$, la fonction n'est même pas définie, donc c'est un peu dur de dire qu'elle est plus grande que $100$. C'est pour cela que nous avons ajouté la condition \og quand $x\in\dom f$\fg{} dans la définition de la limite.

Prenons maintenant un $M\in\eR$ arbitraire, et trouvons un intervalle autour de $0$ sur lequel $f$ est toujours plus grande que $M$. La réponse est évidement l'intervalle de largeur $1/\sqrt{M}$, c'est à dire 
\[ 
  \left[ -\frac{ 1 }{ \sqrt{M} },\frac{ 1 }{ \sqrt{M} } \right].
\]


\section{Limite et continuité}
%++++++++++++++++++++++++++++++


\subsection{Limites quand tout va bien}
%--------------------------------------

D'abord définissons ce qu'on entend par la limite d'une fonction en un point quand il n'y a aucun infini en jeu.
\begin{definition}		\label{DefLimPointSansInfini}
 On dit que la fonction $f$ \defe{tend vers $b$ quand $x$ tend vers $a$}{} si 
\[ 
  \forall \epsilon>0,\exists\delta\tq (| x-a |\leq\delta)\Rightarrow | f(x)-b |\leq \epsilon\text{ quand $x\in\dom f$}.
\]
Dans ce cas, nous notons
\begin{equation}
\lim_{x\to a}f(x)=b.
\end{equation} 
\end{definition}

Commençons par un exemple très simple : prouvons que $\lim_{x\to 0}x=0$. C'est donc $a=b=0$ dans la définition. Prenons $\epsilon>0$, et trouvons un intervalle autour de zéro tel que partout dans l'intervalle, $x\leq \epsilon$. Bon ben c'est clair que $\delta=\epsilon$ fonctionne.

Plus compliqué maintenant, mais toujours sans surprises.

\begin{proposition}
\[ 
  \lim_{x\to 0}x^2=0.
\]

\end{proposition}

\begin{proof}
Soit $\epsilon>0$. On veut un intervalle de largeur $\delta$ autour de zéro tel que $x^2$ soit plus petit que $\epsilon$ sur cet intervalle. Cette fois-ci, le $\delta$ qui fonctionne est $\delta=\sqrt{\epsilon}$. En effet un élément de l'intervalle $[-\delta,\delta]$ est un $r$ de valeur absolue plus petite ou égale à $\delta$ : 
\[ 
| r |\leq\delta=\sqrt{\epsilon}.
\]
En prenant le carré de cette inégalité on a :
\[ 
  r^2\leq\epsilon,
\]
ce qu'il fallait prouver.
\end{proof}

\Exo{204}

Calculer et prouver des valeurs de limites, mêmes très simples, devient vite de l'arrachage de cheveux à essayer de trouver le bon $\delta$ en fonction de $\epsilon$ si on n'a pas quelque théorèmes généraux. Nous allons donc maintenant en prouver quelque-uns.

\begin{theorem}		\label{ThoLimLinMul}
	Si
	\begin{equation} \label{Eqhypmullimlin}
	  \lim_{x\to a}f(x)=b,
	\end{equation}
	alors
	\begin{equation} \label{Eqbutmultlim}
	  \lim_{x\to a}(\lambda f)(x)=\lambda b
	\end{equation}
	pour n'importe quel $\lambda\in\eR$.
\end{theorem}

\begin{proof}
Soit $\epsilon>0$. Affin de prouver la propriété \eqref{Eqbutmultlim}, il faut trouver un $\delta$ tel que pour tout $x$ dans $[a-\delta,a+\delta]$, on ait $| (\lambda f)(x)- \lambda b |\leq\epsilon$. Cette dernière inégalité est équivalente à $|\lambda|| f(x)-b |\leq\epsilon$. Nous devons donc trouver un $\delta$ tel que 
\begin{equation} 
| f(x)-b |\leq\frac{ \epsilon }{ | \lambda | }.
\end{equation}
soit vraie pour tout $x$ dans $[a-\delta,a+\delta]$. Mais l'hypothèse \eqref{Eqhypmullimlin} dit précisément qu'il existe un $\delta$ tel que pour tout $x$ dans $[a-\delta,a+\delta]$ on ait cette inégalité. 
\end{proof}

\begin{theorem}		\label{ThoLimLin}
	Si
	\begin{subequations}
	\begin{align}
		\lim_{x\to a}f(x)&=b_1\\
		\lim_{x\to a}g(x)&=b_2,
	\end{align}
	\end{subequations}
	alors
	\begin{equation}
		\lim_{x\to a}(f+g)(x)=b_1+b_2.
	\end{equation}
\end{theorem}

\begin{proof}
	Soit $\epsilon>0$. Par hypothèse, il existe $\delta_1$ tel que
	\begin{equation}	\label{Eqfbunepsdeux}
	  | f(x)-b_1 |\leq \frac{ \epsilon }{ 2 }
	\end{equation}
	dès que $| x-a |\leq\delta_1$. Il existe aussi $\delta_2$ tel que 
	\begin{equation} 	\label{Eqgbdeuxepsdeux}
	  | g(x)-b_2 |\leq \frac{ \epsilon }{ 2 }.
	\end{equation}
	dès que $| x-a |\leq \delta_2$. Tu notes l'astuce de prendre $\epsilon/2$ dans la définition de limite pour $f$ et $g$. Maintenant, ce qu'on voudrait c'est un $\delta$ tel que l'on ait $| (f+g)(x)-(b_1+b_2) |\leq \epsilon$ dès que $| x-a |\leq \delta$. Moi je dit que $\delta=\min\{ \delta_1,\delta_2 \}$ fonctionne. En effet, en utilisant l'inégalité $| a+b |\leq | a |+| b |$, nous trouvons :
	\begin{align}
	| (f+g)(x)-(b_1+b_2) |=| (f(x)-b_1)+(g(x)-b_2) |
			\leq | f(x)-b_1 |+| g(x)-b_2 |.		\label{Eqfplusgfbun}
	\end{align}
	Comme on suppose que $| x-a |\leq\delta$, on a évidement $| x-a |\leq\delta_1$, et donc l'équation \eqref{Eqfbunepsdeux} tient. Mais si $| x-a |\leq\delta$, on a aussi $| x-a |\leq\delta_2$, et donc l'équation  \eqref{Eqfbunepsdeux} tient également. Chacun des deux termes de \eqref{Eqfplusgfbun} est donc plus petits que $\epsilon/2$, et donc le tout est plus petit que $\epsilon$, ce qu'il fallait montrer.

\end{proof}

Une formule qui résume ces deux théorèmes est que
\begin{equation}	\label{EqLimLinRes}
	\lim_{x\to a}[\alpha f(x)+\beta g(x)]=\alpha\lim_{x\to a}f(x)+\beta\lim_{x\to a}g(x).
\end{equation}

\begin{lemma}		\label{LemLimMajorableVois}
	Si $\lim_{x\to a}f(x)=b$ avec $a$, $b\in\eR$, alors il existe un $\delta>0$ et un $M>0$ tels que 
	\[ 
		(| x-a |\leq\delta)\Rightarrow | f(x) |\leq M.
	\]

\end{lemma}

Ce que signifie ce lemme, c'est que quand la fonction $f$ admet une limite finie en un point, alors il est possible de majorer la fonction sur un intervalle autour du point.

\begin{proof}
	Cela va être démontré par l'absurde. Supposons qu'il n'existe pas de $\delta$ ni de $M$ qui vérifient la condition. Dans ce cas, pour tout $\delta$ et pour tout $M$, il existe un $x$ tel que $| x-a |\leq\delta$ et $| f(x) |> M$. Cela est valable pour tout $M$, donc prenons par exemple $b+1000$. Donc 
	\begin{equation}
	\forall\delta>0,\exists x\text{ tel que } | x-a |\leq\delta\text{ et }| f(x) |>b+1000.
	\end{equation}
	Cela signifie qu'aucun $\delta$ ne peut convenir dans la définition de $\lim_{x\to a}f(x)=b$, ce qui contredit les hypothèses.
\end{proof}

Dans le même ordre d'idée, on peut prouver que si la limite de la fonction en un point est positive, alors elle est positive autour ce ce point. Plus précisément, nous avons la
\begin{proposition}	\label{PropoLimPosFPos}
	Si $f$ est une fonction telle que $\lim_{x\to a}f(x)>0$, alors il existe un voisinage de $a$ sur lequel $f$ est positive.
\end{proposition}	

\begin{proof}
	Supposons que $\lim_{x\to a}f(x)=y_0$. Par la définition de la limite fait que si pour tout $x$ dans un voisinage autour de $a$, on ait $| f(x)-a |<\epsilon$. Cela est valable pour tout $\epsilon$, pourvu que le voisinage soit assez petit. Si je choisit un voisinage pour lequel $| f(x)-a |<\frac{ y_0 }{ 2 }$, alors sur ce voisinage, $f$ est positive.
\end{proof}

Cette propoition ne devrait pas être sans te rappeler le théorème \ref{ThoValInter} des valeurs intermédiaires. Et en effet, c'est la même idée : si on sait que la fonction est positive en un point, on en déduit la positivité autour du point. Cette proposition est toutefois un peu plus forte parce que l'on ne suppose pas que la fonction soit continue.

\begin{theorem}		\label{Tholimfgabab}
	Si
	\begin{align}
		\lim_{x\to a}f(x)&=b_1&\text{et}&&\lim_{x\to a}g(x)=b_2,
	\end{align}
	alors
	\begin{equation}
		\lim_{x\to a}(fg)(x)=b_1b_2.
	\end{equation}
\end{theorem}

\begin{proof}
	Soit $\epsilon>0$, et tentons de trouver un $\delta$ tel que $| f(x)g(x)-b_1b_2 |\leq \epsilon$ dès que $| x-a |\leq \delta$. Nous avons 
	\begin{equation}	\label{EqfgbunbdeuxMin}
	\begin{split}
	| f(x)g(x)-b_1b_2 |&=|  f(x)g(x)-b_1b_2 +f(x)b_2-f(x)b_2 |\\
			&=\left|   f(x)\big( g(x)-b_2 \big)+b_2\big( f(x)-b_1 \big)    \right|\\
			&\leq \left|  f(x)\big( g(x)-b_2 \big)  \right|+\left|  b_2\big( f(x)-b_1 \big)    \right|\\
			&= | f(x) | | g(x)-b_2  |+| b_2 | |f(x)-b_1 |.	
	\end{split}
	\end{equation}
	À la première ligne se trouve la subtilité de la démonstration : on ajoute et on enlève\footnote{Comme exercice, tu peux essayer de refaire la démonstration en ajoutant et enlevant $g(x)b_1$ à la place.} $f(x)b_2$. Maintenant nous savons par le lemme \ref{LemLimMajorableVois} que pour un certain $\delta_1$, la quantité $| f(x) |$ peut être majoré par un certain $M$ dès que $| x-a |\leq \delta_1$. Prenons donc un tel $\delta_1$ et supposons que $| x-a |\leq \delta_1$. Nous savons aussi que pour n'importe quel choix de $\epsilon_2$ et $\epsilon_3$, il existe des nombres $\delta_2$ et $\delta_3$ tels que $| f(x)-b_1 |\leq \epsilon_2$ et $| g(x)-b_1 |\leq \epsilon_3$ dès que $| x-a |\leq\delta_2$ et $| x-a |\leq\delta_3$. Dans ces conditions, la dernière expression \eqref{EqfgbunbdeuxMin} se réduit à
	\begin{equation}
	| f(x)g(x)-b_1b_2 |\leq M\epsilon_2+| b_2 |\epsilon_3.
	\end{equation}
	Pour terminer la preuve, il suffit de choisir $\epsilon_2$ et $\epsilon_3$ tels que $M\epsilon_2+| b_2 |\epsilon_3\leq\epsilon$, et puis prendre $\delta=\min\{ \delta_1,\delta_2,\delta_3 \}$.


	Remetons les choses dans l'ordre. L'on se donne $\epsilon$ au départ. La première chose est de trouver un $\delta_1$ qui permet de majorer $|f(x)|$ par $M$ selon le lemme \ref{LemLimMajorableVois}, et puis choisissons $\epsilon_2$ et $\epsilon_3$ tels que $M\epsilon_2+| b_2 |\epsilon_3\leq\epsilon$. Ensuite nous prenons, en vertu des hypothèses de limites pour $f$ et $g$, les nombres $\delta_2$ et $\delta_3$ tels que $| f(x)-b_1 |\leq \epsilon_2$ et $| g(x)-b_2 |\leq \epsilon_3$ dès que $| x-a |\leq \delta_2$ et $| x-a |\leq \delta_3$.

	Si avec tous ça on prend $\delta=\min\{ \delta_1,\delta_2,\delta_3 \}$, alors la majoration et les deux inégalités sont valables en même temps et au final
	\[ 
	  | f(x)g(x)-b_1b_2 |\leq M\epsilon_2+b_2\epsilon_3\leq \epsilon,
	\]
	ce qu'il fallait prouver.

\end{proof}

À l'aide de ces petits résultats, nous pouvons déjà calculer pas mal de limites. Nous pouvons déjà par exemple calculer les limites de tous les polynomes en tous les nombrs réels. En effet, nous savons la limite de la fonction $f(x)=x$. la fonction $x\mapsto x^2$ n'est rien d'autre que le produit de $f$ par elle-même. Donc
\[ 
  \lim_{x\to a}x^2=\big( \lim_{x\to a}x\big)\cdot\big( \lim_{x\to a}x \big)=a^2.
\]
De la même façon, nous trouvons facilement que 
\begin{equation}
 \lim_{x\to a}x^n=a^n.
\end{equation}

\begin{exercice}
En continuant ce petit jeu, prouvez que $\lim_{x\to a}(x^3-5x+8)=a^3-5a+8$, et puis que si $P(x)$ est n'importe quel polynome, alors $\lim_{x\to a}P(x)=P(a)$.
\end{exercice}

Là, tu as un peu l'impression que l'on dit à tous les coups que $\lim_{x\to a}f(x)=f(a)$. D'ailleurs si tu relis la définition \ref{DefLimPointSansInfini}, tu vois que c'est un peu logique que la limite d'une fonction en un point soit la valeur de la fonction en ce point. Qu'est-ce qu'on a gagné alors ?

D'abord, est-ce que c'est toujours vrai que $\lim_{x\to a}f(x)=a$ ? Pour toute fonction ? Vraimment ? Eh bien non. Cela n'est pas vrai pour toutes les fonctions. Ce n'est vrai que pour une catégorie particulière de fonctions. L'important théorème suivant nous dit sans surprises pour quelles fonctions c'est vrai.

\begin{theorem}[Limite et continuité]			\label{ThoLimCont}
La fonction $f$ est continue au point $a$ si et seulement si $\lim_{x\to a}f(x)=f(a)$.
\end{theorem}

\begin{proof}
Nous commençons par supposer que $f$ est continue en $a$, et nous prouvons que $\lim_{x\to a}f(x)=a$. Soit $\epsilon>0$; ce qu'il nous faut c'est un $\delta$ tel que $| x-a |\leq\delta$ implique $| f(x)-f(a) |\leq\epsilon$. Relis la définition \ref{DefContinue} de la continuité, et tu verras que l'hypothèse de continuité est \emph{exactement} l'existence d'un $\delta$ comme il nous faut.

Dans l'autre sens, c'est à dire prouver que $f$ est continue au point $a$ sous l'hypothèse que $\lim_{x\to a}f(x)=f(a)$, la preuve se fait de la même façon.
\end{proof}

Nous en déduisons que si nous voulons gagner quelque chose à parler de limites, il faut prendre des fonctions non continues. Prenons une fonction qui fait un saut comme celle tracée à la figure \ref{subFigdiscontpasC}. Pour se fixer les idées, prenons celle-ci :
\begin{equation}
f(x)=
\begin{cases}
2x&\text{si $x\in]\infty,2[$}\\
x/2&\text{si $x\in[2,\infty[$}
\end{cases}
\end{equation}  
qui est représentée à la figure \ref{FigFnDiscDeux}. Essayons de trouver la limite de cette fonction lorsque $x$ tend vers $2$.
\begin{figure}
\centering
\begin{pspicture}(-1,-1)(8,5)
   %\psframe[linecolor=cyan](-1,-1)(8,5)
   \psset{PointSymbol=none,PointName=none}
	
	\psaxes(0,0)(-0.9,-0.9)(7.9,4.9)
   \def\Fn{2 x mul}
   \newcommand{\Gn}{x 2 div}
	\psplot{-0.5}{2}{\Fn}
	\psplot{2}{7}{\Gn}
   \pstGeonode(2,4){A}(2,1){B}
	\psline[linestyle=dashed](A)(B)
	
	\pscircle[fillstyle=solid,fillcolor=white,linecolor=black](A){0.1}				
	\pscircle[fillstyle=solid,fillcolor=black,linecolor=black](B){0.1}			
\end{pspicture}
\caption{Une fonction discontinue en $2$.}  \label{FigFnDiscDeux}
\end{figure}
Étant donné que $f$ n'est pas continue en $2$, nous savons déjà que $\lim_{x\to 2}f(x)\neq f(2)$. Donc ce n'est pas $1$. Cette limite ne peut pas valoir $4$ non plus parce que si je prends n'importe quel $\epsilon$, la valeur de $f(2+\epsilon)$ est très proche de $2$, et donc ne peut pas s'approcher de $4$. En fait, tu peux facilement vérifier que \emph{aucun nombre ne vérifie la condition de limite pour $f$ en $2$}. Nous disons que la limite n'existe pas.

Pour résumer, les limites qui ne font pas intervenir l'infini ne servent à rien parce que
\begin{itemize}
\item si la fonction est continue, la limite est simplement la valeur de la fonction par le théorème \ref{ThoLimCont},
\item si la fonction fait un saut, alors la limite n'existe pas (nous n'avons pas prouvé cela en général, mais avoue que l'exemple est convainquant).
\end{itemize}
Nous avons même la proposition suivante :
\begin{proposition}		\label{PropExisteLimVql}
Si $f$ existe en $a$ (c'est à dire si $a\in\dom(f)$) et si $\lim_{x\to a}f(x)=b$, alors $f(a)=b$.
\end{proposition}

\begin{proof}
Du fait que $\lim_{x\to a}f(x)=b$, il découle que pour tout $\epsilon$, il existe un $\delta$ tel que $| x-a |\leq \delta$ implique $| f(x)-b |\leq \epsilon$. Il est évident que pour tout $\delta$, $| x-x |\leq \delta$, donc nous avons que 
\[ 
  | f(a)-b |\leq\epsilon
\]
pour tout $\epsilon$. Cela implique que $f(a)=b$.
\end{proof}
Notons toutefois que l'inverse de cette proposition n'est pas vraie : la figure \ref{FigFnDiscDeux} montre justement une fonction qui prend la valeur $1$ en $2$ sans que la limite en $2$ soit $1$. Quoi qu'il en soit, cette proposition achève de nous convaincre de l'inutilité d'étudier d'étudier les limites sans infinis : dès qu'on a une limite, à tous les coups c'est la valeur de la fonction \ldots heu \ldots en es-tu bien sûr ?


\subsection{Limites et prolongement}
%-----------------------------------

À tous les coups ? Non ! La proposition \ref{PropExisteLimVql} a une terrible limitation : il faut que la fonction existe au point considéré. Or si tu regardes bien la définition \ref{DefLimPointSansInfini}, tu verras que $\lim_{x\to a}f(x)$ peut très bien exister sans que $f(a)$ n'existe.

Voici maintenant le bonus dont on parlait à la page \pageref{PgBonusLimite}\ldots bon d'accord, nous sommes à la page \label{PgIciBonus}\pageref{PgIciBonus}; le bonus est un peu cher !

Reprenons l'exemple de la fonction \eqref{EqOrdiRefuse} que mon ordinateur refusait de calculer en zéro :
\begin{equation}
f(x)=\frac{ x+4 }{ 3x^2+10x-8 }=\frac{ x+4 }{ (x+4)\left( x-\frac{ 2 }{ 3 } \right) }.
\end{equation}
Cette fonction a une condition d'existence en $x=-4$. Et pourtant, tant que $x\neq 4$, cela a un sens de simplifier les $(x+4)$ et d'écrire
\[ 
  f(x)=\frac{ 1 }{ x-\frac{ 2 }{ 3 } }=\frac{ 3 }{ 3x-2 }.
\]
Étant donné que pour toute valeur de $x$ différente de $-4$, la fonction $f$ s'exprime de cette façon, nous avons que
\[ 
  \lim_{x\to -4}f(x)=\lim_{x\to -4}\left(\frac{ 3 }{ 3x-2 }\right).
\]
Oui, mais la fonction\footnote{Cette fonction $g$ n'est pas $f$ parce que $g$ a en plus l'avantage d'être définie en $-4$.} $g(x)=3/(3x-2)$ est continue en $-4$ et donc sa limite vaut sa valeur. Nous en déduisons que
\[ 
  \lim_{x\to -4}f(x)=-\frac{ 3 }{ 14 }.
\]
Que dire maintenant de la fonction ainsi définie ?
\begin{equation}
\tilde f(x)=
\begin{cases}
f(x)&\text{si $x\neq -4$}\\
-3/14&\text{si $x=-4$}.
\end{cases}
\end{equation}
Cette fonction est continue en $-4$ parce qu'elle y est égale à sa limite. Les étapes suivies pour obtenir ce résultat sont :
\begin{itemize}
\item Repérer un point où la fonction n'existe pas,
\item calculer la limite de la fonction en ce point, et en particulier vérifier que cette limite existe, ce qui n'est pas toujours le cas,
\item définir une nouvelle fonction qui vaut partout la même chose que la fonction originale, sauf au point considéré où l'on met la valeur de la limite.
\end{itemize}
\begin{figure}
\centering
   \psset{PointSymbol=none,PointName=none,xunit=1cm,yunit=1cm}
\begin{pspicture}(-5.5,-4)(5.5,4)
   	%\psframe[linecolor=cyan](-5.5,-4)(5.5,4)
	\psaxes(0,0)(-5,-3.9)(5,3.9)
   \newcommand{\Fn}{x 4 add  x 2 exp 3 mul x 10 mul add 8 sub div }
	\psplot[linecolor=blue]{-5}{0.57}{\Fn}
	\psplot[linecolor=blue]{0.75}{5}{\Fn}
	\psline[linecolor=red](0.583,-4)(0.6666,4)
\end{pspicture}
\caption{En $x=-4$, tu vois bien qu'il ne se passe en fait rien : on peut la prolonger. En $2/3$ par contre, elle part vers l'infini et il n'y a aucun espoir de la prolonger par continuité.}  \label{FigProloiuetnon}
\end{figure}
C'est ce qu'on appelle \defe{prolonger la fonction par continuité}{Prolongation par continuité} parce que la fonction résultante est continue. La prolongation de $f$ par continuité est donc en général définie par
\begin{equation}
\tilde f(x)=
\begin{cases}
f(x)			&\text{si $f(x)$ existe}\\
\lim_{y\to x}f(y)	&\text{si $f(x)$ si cette limite existe et est finie.}
\end{cases}
\end{equation}
Dans le cas que nous regardions (voir la figure \ref{FigProloiuetnon}),
\[ 
	f(x)=\frac{ x+4 }{ 3x^2+10x-8 },
\]
le prolongement par continuité est donné par
\begin{equation}
\tilde f =\frac{ 3 }{ 3x-2 }.
\end{equation}
Remarque que cette fonction n'est toujours pas définie en $x=2/3$. 


%+++++++++++++++++++++++++++++++++++++++++++++++++++++++++++++++++++++++++++++++++++++++++++++++++++++++++++++++++++++++++++
\section{Calcul de limites}
%+++++++++++++++++++++++++++++++++++++++++++++++++++++++++++++++++++++++++++++++++++++++++++++++++++++++++++++++++++++++++++

Un résultat pratique pour calculer des limites est la
\begin{proposition}		\label{PropChmVarLim}
Quand la limite existe, nous avons
\[ 
  \lim_{x\to a}f(x)=\lim_{\epsilon\to 0}f(a+\epsilon),
\]
ce qui correspond à un \og changement de variables\fg{} dans la limite.
\end{proposition}

\begin{proof}
Si $A=\lim_{x\to a}f(x)$, par définition,
\begin{equation}		\label{EqCondFaplusespLim}
\forall\epsilon'>0,\,\exists\delta\text{ tel que }| x-a |\leq\delta\Rightarrow| f(x)-A |\leq\epsilon'.
\end{equation}
La seule subtilité de la démonstration est de remarquer que si $| x-a |\leq\delta$, alors $x$ peut être écrit sous la forme $x=a+\epsilon$ pour un certain $| \epsilon |\leq\delta$. En remplaçant $x$ par $a+\epsilon$ dans la condition \ref{EqCondFaplusespLim}, nous trouvons 
\begin{equation}
\forall\epsilon'>0,\,\exists\delta\text{ tel que }| \epsilon |\leq\delta\Rightarrow| f(x+\epsilon)-A |\leq\epsilon',
\end{equation}
ce qui signifie exactement que $\lim_{\epsilon\to 0}f(x+\epsilon)=A$.	
\end{proof}

Il y a une petite différence de point de vue entre $\lim_{x\to a}f(x)$ et $\lim_{\epsilon\to 0}f(a+\epsilon)$. Dans le premier cas, on considère $f(x)$, et on regarde ce qu'il se passe quand $x$ se rapproche de $a$, tandis que dans le second, on considère $f(a)$, et on regarde ce qu'il se passe quand on s'éloigne un tout petit peu de $a$. Dans un cas, on s'approche très près de $a$, et dans l'autre on s'en éloigne un tout petit peu. Ces deux points de vue sont évidement équivalents, comme prouvé par la proposition \ref{PropChmVarLim}.

\Exo{212}

% Il y a des techniques de calcul de limites décrites sur le site
% http://bernard.gault.free.fr/terminale/limites/limite.html

%+++++++++++++++++++++++++++++++++++++++++++++++++++++++++++++++++++++++++++++++++++++++++++++++++++++++++++++++++++++++++++
					\section{Compacité}
%+++++++++++++++++++++++++++++++++++++++++++++++++++++++++++++++++++++++++++++++++++++++++++++++++++++++++++++++++++++++++++
%http://fr.wikipedia.org/wiki/Espace_compact
%http://fr.wikipedia.org/wiki/Théorème_de_Heine-Borel
%http://fr.wikipedia.org/wiki/Émile_Borel
%http://fr.wikipedia.org/wiki/Henri_Léon_Lebesgue

Aussi incroyable que cela puisse paraître, ce que nous allons faire va servir dans la démonstration de la formule \eqref{EqMRUAINT}. Soit $E$, un sous ensemble de $\eR$. Nous pouvons considérer les ouverts suivants : 
\begin{equation}
	\mO_x=B(x,1)
\end{equation}
pour chaque $x\in E$. Évidement,
\begin{equation}
	E\subseteq \bigcup_{x\in E}\mO_x.
\end{equation}
Cette union est très souvent énorme, et même infinie. Elle contient de nombreuses redondances. Si par exemple $E=[-10,10]$, l'élément $3\in E$ est contenu dans $\mO_{3.5}$, $\mO_{2.7}$ et bien d'autres. Pire : même si on enlève par exemple $\mO_2$ de la liste des ouverts, l'union de ce qui reste continue à être tout $E$. La question est : \emph{est-ce qu'on peut en enlever suffisamment pour qu'il n'en reste qu'un nombre fini ?}
\begin{definition}
Soit $E$, un sous ensemble de $\eR$. Une collection d'ouverts $\mO_i$ est un \defe{recouvrement}{Recouvrement} de $E$ si $E\subseteq \bigcup_{i}\mO_i$. Un sous ensemble $E$ de $\eR$ tel que de tout recouvrement par des ouverts, on peut extraire un sous-recouvrement fini est dit \defe{\href{http://fr.wikipedia.org/wiki/Espace_compact}{compact}}{Compact}.
\end{definition}

\begin{proposition}
Les ensembles compacts sont fermés et bornés.
\end{proposition}

\begin{proof}
Prouvons d'abord qu'un ensemble compact est borné. Pour cela, supposons que $K$ est un compact non borné vers le haut\footnote{Nous laissons à titre d'exercice le cas où $K$ est borné par le haut et pas par le bas.}. Donc il existe une suite infinie de nombres strictement croissante $x_1<x_2<\ldots$ tels que $x_i\in K$. Prenons n'importe quel recouvrement ouvert de la partie de $K$ plus petite ou égale à $x_1$, et complétons ce recouvrement par les ouverts $\mO_i=]x_{i-1},x_i[$. Le tout forme bien un recouvrement de $K$ par des ouverts. 

Il n'y a cependant pas moyen d'en tirer un sous recouvrement fini parce que si on ne prends qu'un nombre fini parmi les $\mO_i$, on en aura fatalement un maximum, disons $\mO_k$. Dans ce cas, les points $x_{k+1}$, $x_{k+1}$,\ldots ne seront pas dans le choix fini d'ouverts.

Cela prouve que $K$ doit être borné.

Pour prouver que $K$ est fermé, nous allons prouver que le complémentaire est ouvert. Et pour cela, nous allons prouver que si le complémentaire n'est pas ouvert, alors nous pouvons construire un recouvrement de $K$ dont on ne peut pas extraire de sous recouvrement fini.

Si $\eR\setminus K$ n'est pas ouvert, il possède un point, disons $x$, tel que tout voisinage de $x$ intersecte $K$. Soit $B(x,\epsilon_1)$, un de ces voisinages, et prenons $k_1\in K\cap B(x,\epsilon_1)$. Ensuite, nous prenons $\epsilon_2$ tel que $k_1$ n'est pas dans $B(x,\epsilon_1)$, et nous choisissons $k_2\in K\cap B(x,\epsilon_2)$. De cette manière, nous construisons une suite de $k_i\in K$ tous différents et de plus en plus proches de $x$. Prenons un recouvrement quelconque par des ouverts de la partie de $K$ qui n'est pas dans $B(x,\epsilon_1)$. Les nombres $k_i$ ne sont pas dans ce recouvrement.

Nous ajoutons à ce recouvrement les ensembles $\mO=]k_i,k_{i+1}[$. Le tout forme un recouvrement (infini) par des ouverts dont il n'y a pas moyen de tirer un sous recouvrement fini, pour exactement la même raison que la première fois.
\end{proof}

Le résultat suivant le théorème de \href{http://fr.wikipedia.org/wiki/Théorème_de_Heine-Borel}{Borel-Lebesgue}, et la démonstration vient de wikipédia.
\begin{theorem}[\href{http://fr.wikipedia.org/wiki/Émile_Borel}{borel}-\href{http://fr.wikipedia.org/wiki/Henri_Léon_Lebesgue}{Lebesgue}]	\label{ThoBOrelLebesgue}
	Les intervalles de la forme $[a,b]$ sont compacts.
\end{theorem}

\begin{proof}
	Soit $\Omega$, un recouvrement du segment $[a,b]$ par des ouverts, c'est à dire que
	\begin{equation}
		[a,b]\subseteq\bigcup_{\mO\in\Omega}\mO.
	\end{equation}
	Nous notons par $M$ le sous-ensemble de $[a,b]$ des points $m$ tels que l'intervalle $[a,m]$ peut être recouvert par un sous-ensemble fini de $\Omega$. C'est à dire que $M$ est le sous ensemble de $[a,b]$ sur lequel le théorème est vrai. Le but est maintenant de prouver que $M=[a,b]$.
	\begin{description}
		\item[$M$ est non vide] En effet, $a\in M$ parce que il existe un ouvert $\mO\in\Omega$ tel que $a\in\mO$. Donc $\mO$ tout seul recouvre l'intervalle $[a,a]$. 
		\item[$M$ est un intervalle] Soient $m_1$, $m_2\in M$. Le but est de montrer que si $m'\in[m_1,m_2]$, alors $m'\in M$. Il y a un sous recouvrement fini de l'intervalle $[a,m_2]$ (par définition de $m_2\in M$). Ce sous recouvrement fini recouvre évidement aussi $[a,m']$ parce que $[a,m']\subseteq [a,m_2]$, donc $m'\in M$.
		\item[$M$ est une ensemble ouvert] Soit $m\in M$. Le but est de prouver qu'il y a un ouvert autour de $m$ qui est contenu dans $M$. Mettons que $\Omega'$ soit un sous recouvrement fini qui contienne l'intervalle $[a,m]$. Dans ce cas, on a un ouvert $\mO\in\Omega'$ tel que $m\in\mO$. Tous les points de $\mO$ sont dans $M$, vu qu'ils sont tous recouverts par $\Omega'$. Donc $\mO$ est un voisinage de $m$ contenu dans $M$.
		\item[$M$ est un ensemble fermé] $M$ est un intervalle qui commence en $a$, en contenant $a$, et qui finit on ne sait pas encore où. Il est donc soit de la forme $[a,m]$, soit de la forme $[a,m[$. Nous allons montrer que $M$ est de la première forme en démontrant que $M$ contient son supremum $s$. Ce supremum est un élément de $[a,b]$, et donc il est contenu dans un des ouverts de $\Omega$. Disons $s\in\mO_s$. Soit $c$, un élément de $\mO_s$ strictement plus petit que $c$; étant donné que $s$ est supremum de $M$, cet élément $c$ est dans $M$, et donc on a un sous recouvrement fini $\Omega'$ qui recouvre $[a,c]$. Maintenant, le sous recouvrement constitué de $\Omega'$ et de $\mO_s$ est fini et recouvre $[a,s]$.
	\end{description}
	Nous pouvons maintenant conclure : le seul intervalle non vide de $[a,b]$ qui soit à la fois ouvert et fermé est $[a,b]$ lui-même, ce qui prouve que $M=[a,b]$, et donc que $[a,b]$ est compact.
\end{proof}
Note : il est également vrai que \emph{tous} les compacts de $\eR$ sont fermés et bornés, mais nous n'allons pas démontrer cela ici.

Une propriété très importante des compacts est la suivante :
\begin{theorem}		\label{ThoImCompCotComp}
L'image d'un compact par une fonction continue est un compact
\end{theorem}

\begin{proof}
	Soit $K\subset \eR$, un ensemble compact, et regardons $f(K)$; en particulier, nous considérons $\Omega$, un recouvrement de $f(K)$ par des ouverts. Nous avons que
	\begin{equation}
		f(K)\subseteq\bigcup_{\mO\in\Omega}\mO.
	\end{equation}
	Par construction, nous avons aussi
	\begin{equation}
		K\subseteq\bigcup_{\mO\in\Omega}f^{-1}(\mO),
	\end{equation}
	en effet, si $x\in K$, alors $f(x)$ est dans un des ouverts de $\Omega$, disons $f(x)\in \mO_0$, et évidemment, $x\in f^{-1}(\mO)$.  Les $f^{-1}(\mO)$ recouvrent le compact $K$, et donc on peut en choisir un sous-recouvrement fini, c'est à dire un choix de $\{ f^{-1}(\mO_1),\ldots,f^{-1}(\mO_n) \}$ tels que
	\begin{equation}
		K\subseteq \bigcup_{i=1}^nf^{-1}(\mO_i).
	\end{equation}
	Dans ce cas, nous avons que
	\begin{equation}
		f(K)\subseteq\bigcup_{i=1}^n\mO_i,
	\end{equation}
	ce qui prouve la compacité de $f(K)$.
\end{proof}

Par le théorème des valeurs intermédiaires, l'image d'un intervalle par une fonction continue est un intervalle, et nous avons l'importante propriété suivante des fonctions continues sur un compact.

\begin{theorem}
	Si $f$ est une fonction continue sur l'intervalle compact $[a,b]$. Alors $f$ est bornée sur $[a,b]$ et elle atteint ses bornes.
\end{theorem}

\begin{proof}
	Étant donné que $[a,b]$ est un intervalle compact, son image est également un intervalle compact, et donc est de la forme $[m,M]$. Ceci découle du théorème \ref{ThoImCompCotComp} et le corollaire \ref{CorImInterInter}. Le maximum de $f$ sur $[a,b]$ est la borne $M$ qui est bien dans l'image (parce que $[m,M]$ est fermé). Idem pour le minimum $m$.
\end{proof}

En préparant ces notes, j'ai fait une très jolie faute en essayant de prouver ce théorème\footnote{Qui ne fait pas de fautes en tapant à l'ordi couché sur son lit entre une et deux heures du matin ?}. Voici comment je comptait prouver le fait que $f$ est bornée. Supposons que $f$ ne soit bornée ni vers le haut, ni vers le bas, donc son image est $\eR$ qui est ouvert. Mais $f$ est continue sur $[a,b]$, donc l'image inverse de $\eR$ par $f$ doit être un ouvert. Oui, mais cette image inverse est exactement $[a,b]$ qui est fermé. Cela est une contradiction qui prouve que $f$ doit être bornée au moins soit vers le haut, soit vers le bas.

Peux-tu trouver la faute ?


\section{A mad tea party}	\label{PgMadTeaParty}
%++++++++++++++++++++++++

\textit{\og Reprenez donc un peu de thé\fg{} propose le Lièvre de Mars.}

\textit{\og Je n'ai rien pris du tout, je ne saurai donc reprendre de rien !\fg}

\textit{\og Vous voulez dire que vous ne sauriez reprendre de quelque chose\fg{} repartit le Chapelier.\\
 \og Quand il n'y a rien, ce n'est pas
facile d'en reprendre\fg.}

 \begin{itemize}

 \item Alors comme ça, vous êtes \href{http://fr.wikipedia.org/wiki/Manifestations_de_la_place_Tian'anmen}{étudiante}~?
 \item Oui, en mathématiques par exemple.
 \item Alors que vaut cette fraction : un sur deux sur trois sur quatre~?
 \item Eh bien ...
 \item Elle vaut deux tiers, la devança le Loir.
 \item Ou trois huitièmes si vous préférez, ajouta le Lièvre de Mars.
 \item Ou encore un sur vingt-quatre, affirma le  \href{http://fr.wikipedia.org/wiki/http://fr.wikipedia.org/wiki/Chapelier_fou_(Alice_au_pays_des_merveilles)}{Chapelier}.
 \item En fait, je crois que...
\item Aucune importance ! Dites-nous plutôt combien vous voulez de sucre dans votre thé~?
\item Deux ou trois, ça dépend de la taille de la tasse.
\item Certainement pas, car de toute façon, deux ou trois c'est pareil.
\item Parfaitement~! approuva le Loir en fixant Alice qui écarquillait les yeux.
\item Ce n'est pourtant pas ce qu'on m'a appris, fit celle-ci.
\item Pourtant, ce n'est pas compliqué à comprendre, en voici une démonstration des plus élémentaires. On sait que pour tout entier $n$ on a successivement
                \[ 
			(n+1)^2=n^2+2n+1
		\]
                \[
			(n+1)^2-2n-1=n^2
		\]
	Retranchons $n(2n+1)$ des deux côtés
                \[
			(n+1)^2-(n+1)(2n+1)=n^2-n(2n+1).
		\]
	Mézalor, en ajoutant $(2n+1)^2/4$, on obtient
		\[ 
                	(n+1)^2-(n+1)(2n+1)+\frac{(2n+1)^2}{4}=n^2-n(2n+1)+\frac{(2n+1)^2}{4}
		\]
	Soit
		\[
	                \left((n+1)-\frac{2n+1}{2}\right)^2=\left(n-\frac{2n+1}{2}\right)^2
		\]
	En passant à la racine carrée, on obtient
		\[ 
			(n+1)-\frac{2n+1}{2}=n-\frac{2n+1}{2}
		\]
	d'où
		\[ 
			n+1=n
		\]
Et si je prends $n=2$, j'ai aussitôt $3=2$
\item Alors, qu'est-ce que vous en dites~?
\item Je\ldots commença Alice.
\item D'ailleurs, cela prouve que tous les entiers sont égaux, la coupa le Lièvre de Mars.
\item Pas mal du tout ! Qu'en dites-vous mademoiselle la mathématicienne~?
\item Je vais vous dire tout de suite ce que j'en pense
\item Ah non ! Nous préférerions de loin que vous pensiez ce que vous allez nous dire.
\item C'est pareil ! grinça Alice qui commençait à en avoir assez.
\item Comment ça, c'est pareil~? Dire ce que l'on pense ce serait pareil que penser ce que l'on dit~? s'étrangla le Lièvre de Mars.
\item Incroyable ! Et manger ce qu'on voit ce serait pareil que voir ce qu'on mange~?
\item Mais...
\item Et respirer quand on dort pareil que dormir quand on respire~?
\item En logique, nous vous mettons 3 sur 5.
\item Autant dire moins que un.
\item C'est à dire zéro, puisque si $2=3$ alors $1=0$.
\item Parce que chez vous, 3 c'est moins que 1 ? s'indigna Alice.
\item On se demande ce qu'on vous apprend à l'école ! Bien sûr que oui ! Tenez, considérez
		\[ 
			f(x)=\frac{x^2+32}{2x^2+1}+\frac{|x|+1}{2x+51}
		\]
Eh bien il est facile de voir que cette fonction a pour limite $0$ en moins l'infini et $1$ en plus l'infini.
\item Je ne dis pas le contraire, protesta Alice.
\item Donc l'image par $f$ de $\eR$ est l'intervalle $]0,1[$, or $f(0)=3$, donc $3$ appartient à $]0,1[$ à ce titre : on a bien $3$ plus petit que $1$.
\item C'est de la folie pure, pensa Alice\ldots
\end{itemize}


\begin{remark}
Cette partie de thé de fous est volée avec quelque modifications mineures (dont les liens) et sans en avoir honte de la série de cours de math de Guillaume Connan, \href{http://gconnan.free.fr/}{Tehessin le rezeen}. À lire sans hésiter ! Le livre original de \href{http://fr.wikipedia.org/wiki/Lewis_Carroll}{Lewis Carol} d'où est tiré un très bon dessin animé est également à lire sans hésiter.
\end{remark}




