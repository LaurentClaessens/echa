% This is part of Un soupçon de physique, sans être agressif pour autant
% Copyright (C) 2006-2009
%   Laurent Claessens
% See the file fdl-1.3.txt for copying conditions.


%Copyright (c) 2006 Claessens Laurent. Permission is granted to copy, distribute and/or modify this document under the terms of the  GNU Free Documentation License, Version 1.2 or any later version published by the Free Software Foundation; with no Invariant Sections, no Front-Cover Texts, and no Back-Cover Texts. A  copy of the license is included in the section entitled "GNU Free Documentation License".

\begin{figure}[h]
\centering
\begin{pspicture}(-0.5,-0.5)(4,2.5)
  \psset{PointSymbol=none, PointName=none}
\prefigzerounneuf
   \psline(A)(C)
   \psline(C)(B)
   \psline(A)(B)
  
   \pstMarqueForce{Cc}{bG}{0.4;45}{$\fG$}

\pstInterLL{Cc}{bG}{A}{B}{perc}
\pstInterLL{Cc}{bG}{A}{C}{percd}

\pstTransHom{perc}{B}{Cc}{1}{cAB}
\pstTransHom{perc}{B}{Cc}{-1}{cBA}

   \pstMarqueForce{Cc}{cAB}{0.3;90}{$A\to B$}
   \pstMarqueForce{Cc}{cBA}{0.5;90}{$B\to A$}

\pstMarkAngle{C}{A}{B}{$\alpha$}
\pstMarkAngle[linecolor=blue]{A}{perc}{percd}{$\beta$}
\pstMarkAngle[linecolor=red]{percd}{perc}{B}{$\gamma$}

\end{pspicture}
\caption{Angles entre la gravitation et les déplacements}\label{fig_corrchar}
\end{figure}


\begin{corrige}{019}


Tu dois aller voir la figure \ref{fig:Force_decomp} pour voir comment il faut traiter avec les problèmes de décomposition de forces sur un plan incliné.

La force $\fR$ est perpendiculaire au déplacement (que ce soit pour $A$ vers $B$ ou $B$ vers $A$), donc son travail est toujours nul.

La force $\fF$ est toujours parallèle au déplacement. Durant le déplacement de $A$ vers $B$, elle est dans le même sens et donc $W_F^{A\to B}=F| AB |$; durant le déplacement de $B$ vers $A$, elle est dans le sens inverse et donc $W_F^{B\to A}=-F| AB |$.

Sur la figure \ref{fig_corrchar}, l'angle $\beta$ entre $\fG$ et le déplacement $B\to A$ vaut $90-\alpha$ tandis que l'angle $\gamma$ entre $\fG$ et le déplacement $A\to B$ vaut $180-(90-\alpha)=90+\alpha$.
Durant le déplacement $A\to B$, le travail de $\fG$ vaut donc $G\cdot| AB |\cos\gamma=-G\cdot| AB |\sin(\alpha)=-G| BC |$ parce que $\cos(90+\alpha)=-\sin(\alpha)$.
Durant le déplacement $B\to A$, le travail de $\fG$ vaut 
\[
G\cdot| AB |\cos(90-\alpha)=G\cdot| AB |\sin(\alpha)=G\cdot| BC |.
\]

Lorsque $\| \fF \|=\| \fG \|$, le chariot, l'objet monte. En effet, seule le composante parallèle de $\fG$ n'est utile pour faire descendre le chariot, mais chaque composante d'une force est toujours plus petite que la force elle-mêle, c'est à dire que $\| \fG_{\parallel} \|<\| \fG \|$.

\end{corrige}
