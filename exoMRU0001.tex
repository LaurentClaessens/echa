% This is part of Un soupçon de physique, sans être agressif pour autant
% Copyright (C) 2006-2009
%   Laurent Claessens
% See the file fdl-1.3.txt for copying conditions.

\begin{exercice}\label{exoMRU0001}

	Supposons que le boulevard de la Plaine soit long de $\unit{800}{\meter}$. Un cycliste le descend à $\unit{30}{\kilo\meter\per\hour}$ tandis qu'une voiture arrive à\footnote{Oui, nous sommes d'accord, c'est un non-sens total d'utiliser son moteur dans une descente. Mais si la personne avait deux sous d'intelligence et de conscience écologique, il ne serait pas en voiture hein; rien d'étonnant donc.} $\unit{50}{\kilo\meter\per\hour}$ et le rejoint à la moitié du boulevard. À ce moment, il a l'idée de génie de klaxonner\footnote{Malgré qu'il soit interdit de klaxonner en agglomération. Mais encore une fois, si cette personne réfléchissait à ses actes, il ne serait pas en voiture hein; toujours rien d'étonnant.}.

	Calculer combien de temps on peut gagner en roulant à $\unit{50}{\kilo\meter\per\hour}$ plutôt que $30$ sur la distance qu'il restait.

\corrref{MRU0001}
\end{exercice}
