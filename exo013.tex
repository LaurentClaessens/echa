% This is part of Un soupçon de physique, sans être agressif pour autant
% Copyright (C) 2006-2009
%   Laurent Claessens
% See the file fdl-1.3.txt for copying conditions.




\begin{figure}[ht]
\centering
\begin{pspicture}(-0.5,-2)(4,2)
   \psset{PointSymbol=none, PointName=none}
   \prefigzerotreize
\pstMarqueForce{O}{A}{0.5;30}{$\fF_1$}  
\pstMarqueForce{O}{B}{0.5;0.3}{$\fF_2$}  
\pstMarqueForce{O}{C}{0.5;0.3}{$\fF_3$}  
%   \psline{->}(O)(A)
%   \psline{->}(O)(B)
%   \psline{->}(O)(C)
%   \rput(A){\rput(0.5;30){$\fF_1$}}
%   \rput(B){\rput(0.5;0.3){$\fF_2$}}
%   \rput(C){\rput(0.5;0.3){$\fF_3$}}
\end{pspicture}
\caption{Exercice de composition de forces}\label{fig:exo:comp}
\end{figure}

\begin{exercice} \label{exo013}

Sur la figure \ref{fig:exo:comp},
\begin{enumerate}
\item composez les trois forces,
\item décomposez la force $\fF_1$ en les deux autres.
\end{enumerate}

\corrref{013}
\end{exercice}
