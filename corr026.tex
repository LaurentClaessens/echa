% This is part of Un soupçon de physique, sans être agressif pour autant
% Copyright (C) 2006-2009
%   Laurent Claessens
% See the file fdl-1.3.txt for copying conditions.


\begin{corrige}{026}
Le frein doit appliquer une force parallèle à la pente, dirigée vers le haut, sinon la viture la dévalerait.  Une telle force est donc indispensable pour le faire tenir en équilibre. Regardons la figure \ref{fig:chariotdecomp}. Le fait que la pente soit de $5$ \%  signifie que $AB=\unit{100}{\meter}$ et $BC=\unit{4}{\meter}$, c'est à dire que $\sin(\widehat{BAC})=5/100$. 

Disons que $g=\unit{10}{\meter\per\second\squared}$ pour avoir des nombres ronds.

Nous savons que la partie de $\fG$ perpendiculaire au plan incliné sera compensée par la réaction du plan. On ne s'y intéresse donc pas. En ce qui concerne la partie parallèle au plan incliné de la force de gravitation, la formule \eqref{eq_Stevinpapa} nous enseigne que
\[ 
  G_{\parallel}=G\sin(\widehat{BAC})=\unit{5000\cdot 5/100=250}{\newton},
\]
et est dirigée vers le bas. La force que l'on cherche, vaut la même chose en norme, mais est dirigée vers le haut.
 
 
\end{corrige}
