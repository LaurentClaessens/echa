% This is part of Un soupçon de physique, sans être agressif pour autant
% Copyright (C) 2006-2010
%   Laurent Claessens
% See the file fdl-1.3.txt for copying conditions.

\section{Matériel utilisé}

%---------------------------------------------------------------------------------------------------------------------------
\subsection{Logiciels}
%---------------------------------------------------------------------------------------------------------------------------

\begin{enumerate}
	\item
		Linux, Ubuntu Lucid
	\item
		\sout{Emacs et Gnome} Vim et KDE
	\item
		\AmS-\LaTeX
	\item
		\href{www.sagemath.org}{Sage} pour beaucoup de calculs, la préparation de graphiques, et pour les dernières figures.
	\item 
		\href{http://maxima.sourceforge.net/}{wxMaxima} avant que je ne découvre Sage.
	\item 
		\href{http://fr.wikipedia.org/wiki/PSTricks}{pstricks} pour les figures,
	\item
		\href{http://fr.wikipedia.org/wiki/Python_(langage)}{python} et \href{http://student.ulb.ac.be/~lclaesse/phystricks-doc.pdf}{phystricks} pour la programmation et les figures.
	\item
		Git et \href{www.gitorious.org}{gitorious} pour le partage des sources.
\end{enumerate}
Je tiens à en remercier tous les développeurs.
		
%---------------------------------------------------------------------------------------------------------------------------
\subsection{Sources}
%---------------------------------------------------------------------------------------------------------------------------

\begin{itemize}
	\item
		Le cours \emph{Mécanique et énergie} du \og site de la famille GG\fg. \href{www.cvgg.org}{www.cvgg.org}.
	\item 
		\href{http://gconnan.free.fr/}{Le site de Guillaume Connan}\footnote{À qui j'aurais pu téléphoner, mais\ldots}, et les questions pas toujours idiotes que Tehessin le rezeen y pose à son ô maître Mathémator. Bourré d'humour et d'excellentes explications mathématiques. La mad tea party de la page \pageref{PgMadTeaParty} en est l'emprunt le plus direct.
	\item 
		Le forum usenet de physique\footnote{\href{http://groups.google.fr/group/fr.sci.physique/topics?hl=fr.}{http://groups.google.fr/group/fr.sci.physique/topics?hl=fr.}} pour la linéarité des transformations de Lorentz et une fructueuse discussion sur la maximalité de la vitesse de la lumière. 
	\item 
		Wikipédia pour des liens hypertexte et pour le second degré.
	\item 
		Des discutions avec diverses personnes ayant lu des parties du texte.
	\item 
		les questions de mes élèves particuliers,\ldots
	\item 
		\ldots des idées glanées par-ci par-là dans les cours ---parfois très bien faits--- de leurs profs. Que ceux qui se sentent visés soient remerciés de mettre des cours de qualité entre les mains des enfants; qu'ils soient consolés que lesdits élèves parviennent encore à se faire morfler et à avoir besoin de cours particuliers; et qu'ils n'hésitent pas à poster sur \href{http://www.enseignons.be/}{http://www.enseignons.be/}.
\end{itemize}

Je m'en voudrais de ne pas citer le cours de math libre sesamath\footnote{\href{http://manuel.sesamath.net/}{http://sesamath.net/}}, dont je ne me suis pas inspiré de façon directe, mais qui peut quand même intéresser la lectrice de ces lignes.


%+++++++++++++++++++++++++++++++++++++++++++++++++++++++++++++++++++++++++++++++++++++++++++++++++++++++++++++++++++++++++++
					\section{Un cours de physique libre}
%+++++++++++++++++++++++++++++++++++++++++++++++++++++++++++++++++++++++++++++++++++++++++++++++++++++++++++++++++++++++++++

Ces notes ainsi que les sources \LaTeX{} sont librement publiées sur internet, cela signifie entre autres que chaque lectrice ou lecteur a le droit par tous les moyens de copier, modifier et redistribuer tout ou une partie du texte. En cela, je voudrais utiliser la puissance de l'internet pour mettre à portée de toutes et tous certaines connaissances en physique trop souvent confinées dans des manuels scolaires forts coûteux. J'espère qu'à l'avenir, la publication de texte libres d'enseignements fera à substantiellement baisser le prix de la scolarité.

Le partage est ce qu'on a trouvé de mieux pour propager la connaissance. De nombreuses autres initiatives internautes suivent cette idée. Citons Linux\footnote{\href{www.ubuntu.org}{www.ubuntu.org}} (qui fournit des tonnes de logiciels dans un esprit de liberté), l'encyclopédie libre Wikipedia\footnote{\href{www.wikipedia.org}{www.wikipedia.org}} et le (moins connu mais non moins intéressant) projet mutopia\footnote{\href{www.mutopiaproject.org}{www.mutopiaproject.org}} qui a pour ambition de publier sous format électronique et libre des partitions musicales tombées dans le domaine public (avis aux fans de musique). La plupart des sources données plus haut sont également publiées dans un esprit de partage libre.


%+++++++++++++++++++++++++++++++++++++++++++++++++++++++++++++++++++++++++++++++++++++++++++++++++++++++++++++++++++++++++++
					\section{Originalité de ce texte}
%+++++++++++++++++++++++++++++++++++++++++++++++++++++++++++++++++++++++++++++++++++++++++++++++++++++++++++++++++++++++++++

Le peu d'expérience de l'enseignement dont l'auteur de ces lignes peut se vanter sont trois cessions d'août d'Échec à l'echec et un bon nombre de cours particuliers, presque toujours entre la quatrième et la sixième secondaire. Les moments qui font plaisir, c'est quand un élève dit \og eh bien ! en fait c'est pas n'importe quoi la physique, maintenant que je comprends comment ça marche c'est chouette \fg.
   
J'ai donc développé une série de théories personnelles sur la façon dont il faudrait enseigner la physique et la mathématique. Je me suis donc mit à taper à l'ordinateur les mots que j'aurais dit si j'avais à expliquer oralement.

Parmi mes idées sur l'enseignement de la physique, celles qui diffèrent des cours \og traditionnels\fg{} sont plus ou moins celles-ci :
\begin{itemize}
\item je donne une préférence aux mots plutôt qu'aux formules : il ne s'agit pas de retenir que $P=F/S$, mais que \og Pression égal force divisé par surface\fg.
\item des exemples, des exemples, des exemples et encore des exemples, (sur ce point je ne suis pas encore du tout satisfait de mon résultat)
\item de l'humour, c'est marrant et ça détend l'atmosphère,
\item une définition peut avoir deux utilités : l'utilité première est d'être sûr que deux personnes parlent de la même chose quand elles utilisent le même mot; la seconde est d'avoir une correspondance entre les objets physiques étudiés et les objets mathématiques précis qui les décrivent. Je ne comprends pas pourquoi on assomme les enfants avec des définitions compliquées. Pourquoi définir une \og mesure\fg{} comme une \og comparaison avec une grandeur de même nature choisie arbitrairement comme étalon\fg ? Je suis sûr que la moitié des élèves qui connaissent cette définition n'ont pas été voir \emph{étalon} au dictionnaire.
\item De la rigueur mathématique. Ça ne fait pas de mal de faire remarquer que la vitesse se trouve en dérivant la position, et que bien souvent les conditions d'existences qui arrivent dans les résolutions d'équations décrivent des situations où la physique même du problème devient absurde. (vitesse infinie, bille qui roule sur un plan vertical). Un honnête personne de nos jours doit au moins une fois dans sa vie avoir entendu que la mathématique a une capacité à décrire la nature qui demeure au bas-mot étonnante.
\item Il y a je crois une grande différence entre la pédagogie de la mathématique et celle de la physique : en mathématique quand on remplace les lettres par des chiffres on comprend mieux. Si on ne comprend pas que $ax=b\Rightarrow x=b/a$, on peut regarder avec des nombres : $5\cdot 2=10\Rightarrow 2=10/5$, et on comprend mieux. En physique ce n'est pas vrai. Je ne crois pas qu'on comprend mieux la physique de ce qu'il se passe dans un exercice quand on remplace systématiquement toutes les lettres par les données de l'énoncé. Au contraire : le fond de la physique se trouve dans la manière dont les différentes grandeurs physiques se combinent (ou se simplifient !) dans les équations. Toute cette dynamique se perd quand on écrit des nombres à la place des lettres.
\item La physique est une science vivante. Il reste de nombreuses questions ouvertes. Certaines peuvent être énoncées sans connaissances très pointues, comme par exemple la miraculeuse coïncidence entre la masse dans la formule d'inertie $F=ma$ et la \emph{même} masse qui intervient dans la formule de l'énergie potentielle de gravitation $E=mgh$. La gravitation est liée à l'inertie d'une façon que personne ne comprend bien. Il y a un prix Nobel à prendre sur la question. Comme disait mon prof de relativité : \og En relativité générale, on postule l'égalité de la masse inertielle avec la masse pesante; mais c'est pas pour ça qu'on comprend mieux\fg.
\end{itemize}
Je crois pouvoir dire que dans le cadre de cours particuliers, ce petit programme a souvent été efficace. J'ai bien entendu eut des élèves qui sont restés indifférents à la physique et qui se sont contentés de réussir leur examen dans l'apparente ignorance de la beauté de la chose. Mais j'ai également eut des élèves qui ont finit par accepter que c'est amusant de réfléchir à un problème de physique quand on comprend ce qu'il veut dire.

Ici, ce sont des notes écrites dans lesquelles j'ai essayé de remettre ce que j'ai dit à différents élèves. On verra.

\section{Rappels}
%++++++++++++++++

\subsection*{Conseils de travail}
%----------------------------


\begin{enumerate}
\item Convertis les unités donnés en unités SI. 
\item Fais des scraboutchas, des oeuvres, des schémas, gribouilles, ce que tu veux mais dessines !
    Comme tu le vois, ce manuel contient un certain nombre de dessins. Ils ne sont pas faits pour être regardés avec admiration béate; ils sont faits pour inspirer les dessins que tu feras sur ta feuille. Tu dois comprendre le sens des dessins et être capable de les réinventer.
\item Si on te parle d'un triangle quelconque, n'en dessine pas un équilatéral, isocèle ou rectangle.
\item Écris de phrases complètes sur ton papier de brouillon. N'écris pas \og $R_c=3$ et $v_A=10$ \fg{} mais \og Le rayon du cylindre est \unit{3}{\meter} et la vitesse de $A$ est \unit{10}{\meter\per\second}\fg.


En particulier, méfies toi des \og formules\fg{} toute faites. Le symbole $F$ peut très bien désigner,  d'après le contexte,  une force quelconque ou un frottement; tout comme $P$ peut désigner la pesanteur, la pression ou la puissance. Ces symboles peuvent donc arriver dans plusieurs formules qui ne peuvent absolument pas être comparée. Donc il vaut toujours mieux un long discours qu'une petite formule. 

Tu dois pouvoir donner avec des mots la signification de tous les symboles que tu écris.


\item On n'efface pas une faute : on la barre proprement. Quand tu étudieras, il est tout aussi important de \emph{comprendre} les fautes que tu as faites que de comprendre les réponses justes. Si tu ne comprends pas pourquoi tu as fais une faute une fois, tu referas la faute.
\item Dans le même ordre d'idée, exit les correcteurs et tip-ex : quelque chose que tu crois faux pendant 10 secondes peut très bien être vrai. Ce serait bête de perdre l'idée pour quelque instants de doute.
\item Quand tu as la réponse, prends une feuille blanche, et récris \emph{proprement et à fond} toutes les étapes du raisonnement
\end{enumerate}

Ce n'est pas au moment où tu peux dire par coeur une définition que tu  \emph{connais} la définition. La physique est une science qui étudie la réalité; l'étude des mots compliqués qui rentrent dans une définition est du ressort de la \href{http://fr.wikipedia.org/wiki/Lexicologie}{lexicographique} ou de la \href{http://fr.wikipedia.org/wiki/Sémantique}{sémantique} et n'a aucun intérêt ici.

Pour voir si tu as compris une définition, tu dois pouvoir donner des exemple avec de préférence des objets que tu as sous les yeux au moment où tu lis la théorie -- dans ta chambre, dans ta classe, sur ton lit ou dans ta douche. Tu dois également être capable d'expliquer pourquoi tel ou tel exemple rentre bien dans la définition donnée.

\subsection{Avertissement}
%-------------------------

J'essayerai de ne faire aucun effort pour avoir une notation constante dans le texte. Les distances seront parfois notées $l$ (comme longueur), parfois $d$ (comme distance), la force de pesanteur sera notée indifféremment $ \fG$ (comme gravitation) ou $\fP$ (comme pesanteur) à ne pas confondre avec le $P$ de la puissance ou celui de la pression. Pourquoi ? Parce que je ne veux pas que tu retiennes des formules sous forme analytique : pas question de retenir $G=mg$ ou $W=Fd$. Il faut retenir
\begin{center}
force de gravitation sur un objet $=$ masse de l'objet fois constante $g$,
\end{center}
et
\begin{center}
travail d'une force qui bouge $=$ grandeur de la force fois le déplacement.
\end{center}


\subsection{Du système international}
%------------------------------------

Les unités se répartissent en trois catégories : les unités \emph{de base} du système international (SI), les unités \emph{dérivées} et les unités \emph{hors SI}. Affin d'éviter les erreurs de calcul, il est \emph{hyper-important} (!!) de n'utiliser que les unités du SI et ses dérivées.


\subsubsection{Les unités de base}
\begin{description}
\item[masse] kilogramme, noté \kilogram 
\item[longueur] mètre, noté \meter
\item[temps] seconde, noté \second
\item[courant électrique] ampère, noté \ampere.
\end{description}
\subsubsection{Les unités dérivées}
Ce sont toutes les unités qu'on peut obtenir en multipliant ou divisant les unités fondamentales entre-elles. Les plus courantes sont données dans le tableau \ref{TabUnitsSI}.

\begin{table}[hc]
\centering
\begin{tabular}{lccc}
Quantité & Unité & Abréviation & Conversion\\
\hline
Force              & newton  & \newton  & \kilogram\usk\meter\per\second\squared \\
Énergie et travail & joule   & \joule  & \kilogram\usk\meter\squared\per\second\squared\\
Puissance          & watt    & \watt  &  \kilogram\usk\meter\squared\per\cubic\second\\
Pression           & pascal  & \pascal & \kilogram\per\meter\usk\square\second\\
Fréquence          & hertz   & \hertz & $\reciprocal\second$
\end{tabular}
\caption{Les unités les plus courantes qui sont valables dans le système international.}\label{TabUnitsSI}
\end{table}

Toutes les autres unités sont à proscrire et à convertir avant de commencer quoi que ce soit.

\begin{description}
\item[longueurs] centimètres, kilomètres, années-lumière,\ldots $\rightarrow$ mètres
\item[temps] jours, années, minutes, heures,\ldots $\rightarrow$ secondes
\item[vitesses] kilomètres/heures\ldots $\rightarrow$ mètres/seconde
\end{description}

\begin{exemple}
Un exemple où on a l'impression que ce blabla ne sert à rien : un train avance à \unit{240}{\kilo\meter\per\hour} pendant une heure et demi. On multiplie 240 par 1,5 et on trouve bien que la distance parcourue est \unit{360}{\kilo\meter}. 

Pour bien faire, on devrait convertir $\unit{120}{\kilo\meter\per\hour}= \unit{33.333}\meter\per\second$ et $\unit{1.5}{\hour}=\unit{5400}{\second}$ puis multiplier pour trouver \unit{180000}{\meter}, qui est effectivement la même chose que \unit{180}{\kilo\meter}, mais obtenu de façon plus compliquée.
\end{exemple}

\begin{exemple}
Un exemple où l'on remarque que c'est important : une pierre est en chute libre pendant 2 minutes. Pour trouver la distance parcourue, on fait $gt^2/2$ avec $t=2$ et $g=9,81$. On trouve 19.62. Et c'est faux. Pourquoi ? Parce que quand on prend la peine d'écrire les unités, on remarque que $g=\unit{9.81}{\underline{\meter\per\square\second}}$. Dire \og $g=9.81$\fg, ça présuppose donc déjà un choix d'unité !

En fait à chaque fois qu'on utilise une constante physique \og qu'on connaît par coeur\fg{} sans convertir les données en système SI, on va droit dans le panneau.
\end{exemple}

\subsection{Le point matériel}
%------------------------------


Un corps matériel occupe un volume et possède une masse. Souvent, le volume occupé est petit par rapport au volume ambiant. Par exemple si on dit que la maison de la schtroumfette est à \unit{25}{\meter} ce celle du grand schtroumpf, on fait \og comme si\fg{} les deux maisons étaient des points dans le village; il est en effet possible que la penderie de bonnets rouges du grand schtroumf soit à \unit{24}{\meter} de la porte d'entrée de la schtroumfette ou que le PC de la schtroumfette soit à \unit{27}{\meter} de la réserve secrète d'alcools du grand schtroumpf. Quoi qu'il en soit, la plupart du temps,  on décrit la position des maisons avec des espèces de moyennes sans tenir compte de la structure interne des maisons.

Autre exemple : la croix sur la carte au trésor indique juste un point de l'île, sans préciser si ce point est le coin arrière gauche ou avant droit du coffre. Le pirate a fait comme si le coffre était juste un point. À moins que le butin soit vraiment énorme ou que l'île soit minuscule, le trésor est bien un point par rapport à l'ensemble de l'île.

\begin{definition}
Lorsque un objet est petit par rapport à l'espace dans lequel il se déplace, nous faisons souvent comme si il se réduisait à un seul point. Cette approximation est ce que nous appelons l'hypothèse du \defe{point matériel}{Point matériel}. 
\end{definition}

Dans les problèmes où la masse de l'objet est importante, le point que l'on confond avec l'objet complet est le \emph{centre de gravité} du corps matériel.


\begin{exemple}
Les objets suivants peuvent être considérés comme des points matériels.
\begin{enumerate}
\item Un vélo qui se déplace entre Pékin et Madrid
\item un ballon sur un terrain de football,
\item la lune autour de la terre,
\item une mouche dans la cuisine.
\end{enumerate}
\end{exemple}

\begin{exercice}
Donnez d'autres exemples de point matériel. Donnez aussi quelque exemples de choses qui ne sont pas des points matériels, comme le blanc dans un oeuf, par exemple. Donnez en particulier une situation dans laquelle une rame de métro n'est pas un point matériel.
\end{exercice}

