\section{Topologie}
%+++++++++++++++++

Maintenant que nous avons vu des choses amusantes avec l'infini, nous devons passer à une partie mois drôle et un peu plus formelle. Tu verras plus tard que ce qui intéresse le physicien, c'est les dérivées. Mais hélas pour toi, ô lecteur poly-dégouté par les math, pour pouvoir au mieux tirer parti de la notion de dérivée, il faut d'abord savoir correctement manipuler des limites ainsi savoir déterminer si une fonction est continue ou non. Or le cadre naturel de l'étude de la continuité est la topologie.

Les concepts de topologie et de continuité ne seront pour ainsi dire jamais invoqués dans des problèmes concrets de physique\footnote{Si si : des limites en l'infini, tu vas en voir dans tes cours de physique.}. Ceci pour la simple raison que pratiquement toutes les fonctions qui arrivent en physique sont continues. Le physicien ne s'embarrasse donc jamais de savoir si les fonctions qu'il manipule sont continues ou non. D'ailleurs, une des seules façons de distinguer à coup sûr un physicien d'un mathématicien, c'est que le mathématicien vérifie la continuité de ses fonctions avant de travailler tandis que le physicien pas : le physicien suppose que tout va bien, et quand il a vraiment un problème, il va aller pleurer chez un ami mathématicien pour qu'il lui dise si il n'y a pas un problème de topologie caché quelque part.

\subsection{Maximum et supremum}
%+++++++++++++++++++++++++++++++

\subsubsection{Un tout petit peu de théorie\ldots}
%//////////////////////

Ce n'est un secret pour personne que $\eR$ est un \href{http://fr.wikipedia.org/wiki/Relation_d'ordre}{ensemble ordonné} : il y a des éléments plus grands que d'autres, et mieux : à chaque fois que je prends deux éléments différents dans $\eR$, il y en a un des deux qui est plus grand que l'autre. Il n'y a pas d'\emph{ex aequo} dans $\eR$.

  Si je regarde l'intervalle $I=[0,1]$, je peux même dire que $10$ est plus grand que tous les éléments de $I$. Nous disons que $10$ est un \emph{majorant} de $[0,1]$. La définition est la suivante.
\begin{definition}
Lorsque $A$ est un sous-ensemble de $\eR$, on dit que $s$ est un \defe{majorant}{Majorant} de $A$ si $s$ est plus grand que tous les éléments de $A$. En d'autres termes, si
\[
  \forall x\in A,\,s\geq x.
\]
\end{definition}
Je me permet d'insister sur le fait que l'inégalité n'est pas stricte. Ainsi, $1$ est un majorant de $[0,1]$. Dès qu'un ensemble a un majorant, il en a plein. Si $s$ majore l'ensemble $A$, alors évident $s+1$, $s+4$, $s+\pi^2$ majorent également $A$.
\begin{exemple}
Une petite galerie d'exemples de majorants.
\begin{itemize}
\item L'intervalle fermé $[4,8]$ admet entre autres $8$ et $130$ comme majorants,
\item l'intervalle ouvert $]4,8[$ admet également $8$ et $130$ comme majorants,
\item $7$ n'est pas un majorant de $[1,5]\cup]8,32]$,
\item $10/10$ majore les côtes qu'on peut obtenir à une interrogation,
\item l'intervalle $[4,\infty[$ n'a pas de majorants.
\end{itemize}
\end{exemple}
Maintenant nous allons voir le premier concept vraiment subtile de toute l'histoire de tes cours de math\footnote{Quoi ? Il y a déjà des trucs que tu avais trouvé compliqué ? Eh bien rassures-toi : ce qui suit n'est pas spécialement \emph{compliqué}; c'est \emph{subtil} !}.
\begin{definition}
Le \defe{supremum}{Supremum} d'un ensemble est le plus petit majorant. En d'autres terme, $s$ est un supremum de $A$ si tout nombre plus petit que $s$ ne majore pas $A$, ou encore,
\[
  \forall x<s,\exists y\in A\text{ tel que } y>x.
\]
Nous disons que $M$ est un \defe{maximum}{Maximum} de $A$ si $M$ est un supremum \emph{et} $M\in A$.
\end{definition}
Quand $s$ est un supremum de $A$, ça veut dire que le moindre pas vers la gauche que l'on fait à partir de $s$ (c'est à dire le moindre $\epsilon$), et on tombe dans $A$, ou tout au moins, il existe des éléments de $A$ qui sont plus grand que $s-\epsilon$.

\subsubsection{\ldots et quelque exemples}
%//////////////////////

En matières de notations, le maximum de l'ensemble $A$ est noté $\max A$, le supremum est noté $\sup A$. Le minimum et l'infimum sont notés $\min A$ et $\inf A$.

\begin{exemple}
Exemples de différence entre majorant, supremum et maximum.
\begin{itemize}
\item Le nombre $10$ est un supremum, majorant et maximum de l'intervalle fermé $[0,10]$,
\item Le nombre $10$ est un majorant et un supremum, mais pas un maximum de l'intervalle ouvert $]0,10[$,
\item Le nombre $136$ est un majorant, mais ni un maximum ni un supremum de l'intervalle $[0,10]$.
\end{itemize}
\end{exemple}

En utilisant les notations concises, ces différents cas s'écrivent ainsi :
\begin{align*}
10&=\max[0,10]=\sup[0,10]	& 10&=\sup[0,10[
\end{align*}


\begin{exemple}
Si on dit que un pont s'effondre à partir d'une charge de $10$ tonnes, alors $10$ tonnes est un \emph{supremum} des charges que le pont peut supporter : si on met $9,999999$ tonnes dessus, il tient encore le coup, mais si on ajoute un gramme, alors il s'effondre (on sort de l'ensemble des charges acceptables).
\end{exemple}

\begin{exemple}
Si on dit qu'un pont résiste jusqu'à $10$ tonnes, alors $10$ tonnes est un \emph{maximum} de la charge acceptable. Sur ce pont-ci, on peut ajouter le dernier gramme. Mais à partir de là, le moindre truc qu'on ajoute, il s'effondre.
\end{exemple}

\begin{exemple}
Lorsqu'on dit que $50\%$ est le minimum requis pour passer son année, alors $50\%$ est un supremum des côtes avec lesquelles on rate, mais pas un maximum. En effet, pour tout $\epsilon$ plus grand que zéro, il y a une côte plus grande que $50-\epsilon$ qui fait rater : la côte $(50-\epsilon/2)$ par exemple.
\end{exemple}

\Exo{206}

Maintenant il est important de se rendre compte d'une chose : un ensemble ne peut avoir qu'un seul maximum et supremum. Jusqu'à présent nous avons toujours dit \emph{un} supremum. À partir de maintenant nous pouvons dire \emph{le} supremum. La preuve de cela est assez simple.
\begin{proposition}
Si $A$ est un sous-ensemble de $\eR$ admettant un supremum, alors il n'a qu'un seul supremum; et si il accepte un maximum, il n'en accepte un seul, et le maximum est égal au supremum.
\end{proposition}

\begin{proof}
Commençons par l'affirmation concernant le supremum. Supposons que $x$ et $y$ soient tous les deux suprema différents de $A$. Étant donné que $x\neq y$, nous pouvons supposer que $x<y$, et donc, par définition du fait que $y$ est un supremum, il existe un élément de $A$ qui est plus grand que $x$. Cela contredit le fait que $x$ soit supremum. En conclusion, il ne peut pas y avoir deux suprema différents pour un même ensemble.

Étant donné qu'un maximum est un supremum, il ne peut pas y avoir deux maxima différents vu qu'il ne peut pas y avoir deux suprema différents.
\end{proof}

\Exo{207}

\subsection{Espaces métriques}
%-----------------------------

Nous allons présenter maintenant les bases de la topologie sur des espaces métriques en prenant $\eR$ et $\eR^2$ comme exemple principaux. La topologie est un des fondements de la mathématique et est une prolongation de la théorie des ensembles. Nous n'en trouvons hélas pas beaucoup d'application directes en physique.

Si $E$ est un ensemble quelconque, nous disons qu'une \defe{distance}{Distance} sur $E$ est une fonction $d\colon E\times E\to \eR^+$ telle que
\begin{description}
\item[Symétrie] $d(x,y)=d(y,x)$,
\item[Séparation] $d(x,y)=0$ ssi $x=y$. Insistons sur le fait que dans tous les cas, nous devons avoir $d(x,y)\geq 0$,
\item[Inégalité triangulaire] $d(x,z)\leq d(x,y)+d(y,z)$
\end{description}
pour tout $x$, $y$, $z\in E$. Un ensemble muni d'une loi de distance s'appelle un \href{http://fr.wikipedia.org/wiki/Espace_métrique}{espace métrique}.

Le premier exemple d'espace métrique que nous connaissons est $\eR$ muni de la distance usuelle ente deux nombres :
\begin{equation}
d(x,y)=| y-x |.
\end{equation}
Je me permet de te faire remarquer la valeur absolue.

\begin{exercice}
Que penses-tu de la formule $d(x,y)=y-x$ pour définir une distance sur $\eR$ ?
\end{exercice}

À partir de là, nous définissons la notion de \defe{boule ouverte}{Boule!ouverte} sur l'ensemble $E$ centrée au point $x$ et de rayon $r>0$ comme
\[
  B(x,r)=\{ y\in\eR\tq d(x,y)< r \}.
\]
La \defe{boule fermée}{Boule!fermée} centrée en $x$ et de rayon $r>0$ est définie par
\[
  \bar B(x,r)=\{ y\in\eR\tq d(x,y)\leq r \}.
\]
La différence est que dans la première l'inégalité est stricte.

\begin{theorem}		\label{ThoBoulOuvVois}
Une boule ouverte contient une boule ouverte autour de chacun de ses points.
\end{theorem}

\begin{proof}
Prenons $y\in B(x,r)$, et prouvons que la boule $B(y,r-d(x,y))$ est contenue dans $B(x,r)$. Première chose : $r-d(x,y)>0$ parce que $y$ est dans la boule ouverté centrée en $x$ et de rayon $r$. Pour prouver que  $B(y,r-d(x,y))\subset B(x,r)$, prenons un point dans le premier ensemble et montrons qu'il est dans le second ensemble.

Soit donc $z\in B\big(y,r-d(x,y)\big)$ et testons $d(x,z)$ que nous voudrions être plus petit que~$r$. Et, miracle, il l'est parce que
\begin{align*}
  d(x,z)	&\leq d(x,y)+d(y,z)&\text{inégalité triangulaire}\\
		&<d(x,y)+\big(r-d(x,y)\big)&\text{$z\in B\big(y,r-d(x,y)\big)$}\\
		&=r.
\end{align*}
Remarquez que la première inégalité n'est pas stricte, tandis que la seconde est stricte. Nous avons donc bien $d(x,z)<r$ (strictement) comme le demandé pour que $z$ soit dans la boule \emph{ouverte} de centre $x$ et de rayon $r$.
\end{proof}

Lorsque $x\in E$, nous disons qu'un \defe{voisinage}{Voisinage} de $x$ est n'importe quel sous-ensemble de $E$ qui contient une boule ouverte centrée en $x$. Nous disons qu'un ensemble est \defe{ouvert}{Ouvert} si il contient un voisinage de chacun de ses points. Évidement les boules ouvertes sont les prototypes d'ouverts par le théorème \ref{ThoBoulOuvVois}. Par convention, nous disons que l'ensemble vide est ouvert.

\begin{definition}
L'ensemble des boules ouvertes d'un espace métrique forment la \defe{topologie}{Topologie!métrique} de l'espace.
\end{definition}

Un ensemble est ouvert si et seulement si il contient une boule autour de chacun de ses points.

Nous allons dire qu'une partie $A$ d'un espace métrique est \defe{bornée}{Bornée} si il existe une boule\footnote{À titre d'exercice, je te laisse te convaincre que l'on peut dire boule \emph{ouverte} ou \emph{fermée} au choix sans changer la définition.} qui contient $A$.

\begin{lemma}  \label{LemSupOuvPas}
Le supremum d'un ensemble ouvert n'est pas dans l'ensemble (et n'est donc pas un maximum).
\end{lemma}

\begin{proof}
Soit $\mO$, un ensemble ouvert et $s$, son supremum. Si $s$ était dans $\mO$, on aurait un voisinage $B=B(s,r)$ de $s$ contenu dans $\mO$. Le point $s+r/2$ est alors à la fois dans $\mO$ et plus grand que $s$, ce qui contredit le fait que $s$ soit un supremum de $\mO$.
\end{proof}

\begin{exercice}
Par le même genre de raisonnements, montrez que l'union et l'intersection de deux ouverts sont encore des ouverts.
\end{exercice}

\begin{remark}
L'intersection d'une \emph{inifinté} d'ouverts n'est pas spécialement un ouvert comme le montre l'exemple suivant :
\[ 
  \mO_i=]1,2+\frac{ 1 }{ i }[.
\]
Tous les ensembles $\mO_i$ contiennent le point $2$ qui est donc dans l'intersection. Mais quel que soit le $\epsilon>0$ que l'on choisisse, le point $2+\epsilon$ n'est pas dans $\mO_{(1/\epsilon)+1}$. Donc aucun point au-delà de $2$ n'est dans l'intersection, ce qui prouve que $2$ ne possède pas de voisinages contenus dans $\cap_{i=1}^{\infty}\mO_i$.
\end{remark}

\begin{exercice}
Prouver que, quels que soient les ensembles $A$ et $B$ dans $\eR$, nous avons
\[ 
  \sup(A\cap B)\leq\sup A\leq\sup(A\cup B).
\]
\end{exercice}


\subsection{Connexité}
%----------------------

Dès qu'un ensemble est muni d'une métrique, nous pouvons définir les boules ouvertes, les voisinages et les sous-ensembles ouverts. Dès que l'on a identifié les sous-ensemble ouverts de $E$, nous disons que $E$ devient un \defe{espace topologique}{Espace topologique}. Nous allons maintenant un pas plus loin.

Nous voulons maintenant décrire ce qu'est un ensemble connexe. La notion intuitive d'un ensemble connexe est le fait que l'on puisse aller d'un point à l'autre sans sortir. En d'autres mots, nous voulons dire qu'un ensemble est connexe quand il est en un seul morceau. Nous avons donc envie de dire que le sous-ensemble $A$ de $E$ est connexe quand il ne peut pas être écrit comme une union disjointe de deux ensembles.

Cette définition ne peut pas fonctionner telle quelle, parce que tout ensemble peut être écrit comme l'union disjointe de deux ensembles. Il faut donc un peu contraindre le choix d'ensembles en lesquels on ne veut pas que $A$ se décompose. Il se faut que la bonne définition est la suivante :

\begin{definition}
 Lorsque $E$ est un espace topologique, nous disons qu'un sous-ensemble $A$ est \defe{non connexe}{Connexe} quand on peut trouver des ouverts $O_1$ et $O_2$ tels que
\begin{equation} 	\label{EqDefnnCon}
  A=(A\cap O_1)\cup (A\cap O_2),
\end{equation}
et tels que $A\cap O_1\neq\emptyset$, et $A\cap O_2\neq\emptyset$.
Si un sous-ensemble n'est pas non-connexe, alors on dit qu'il est connexe.
\end{definition}
Une autre façon d'exprimer la condition \eqref{EqDefnnCon} est de dire que $A$ n'est pas connexe quand il est contenu dans la réunion de deux ouverts disjoints qui intersectent tous les deux $A$.
\begin{figure}
\centering
\psset{xunit=1cm,yunit=1cm}
\begin{pspicture}(-4,-1.3)(3.2,2.7)
   %\psframe[linecolor=cyan](-4,-1.3)(3.2,2.7)
   \psset{PointSymbol=none,PointName=none}
	%\pspolygon[fillstyle=vlines,hatchcolor=red,linecolor=red](0,0)(0,2)(2.5,2)(1,-1)
	%\pspolygon[fillstyle=vlines,hatchcolor=red,linecolor=red](0,0)(0,2)(2.5,2)(1,-1)
	\pspolygon[fillstyle=vlines](0,0)(0,2)(2.5,2)(1,-1)
	\pspolygon[fillstyle=vlines](-2,0)(-3,0.4)(-2.5,2)
	\pstGeonode(0,1){A}(0,3){B}(0,2.5){C}
	\pstCircleOA[linecolor=red,Radius=\pstDistAB{A}{B}]{1.2,0.7}{}
	\pstCircleOA[linecolor=blue,Radius=\pstDistAB{A}{C}]{-2.5,1}{}
\end{pspicture}

\caption{La figure hachurée n'est pas connexe parce qu'on peut dessiner deux ouverts disjoints qui la sépare.}  \label{FigExnnConn}
\end{figure}
Le cas de la figure \ref{FigExnnConn} montre une surface dans $\eR^2$ qui est clairement non connexe. Mais il y a des cas nettement moins faciles à traiter. La figure \ref{FigConnPapi} montre deux parties de $\eR^2$ qui ne diffèrent que de un seul point. Est-ce que tu pourrais dire si il y en a un des deux qui est connexe ?
\begin{figure}
\centering
\psset{xunit=1cm,yunit=1cm}
\subfigure[Cet ensemble contient le point central.]{%
\begin{pspicture}(-2,-1)(2,1)
	\psset{PointSymbol=none, PointName=none}
	\pspolygon[fillstyle=vlines](0,0)(-2,1)(-2,-1)(2,1)(2,-1)
\end{pspicture}
%
}
\subfigure[Cet ensemble ne contient pas le point central.]{%
\begin{pspicture}(-2,-1)(2,1)
	\psset{PointSymbol=none, PointName=none}
	\pspolygon[fillstyle=vlines](0,0)(-2,1)(-2,-1)(2,1)(2,-1)
   \pstGeonode(0,0){A}(0.1,0){B}
	\pstCircleOA[fillstyle=solid,fillcolor=white,linecolor=black]{A}{B}
\end{pspicture}
}
\caption{Exemple de deux ensembles dont la connexité se joue à un point près.}	\label{FigConnPapi}
\end{figure}
Nous n'allons pas traiter plus avant cet exemple. Au lieu de cela, nous allons déterminer tous les sous-ensembles connexes de $\eR$. Pour cela nous avons besoin d'une définition précise de ce que l'on appelle un \emph{intervalle} dans~$\eR$.
\begin{definition}
	Un \defe{intervalle}{Intervalle} est une partie de $\eR$ telle que tout élément compris entre deux éléments de la partie soit dedans. En formule, la partie $I$ de $\eR$ est un intervalle si
	\[
	  \forall a,b\in I,(a\leq x\leq b)\Rightarrow x\in I.
	\]
\end{definition}
Cette définition englobe tous les exemples que tu connais d'intervalles ouverts, fermés avec ou sans infini : $[a,b]$, $[a,b[$, $]-\infty,a]$, \ldots

Une des nombreuses propositions qui vont servir à prouver le théorème des \href{http://fr.wikipedia.org/wiki/Théorème_des_valeurs_intermédiaires}{valeurs intermédiaires} (théorème numéro \ref{ThoValInter}) est la suivante.
\begin{proposition}	\label{PropInterssiConn}
	Une partie de $\eR$ est connexe si et seulement si c'est un intervalle.
\end{proposition}

\begin{proof}
	La preuve est en deux partie. D'abord nous démontrons que si un sous-ensemble de $\eR$ est connexe, alors c'est un intervalle; et ensuite nous démontrons que tout intervalle est connexe. Je te préviens que les deux parties sont difficiles, alors ouvres bien grand tes oreilles.

	Affin de prouver qu'un ensemble connexe est toujours un intervalle, nous allons prouver que si un ensemble n'est pas un intervalle, alors il n'est pas connexe. Prenons $A$, une partie de $\eR$ qui n'est pas un intervalle. Il existe donc $a$, $b\in A$ et un $x_0$ entre $a$ et $b$ qui n'est pas dans $A$. Comme le but est de prouver que $A$ n'est pas connexe, il faut couper $A$ en deux ouverts disjoints. L'élément $x_0$ qui n'est pas dans $A$ est le bon candidat pour effectuer cette coupure. Prenons $M$, un majorant de $A$ et $m$, un minorant de $A$, et définissons (voir figure \ref{FigChoixabxz})
	\begin{align*}
		\mO_1&=]m,x_0[\\
		\mO_2&=]x_0,M[.
	\end{align*}
	Si $A$ n'a pas de minorant, nous remplaçons la définition de $\mO_1$ par $]-\infty,x_0[$, et si $A$ n'a pas de majorant, nous remplaçons la définition de $\mO_2$ par $]x_0,\infty[$. Dans tous les cas, ce sont deux ensembles ouverts dont l'union recouvre tout $A$. En effet, $\mO_1\cup \mO_2$ contient tous les nombres entre un minorant de $A$ et un majorant sauf $x_0$, mais on sait que $x_0$ n'est pas dans $A$. Cela prouve que $A$ n'est pas connexe.
	\begin{figure}[ht]
	\centering
	\begin{pspicture}(-3.5,-0.5)(7.5,0.7)
	   %\psframe[linecolor=cyan](-3.5,-0.5)(7.5,0.7)
	   \psset{PointSymbol=none,PointName=none}
	   \pstGeonode[PointSymbol=*](0,0){Pa}(5,0){Pb}

	% Une petite manip pour élargir le segment Pa,Pb de façon symétrique en ne devant taper qu'une seule fois le facteur. 
	%   Le résultat est La,Lb
	   \pstHomO[HomCoef=1.3]{Pb}{Pa}[La]
	   \pstTransHom{La}{Pa}{Pb}{1}{Lb}

		\psline(La)(Lb)
		\pstMarquePoint{La}{0.3;180}{$A$}
		\pstMarquePoint{Pa}{0.3;90}{$a$}
		\pstMarquePoint{Pb}{0.3;90}{$b$}

	   \pstGeonode(0,0){O}(-3,0.5){Ea}(2,0.5){Eb}			% Les tailles des éllipses, plus un point O en (0,0).
		\psellipse[linecolor=red](Pa)(Ea)
		\psellipse[linecolor=blue](Pb)(Eb)
	   \pstGeonode[PointSymbol=*](3,0){Xz}
		\pstMarquePoint{Xz}{0.4;-90}{$x_0$}

	   \pstTransHom{O}{Eb}{Pb}{1}{mOb}
		\pstMarquePoint{mOb}{0,0}{ {\blue $\mO_2$}}
	   \pstTransHom{O}{Ea}{Pa}{1}{mOa}
		\pstMarquePoint{mOa}{0,0}{ {\red $\mO_1$}}

	\end{pspicture}

	\caption{Le point $x_0$ et les ouverts qui coupent en deux la partie $A$. Le fait déterminant dans la démonstration est que $x_0$ se trouve à la frontière entre $\mO_1$ et $\mO_2$}  \label{FigChoixabxz}
	\end{figure}

	Jusqu'à présent nous avons prouvé que si un ensemble n'est pas un intervalle, alors il ne peut pas être connexe. Pour remettre les choses à l'endroit, prenons un ensemble connexe, et demandons-nous si il peut être autre chose qu'un intervalle ? La réponse est \emph{non} parce que si il était autre chose, il ne serait pas connexe.

	Prouvons à présent que tout intervalle est connexe. Pour cela, nous refaisons le coup de \href{http://fr.wikipedia.org/wiki/Contraposée}{la contraposée}. Nous allons donc prendre une partie $A$ de $\eR$, supposer qu'elle n'est pas connexe et puis prouver qu'elle n'est alors pas un intervalle. Nous avons deux ouverts disjoints $\mO_1$ et $\mO_2$ tels que $A\subset \mO_1\cup \mO_2$. Prenons $a\in A_1$ et $b\in A_2$. Pour fixer les idées, on suppose que $a<b$. Maintenant, le jeu est de montrer qu'il existe une point $x_0$ entre $a$ et $b$ qui ne soit pas dans $A$ (cela montrerait que $A$ n'est pas un intervalle). Nous allons prouver que c'est le cas du point
	\[ 
	  x_0=\sup\{ x\in\mO_1\tq x<b \}.
	\]
	Étant donné que l'ensemble $\mA=\{ x\in\mO_1\tq x<b \}$ est ouvert\footnote{C'est l'intersection entre l'ouvert $\mO_1$ et l'ouvert $\{x\tq x<b \}$.}, le point $x_0$ n'est pas dans l'ensemble par le lemme \ref{LemSupOuvPas}. Nous avons donc
	\begin{itemize}
		\item soit $x_0$ n'est pas dans $\mO_1$,
		\item soit $x_0\leq b$,
		\item soit les deux en même temps.
	\end{itemize}
	Nous allons montrer qu'un tel $x_0$ ne peut pas être dans $A$. D'abord, remarquons que $\sup\mA\leq\sup\mO$ parce que $\mA$ est une intersection de $\mO$ avec quelque chose. Ensuite, il n'est pas possible que $x_0$ soit dans $\mO_2$ parce que tout élément de $\mO_2$ possède un voisinage contenu dans $\mO_2$. Un point de $\mO_2$ est donc toujours strictement plus grand que le supremum de $\mO_1$.

	Maintenant, remarque que si $x_0\leq b$, alors $x_0=b$, sinon $b$ serait un majorant de $\mA$ plus petit que $x_0$, ce qui n'est pas possible vu que $x_0$ est le supremum de $\mA$ et donc le plus petit majorant. Oui mais si $x_0=b$, c'est que $x_0\in\mO_2$, ce qu'on vient de montrer être impossible. Nous voila déjà débarrassé des deuxièmes et troisième possibilités. 

	Si la première possibilité est vraie, alors $x_0$ n'est pas dans $A$ parce qu'on a aussi prouvé que $x_0\notin\mO_2$. Or n'être ni dans $\mO_1$ ni dans $\mO_2$ implique de ne pas être dans $A$. Ce point $x_0=\sup\mA$ est donc hors de $A$.

	Oui, mais comme $a\in\mA$, on a obligatoirement que $x_0\geq a$. Mais par construction, on a aussi que $x_0\leq b$ (ici, l'inégalité est même stricte, mais ce n'est pas important). Donc
	\[ 
	  a\leq x_0\leq b
	\]
	avec $a$, $b\in A$, et $x_0\notin A$. Cela finit de prouver que $A$ n'est pas un intervalle.
\end{proof}

