% This is part of Un soupçon de physique, sans être agressif pour autant
% Copyright (C) 2006-2009
%   Laurent Claessens
% See the file fdl-1.3.txt for copying conditions.


\begin{exercice} \label{exo:deplac2}\label{exo005}
Un limaçon marche en ligne droite de la salade qu'il vient de manger en $A$ vers le pied du muret du potager en $B$. Ensuite, il monte jusqu'au sommet du muret en $C$.
\begin{enumerate}
\item Quelle est la valeur du déplacement ?
\item Quelle est la distance qu'a parcourue\footnote{Exercice complémentaire : cet accord est-il correct ? Si oui, pourquoi ?} notre ami gastéropode ?
\end{enumerate}
Les axes de la figure \ref{fig:deplac2} sont gradués en mètres. Donnez les réponses en centimètres, kilomètres et milimètres. Si le coeur vous en dit, donnez aussi les réponses en unités de longueur de la taille de votre pied et en \href{http://fr.wikipedia.org/wiki/M%C3%A9ga}{méga}\href{http://fr.wikipedia.org/wiki/Parsec}{parsec}.
\end{exercice}



