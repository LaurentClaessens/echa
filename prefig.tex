% This is part of Un soupçon de physique, sans être agressif pour autant
% Copyright (C) 2006-2009
%   Laurent Claessens
% See the file fdl-1.3.txt for copying conditions.


% Ce fichier contient les commandes qui définissent la géométrie d'un certain nombre de figures des exercices. 
% Ce sont des commandes non immédiatement intégrées à l'environnement figure de l'exercice, de façon à pouvoir avoir la même géométrie dans le corrigé. Ainsi je peux modifier le dessin de l'exercice et du corrigé en même temps.

% Je les rassemble dans un fichier qui est toujours chargé par \input pour des raisons de simplicité lors de la compilation par pytex de morceaux contenant des exercices différents.

%Pour exo013
\newcommand{\prefigzerotreize}{
\pstGeonode(0,0){O}(3;25){A}(0.75;-11){B}(1.5;-100){C}
}

% Pour exo018
\newcommand{\prefigzerounhuit}{%
   \pstGeonode[PointName=default,PosAngle={180,0,40}](0,0){A}(4,0){C}(4,2){B}
 \pstHomO[HomCoef=0.3]{A}{B}[Oa]
\pstRotation{Ao}{B}[Obu]
\pstHomO[HomCoef=0.15]{Oa}{Obu}[Ob]
   \pstMiddleAB{Oa}{Ob}{Cc}
   \rput(Cc){\pstGeonode(4,0){bF}}
				}
% Pour exo019
\newcommand{\prefigzerounneuf}{%
   \pstGeonode[PointName=default,PosAngle={180,0}](0,0){A}(4,0){C}(4,2){B}

\pstHomO[HomCoef=0.7]{A}{B}[Oa]			% Place de la première roue
\pstHomO[HomCoef=0.85]{A}{B}[Pa]			% Place de la seconde roue
\pstRotation[RotAngle=90]{Oa}{B}[Obu]
\pstHomO[HomCoef=0.2]{Oa}{Obu}[Ob] 		% Rayon des roues
\pstMiddleAB{Oa}{Ob}{Oc}
   \pstTranslation{Oa}{Pa}{Ob,Oc}[Pb,Pc]	

\pstHomO[HomCoef=1.5]{Pc}{Oc}[Cg]
\pstHomO[HomCoef=1.5]{Oc}{Pc}[Cd]		% Longueur du chariot
\pstTranslation{Oa}{Ob}{Cd}[pCdh]
\pstTranslation{Oa}{Ob}{Cg}[pCgh]
\pstHomO[HomCoef=1.5]{Cd}{pCdh}[Cdh]		% Hauteur du chariot
\pstHomO[HomCoef=1.5]{Cg}{pCgh}[Cgh]


\pstMiddleAB{Cg}{Cdh}{Cc}
\pstTranslation{Oa}{Ob}{Cc}[pbR]
\pstHomO[HomCoef=4]{Cc}{pbR}[bR]		% Longueur de R
\pstTransHom{Oc}{Pc}{Cc}{4}{bF}

\pstTransHom{B}{C}{Cc}{0.8}{bG}
}


% Pour corr028, et est utilisé dans la partie «Équilibre sur un plan incliné»
\newcommand{\prefigplincl}{%
\pstGeonode[PointName=default,PosAngle={180,0}](0,0){A}(3.5,0){C}(3.5,2.5){B}

\pstHomO[HomCoef=0.3]{A}{B}[Oa]			% Place de la première roue
\pstHomO[HomCoef=0.4]{A}{B}[Pa]			% Place de la seconde roue
\pstRotation[RotAngle=90]{Oa}{B}[Obu]
\pstHomO[HomCoef=0.07]{Oa}{Obu}[Ob] 		% Rayon des roues
\pstMiddleAB{Oa}{Ob}{Oc}

\pstTranslation{Oa}{Pa}{Ob,Oc}[Pb,Pc]	

\pstHomO[HomCoef=1.5]{Pc}{Oc}[Cg]
\pstHomO[HomCoef=1.5]{Oc}{Pc}[Cd]		% Longueur du chariot
\pstTranslation{Oa}{Ob}{Cd}[pCdh]
\pstTranslation{Oa}{Ob}{Cg}[pCgh]
\pstHomO[HomCoef=1.5]{Cd}{pCdh}[Cdh]		% Hauteur du chariot
\pstHomO[HomCoef=1.5]{Cg}{pCgh}[Cgh]


\pstMiddleAB{Cg}{Cdh}{Cc}
\pstTranslation{Oa}{Ob}{Cc}[pbR]
\pstHomO[HomCoef=4]{Cc}{pbR}[bR]		% Longueur de R


\pstTransHom{Oc}{Pc}{Cc}{4}{bF}
\pstTransHom{B}{C}{Cc}{0.7}{bG}
}

