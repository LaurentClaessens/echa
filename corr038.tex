% This is part of Un soupçon de physique, sans être agressif pour autant
% Copyright (C) 2006-2009
%   Laurent Claessens
% See the file fdl-1.3.txt for copying conditions.


\begin{corrige}{038}
\begin{enumerate}

\item Au départ l'énergie du cahier est potentielle, et à la fin elle est cinétique. L'énergie potentielle de départ est celle d'une masse de \unit{0.2}{\kilo\gram} à une hauteur de \unit{5}{\meter}, c'est à dire $E_P=mgh=\unit{9.81}{\joule}$. 

Pour trouver la vitesse $v$ à laquelle la pierre va toucher le sol, il faut égaliser cette énergie à $mv^2/2$ et en déduire $v$ :
\[ 
  v=\sqrt{ \frac{ 2E_P }{ m } }=\sqrt{\frac{ 2mgh }{ m }}=\sqrt{2gh}=\unit{9.904}{\meter\per\second}.
\]
Une fois de plus, {\bf la réponse ne dépend  pas de la masse !}


\item  Lorsque l'on tient compte des frottement, l'énergie potentielle de départ est convertie en énergie cinétique plus de l'énergie perdue. Cette énergie perdue vaut le travail des forces de frottement : $Fh=0.2\cdot 40=\unit{8}{\joule}$. Donc,
\[ 
  \frac{ mv^2 }{ 2 }-Fh=mgh,
\]
et donc
\[ 
  v=\sqrt{  \frac{ 2(mgh+Fh) }{ m }  }=\unit{7.94}{ \meter\per\second}.
\]

\end{enumerate}


\end{corrige}
