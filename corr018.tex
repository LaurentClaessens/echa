% This is part of Un soupçon de physique, sans être agressif pour autant
% Copyright (C) 2006-2009
%   Laurent Claessens
% See the file fdl-1.3.txt for copying conditions.


%Copyright (c) 2006 Claessens Laurent. Permission is granted to copy, distribute and/or modify this document under the terms of the  GNU Free Documentation License, Version 1.2 or any later version published by the Free Software Foundation; with no Invariant Sections, no Front-Cover Texts, and no Back-Cover Texts. A  copy of the license is included in the section entitled "GNU Free Documentation License".
\begin{corrige}{018}
Le mobile effectue à la fois le déplacement $AB$ et le déplacement $AC$. Mais on sait que dans le travail, seule compte la composante du déplacement parallèle à la force. Donc le problème serait exactement le même si on faisait $\| BC \|=0$, c'est à dire si le mobile se déplaçait sur un plan horizontal. Dans ce cas, le déplacement est parallèle à la force et donc le travail vaut $F\| AC \|$.


Une autre façon de répondre est de considérer la formule du travail dans le cas non parallèle, c'est à dire
  $W=F\cdot d\cdot\cos(\fF,\overrightarrow{d})$ où $\cos(\fF,\overrightarrow{d})$ désigne le cosinus de l'angle entre $\fF$ et le déplacement $\overrightarrow{d}$, c'est à dire l'angle $\widehat{BAC}$. Donc
\[ 
  W=F\cdot | AB |\cdot \cos(\widehat{ABC}),
\]
mais $| AB |\cos(\widehat{BAC})=| AC |$, et donc
\[ 
  W=F\cdot | AC |.
\]

La façon la plus efficace de tirer cet objet est de le tirer parallèlement au déplacement (de façon à ce que le cosinus soit $1$), c'est à dire avec une force parallèle au plan incliné. En effet, la composante perpendiculaire est annulée par une réaction du sol.

\begin{figure}[h]
\centering
\begin{pspicture}(-0.5,-1.5)(5.5,3)

   \psset{PointSymbol=none, PointName=none}
   \prefigzerounhuit
   \psline(A)(C)
   \psline(C)(B)
   \psline(A)(B)
   \pstCircleAB[fillstyle=crosshatch,fillcolor=black]{Oa}{Ob}
   \pstMarqueForce{Cc}{bF}{0.3;270}{$\fF$}
   \pstDecompForce{Cc}{bF}{A}{B}{Oa}{Ob}{Fu}{Fd}
{%
\psset{linecolor=blue}
   \pstMarqueForce{Cc}{Fu}{0.3;0}{$F_{\parallel}$}
   \pstMarqueForce{Cc}{Fd}{0.3;0}{$F_{\perp}$}
}
\end{pspicture}
\caption{Décomposition de la force de la correction \ref{corr018}.}
\end{figure}
 

\end{corrige}
