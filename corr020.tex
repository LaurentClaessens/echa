% This is part of Un soupçon de physique, sans être agressif pour autant
% Copyright (C) 2006-2009
%   Laurent Claessens
% See the file fdl-1.3.txt for copying conditions.


%Copyright (c) 2006 Claessens Laurent. Permission is granted to copy, distribute and/or modify this document under the terms of the  GNU Free Documentation License, Version 1.2 or any later version published by the Free Software Foundation; with no Invariant Sections, no Front-Cover Texts, and no Back-Cover Texts. A  copy of the license is included in the section entitled "GNU Free Documentation License".
\begin{corrige}{020}
Les forces $F_2$ et $\fG_1$ sont perpendiculaires au déplacement, donc leur travail est nul. Les forces $\fF_4$ et $\fG_G$ sont parallèles au déplacement, dans le même sens; donc leur travail vaut juste le produit de leurs normes par la longueur du déplacement : $W_{\fF_4}=\| \fF_4 \|\cdot d$ et $W_{\fG_2}=\| \fG_2 \|\cdot d$. La force $\fF_1$ va dans le sens inverse de son déplacement, ce qui fait venir un signe moins (penser au cosinus de $180$) : $W_{\fF_1}=-\| \fF_1 \|\cdot d$.

Si le tout se déplace d'une distance $d$ vers la gauche, il faut juste changer tous les signes.
\end{corrige}
