% This is part of Un soupçon de physique, sans être agressif pour autant
% Copyright (C) 2006-2010
%   Laurent Claessens
% See the file fdl-1.3.txt for copying conditions.


%+++++++++++++++++++++++++++++++++++++++++++++++++++++++++++++++++++++++++++++++++++++++++++++++++++++++++++++++++++++++++++
					\section{Des exemples}
%+++++++++++++++++++++++++++++++++++++++++++++++++++++++++++++++++++++++++++++++++++++++++++++++++++++++++++++++++++++++++++

%---------------------------------------------------------------------------------------------------------------------------
					\subsection{La vitesse}
%---------------------------------------------------------------------------------------------------------------------------

Si tu lis ces lignes, c'est que tu as plus que probablement déjà entendu le baratin des physiciens à propos de la nuance entre les vitesses instantanées et vitesses moyennes. Relis par exemple la section \ref{SecVitmoyinst}, et en particulier les exemples \ref{ExoDerrmouvunif} et \ref{ExDerchutelobre} qui y sont donnés. Imprègne toi bien de ces idées.

Lorsqu'un mobile se déplace à une vitesse variable, nous obtenons la \emph{vitesse instantanée} en calculant une vitesse moyenne sur des intervalles de plus en plus petits. Si le mobile a un mouvement donné par $x(t)$, la vitesse moyenne entre $t=2$ et $t=5$ sera
\[ 
  v_{\text{moy}}(2\to 5)=\frac{ x(5)-x(2) }{ 5-2 }.
\]
Plus généralement, la vitesse moyenne entre $2$ et $2+\Delta t$ est donnée par
\[ 
  v_{\text{moy}}(2\to 2+\Delta t)=\frac{ x(2+\Delta t)-x(2) }{ \Delta t }.
\]
Cela est une fonction de $\Delta t$. Oui, mais je te rappelle qu'on a dans l'idée de calculer une vitesse instantanée, c'est à dire de voir ce que vaut la vitesse moyenne sur un intervalle très {\small très} {\footnotesize très} {\scriptsize très} {\tiny petit}. La notion de limite de la définition \ref{DefLimPointSansInfini} semble toute indiquée pour décrire mathématiquement l'idée physique de vitesse instantanée.

Nous allons dire que la vitesse instantanée d'un mobile est la limite quand $\Delta t$ tends vers zéro de sa vitesse moyenne sur l'intervalle de temps $\Delta t$, ou en formule :
\begin{equation}		\label{Eqvinstlimite}
	v(t_0)=\lim_{\Delta t\to 0}\frac{ x(t_0)-x(t_0+\Delta t) }{ \Delta t }.
\end{equation}

%---------------------------------------------------------------------------------------------------------------------------
					\subsection{La tangente à une courbe}
%---------------------------------------------------------------------------------------------------------------------------

Passons maintenant à tout autre chose, mais toujours dans l'utilisation de la notion de limite pour résoudre des problèmes intéressants. Comment trouver l'équation de la tangente à la courbe $y=f(x)$ au point $(x_0,f(x_0))$ ?

Essayons de trouver la tangente au point $P$ donné de la courbe donnée à la figure \ref{fig_derrtrois}.

% This is part of Un soupçon de physique, sans être agressif pour autant
% Copyright (C) 2006-2009
%   Laurent Claessens
% See the file fdl-1.3.txt for copying conditions.


% Fichier généré automatiquement. Ne pas modifier à la main.
\begin{figure}[ht]
\centering
\begin{pspicture}(0.5,-1.0)(8.0,4.625)
\psset{PointSymbol=none,PointName=none,algebraic=true}

\psplot[]{0.5}{8}{(-3/x)+5}
\pstGeonode[PointSymbol=none,PointName=none](1.0,2.0){aaa}
\pstGeonode[PointSymbol=*](1.0,2.0){aaa}
\rput(aaa){\rput(0.3;135){$P$}}
\end{pspicture}

\psset{xunit=1,yunit=1}

\caption{Comment trouver la tangente au point $P$ à cette courbe ?}\label{fig_derrtrois}
\end{figure}


% Fichier généré automatiquement. Ne pas modifier à la main.
\begin{figure}[ht]
\centering
\begin{pspicture}(-0.264911064067,-1.7947331922)(8.0,5.7947331922)
\psset{PointSymbol=none,PointName=none,algebraic=true}

\psplot[]{0.5}{8}{(-3/x)+5}
\pstGeonode[PointSymbol=none,PointName=none](-0.264911064067,-1.7947331922){abq}
\pstGeonode[PointSymbol=none,PointName=none](2.26491106407,5.7947331922){abr}
\pstLineAB[linecolor=green]{abq}{abr}
\pstGeonode[PointSymbol=none,PointName=none](1.0,2.0){aaa}
\pstGeonode[PointSymbol=none,PointName=none](3.0,2.0){abd}
\pstLineAB[linestyle=dashed]{aaa}{abd}
\pstGeonode[PointSymbol=none,PointName=none](3.0,4.0){aab}
\pstGeonode[PointSymbol=none,PointName=none](3.0,2.0){abd}
\pstLineAB[linestyle=dashed]{aab}{abd}
\pstGeonode[PointSymbol=none,PointName=none](0.292893218813,1.29289321881){ace}
\pstGeonode[PointSymbol=none,PointName=none](3.70710678119,4.70710678119){ach}
\pstLineAB[linecolor=blue]{ace}{ach}
\pstGeonode[PointSymbol=*](1.0,2.0){aaa}
\rput(aaa){\rput(0.3;135){$P$}}
\pstGeonode[PointSymbol=*](3.0,4.0){aab}
\rput(aab){\rput(0.3;90){$Q$}}
\pstGeonode[PointSymbol=*](3.0,2.0){abd}
\rput(abd){\rput(0.3;-45){$R$}}
\pstGeonode[PointSymbol=none,PointName=none](2.0,2.0){abe}
\pstGeonode[PointSymbol=none](2.0,2.0){abe}
\rput(abe){\rput(0.3;-90){$\Delta x$}}
\pstGeonode[PointSymbol=none,PointName=none](3.0,3.0){abf}
\pstGeonode[PointSymbol=none](3.0,3.0){abf}
\rput(abf){\rput(0.4;0){$\Delta y$}}
\end{pspicture}

\psset{xunit=1,yunit=1}

\caption{Nous plaçons le point $P$ à l'abcisse $x$, et le point $Q$ un peu plus loin : en $x+\Delta x$. En vert, la tangente que nous cherchons.}\label{fig_derrun}
\end{figure}
% This is part of Un soupçon de physique, sans être agressif pour autant
% Copyright (C) 2006-2009
%   Laurent Claessens
% See the file fdl-1.3.txt for copying conditions.




% This is part of Un soupçon de physique, sans être agressif pour autant
% Copyright (C) 2006-2009
%   Laurent Claessens
% See the file fdl-1.3.txt for copying conditions.


% Fichier généré automatiquement. Ne pas modifier à la main.
\begin{figure}[ht]
\centering
\psset{xunit=0.5,yunit=0.5}

\subfigure[Pas très bon \ldots]{%
\begin{pspicture}(-0.828427124746,-1.7947331922)(8.0,5.82842712475)
\psset{PointSymbol=none,PointName=none,algebraic=true}

\psplot[]{0.5}{8}{(-3/x)+5}
\pstGeonode[PointSymbol=none,PointName=none](-0.264911064067,-1.7947331922){abq}
\pstGeonode[PointSymbol=none,PointName=none](2.26491106407,5.7947331922){abr}
\pstLineAB[linecolor=green]{abq}{abr}
\pstGeonode[PointSymbol=none,PointName=none](-0.828427124746,0.171572875254){afv}
\pstGeonode[PointSymbol=none,PointName=none](4.82842712475,5.82842712475){afw}
\pstLineAB[linecolor=blue]{afv}{afw}
\pstGeonode[PointSymbol=none,PointName=none](1.0,2.0){aaa}
\pstGeonode[PointSymbol=*](1.0,2.0){aaa}
\rput(aaa){\rput(0.3;135){$P$}}
\pstGeonode[PointSymbol=none,PointName=none](3.0,4.0){aft}
\pstGeonode[PointSymbol=*](3.0,4.0){aft}
\rput(aft){\rput(0.4;-45){$Q_0$}}
\end{pspicture}
}					% Fermeture de la sous-figure 1
%
\subfigure[\ldots de mieux en mieux \ldots]{%
\begin{pspicture}(-0.824122021986,-1.7947331922)(8.0,5.92713727473)
\psset{PointSymbol=none,PointName=none,algebraic=true}

\psplot[]{0.5}{8}{(-3/x)+5}
\pstGeonode[PointSymbol=none,PointName=none](-0.264911064067,-1.7947331922){abq}
\pstGeonode[PointSymbol=none,PointName=none](2.26491106407,5.7947331922){abr}
\pstLineAB[linecolor=green]{abq}{abr}
\pstGeonode[PointSymbol=none,PointName=none](-0.824122021986,-0.0521372747346){ahl}
\pstGeonode[PointSymbol=none,PointName=none](4.49078868865,5.92713727473){ahm}
\pstLineAB[linecolor=blue]{ahl}{ahm}
\pstGeonode[PointSymbol=none,PointName=none](1.0,2.0){aaa}
\pstGeonode[PointSymbol=*](1.0,2.0){aaa}
\rput(aaa){\rput(0.3;135){$P$}}
\pstGeonode[PointSymbol=none,PointName=none](2.66666666667,3.875){ahj}
\pstGeonode[PointSymbol=*](2.66666666667,3.875){ahj}
\rput(ahj){\rput(0.4;-45){$Q_1$}}
\end{pspicture}
}					% Fermeture de la sous-figure 2
%
\subfigure[\ldots de mieux en mieux \ldots]{%
\begin{pspicture}(-0.789095787393,-1.7947331922)(8.0,6.01455172665)
\psset{PointSymbol=none,PointName=none,algebraic=true}

\psplot[]{0.5}{8}{(-3/x)+5}
\pstGeonode[PointSymbol=none,PointName=none](-0.264911064067,-1.7947331922){abq}
\pstGeonode[PointSymbol=none,PointName=none](2.26491106407,5.7947331922){abr}
\pstLineAB[linecolor=green]{abq}{abr}
\pstGeonode[PointSymbol=none,PointName=none](-0.789095787393,-0.300266012362){ajb}
\pstGeonode[PointSymbol=none,PointName=none](4.12242912073,6.01455172665){ajc}
\pstLineAB[linecolor=blue]{ajb}{ajc}
\pstGeonode[PointSymbol=none,PointName=none](1.0,2.0){aaa}
\pstGeonode[PointSymbol=*](1.0,2.0){aaa}
\rput(aaa){\rput(0.3;135){$P$}}
\pstGeonode[PointSymbol=none,PointName=none](2.33333333333,3.71428571428){aiz}
\pstGeonode[PointSymbol=*](2.33333333333,3.71428571428){aiz}
\rput(aiz){\rput(0.4;-45){$Q_2$}}
\end{pspicture}
}					% Fermeture de la sous-figure 3
%
\subfigure[\ldots de mieux en mieux \ldots]{%
\begin{pspicture}(-0.718800784901,-1.7947331922)(8.0,6.07820117735)
\psset{PointSymbol=none,PointName=none,algebraic=true}

\psplot[]{0.5}{8}{(-3/x)+5}
\pstGeonode[PointSymbol=none,PointName=none](-0.264911064067,-1.7947331922){abq}
\pstGeonode[PointSymbol=none,PointName=none](2.26491106407,5.7947331922){abr}
\pstLineAB[linecolor=green]{abq}{abr}
\pstGeonode[PointSymbol=none,PointName=none](-0.718800784901,-0.578201177351){akr}
\pstGeonode[PointSymbol=none,PointName=none](3.7188007849,6.07820117735){aks}
\pstLineAB[linecolor=blue]{akr}{aks}
\pstGeonode[PointSymbol=none,PointName=none](1.0,2.0){aaa}
\pstGeonode[PointSymbol=*](1.0,2.0){aaa}
\rput(aaa){\rput(0.3;135){$P$}}
\pstGeonode[PointSymbol=none,PointName=none](2.0,3.5){akp}
\pstGeonode[PointSymbol=*](2.0,3.5){akp}
\rput(akp){\rput(0.4;-45){$Q_3$}}
\end{pspicture}
}					% Fermeture de la sous-figure 4
%
\subfigure[\ldots de mieux en mieux \ldots]{%
\begin{pspicture}(-0.609238391381,-1.7947331922)(8.0,6.09662910449)
\psset{PointSymbol=none,PointName=none,algebraic=true}

\psplot[]{0.5}{8}{(-3/x)+5}
\pstGeonode[PointSymbol=none,PointName=none](-0.264911064067,-1.7947331922){abq}
\pstGeonode[PointSymbol=none,PointName=none](2.26491106407,5.7947331922){abr}
\pstLineAB[linecolor=green]{abq}{abr}
\pstGeonode[PointSymbol=none,PointName=none](-0.609238391381,-0.896629104486){amh}
\pstGeonode[PointSymbol=none,PointName=none](3.27590505805,6.09662910449){ami}
\pstLineAB[linecolor=blue]{amh}{ami}
\pstGeonode[PointSymbol=none,PointName=none](1.0,2.0){aaa}
\pstGeonode[PointSymbol=*](1.0,2.0){aaa}
\rput(aaa){\rput(0.3;135){$P$}}
\pstGeonode[PointSymbol=none,PointName=none](1.66666666667,3.2){amf}
\pstGeonode[PointSymbol=*](1.66666666667,3.2){amf}
\rput(amf){\rput(0.4;-45){$Q_4$}}
\end{pspicture}
}					% Fermeture de la sous-figure 5
%
\subfigure[\ldots presque parfait.]{%
\begin{pspicture}(-0.457887197547,-1.7947331922)(8.0,6.03024619448)
\psset{PointSymbol=none,PointName=none,algebraic=true}

\psplot[]{0.5}{8}{(-3/x)+5}
\pstGeonode[PointSymbol=none,PointName=none](-0.264911064067,-1.7947331922){abq}
\pstGeonode[PointSymbol=none,PointName=none](2.26491106407,5.7947331922){abr}
\pstLineAB[linecolor=green]{abq}{abr}
\pstGeonode[PointSymbol=none,PointName=none](-0.457887197547,-1.28024619448){anx}
\pstGeonode[PointSymbol=none,PointName=none](2.79122053088,6.03024619448){any}
\pstLineAB[linecolor=blue]{anx}{any}
\pstGeonode[PointSymbol=none,PointName=none](1.0,2.0){aaa}
\pstGeonode[PointSymbol=*](1.0,2.0){aaa}
\rput(aaa){\rput(0.3;135){$P$}}
\pstGeonode[PointSymbol=none,PointName=none](1.33333333333,2.74999999999){anv}
\pstGeonode[PointSymbol=*](1.33333333333,2.74999999999){anv}
\rput(anv){\rput(0.4;-45){$Q_5$}}
\end{pspicture}
}					% Fermeture de la sous-figure 6
%
\caption{Au fur et à mesure que le point $Q_i$ se rapproche de $P$, l'approximation se rapproche de la tangente.}\label{FigTanApproxSuite}
\end{figure}


La tangente est la droite qui touche la courbe en un seul point sans la traverser. Affin de la construire, nous allons dessiner des droites qui touchent la courbe en $P$ et un autre point $Q$, et nous allons voir ce qu'il se passe quand $Q$ est très proche de $P$. Cela donnera une droite qui, certes, touchera la courbe en deux points, mais en deux point \emph{tellement proche que c'est comme si c'étaient les mêmes}. Tu sens que la notion de limite va encore venir.

Nous avons placé le point $P$ en $x=x_P$ et le point $Q$ un peu plus loin $x=x_P+\Delta x$. En d'autres termes leurs coordonnées sont
\begin{align}
	P=\big(x_P,f(x_P)\big)&& Q=\big(x_P+\Delta x,f(x_P+\Delta x)\big).
\end{align}
Comme tu devrais le savoir sans même regarder la figure \ref{fig_derrun}, le coefficient angulaire de la droite qui passe par ces deux points est donné par
\begin{equation}
	\frac{ f(x+\Delta x)-f(x) }{ \Delta x },
\end{equation}
et bang ! Encore le même rapport. Si tu regardes la figure \ref{FigTanApproxSuite}, tu verras que réellement en faisant tendre $\Delta x$ vers zéro on obtient la tangente.



\begin{definition}
Si $f\colon \eR\to \eR$ est une fonction, la \defe{dérivée}{Dérivée d'une fonction} de $f$ au point $x_0$ est le nombre (si il existe)
\begin{equation}
	f'(x_0)=\lim_{\epsilon\to 0}\frac{ f(x_0+\epsilon)-f(x) }{ \epsilon }.
\end{equation}
\end{definition}
Imaginons que $f'(x_0)$ puisse être calculée en chaque point $x_0\in\eR$. Dans ce cas, nous nous retrouvons avec une nouvelle fonction $f'\colon \eR\to \eR$ qui s'appelle la \defe{fonction dérivée}{} de $f$. Nous allons, durant les prochaines pages, étudier comment calculer $f'$.

La dérivée de la fonction $f$ par rapport à $x$ se note aussi souvent par
\begin{equation}
	\frac{ df }{ dx }.
\end{equation}
Cette notation a, au moins, l'avantage de préciser par rapport à quelle variable on dérive (ici : $x$).

%---------------------------------------------------------------------------------------------------------------------------
					\subsection{L'aire en dessous d'une courbe}		\label{SubSecAirePrimInto}
%---------------------------------------------------------------------------------------------------------------------------

Encore un exemple. Nous voudrions bien pouvoir calculer l'aire en-dessous d'une courbe. Nous notons $S_f(x)$ l'aire en-dessous de la fonction $f$ entre l'abcisse $0$ et $x$, c'est à dire l'aire bleue de la figure \ref{fig_surface}. 
\input{fig_surface.pstricks}
Si la fonction $f$ est continue et que $\Delta x$ est assez petit, la fonction ne varie pas beaucoup entre $x$ et $x+\Delta x$. L'augmentation de surface entre $x$ et $x+\Delta x$ peut donc être approximé par le rectangle de surface $f(x)\Delta x$. Ce que nous avons donc, c'est que quand $\Delta x$ est très petit,
\begin{equation}
	S_f(x+\Delta x)-S_f(x)=f(x)\Delta x,
\end{equation}
c'est à dire
\begin{equation}
	f(x)=\lim_{\Delta x\to 0}\frac{  S_f(x+\Delta x)-S_f(x)}{ \Delta x }.
\end{equation}
Donc, la fonction $f$ est la dérivée de la fonction qui représente l'aire en-dessous de $f$. Calculer des surfaces revient donc au travail inverse de calculer des dérivées.

Nous avons déjà vu que calculer la dérivée d'une fonction n'est pas très compliqué. Aussi étonnant que cela puisse paraître, il se fait que le processus inverse est très compliqué : il est en général extrêmement difficile (et même souvent impossible) de trouver une fonction dont la dérivée est une fonction donnée.

Une fonction dont la dérivée est la fonction $f$ s'appelle une \defe{primitive}{Primitive} de $f$, et la fonction qui donne l'aire en-dessous de la fonction $f$ entre l'abcisse $0$ et $x$ est notée
\begin{equation}
	S_f(x)=\int_0^xf(t)dt.
\end{equation}
Nous pouvons nous demander si, pour une fonction $f$ donnée, il existe une ou plusieurs primitives, c'est à dire si il existe une ou plusieurs fonctions $F$ telles que $F'=f$. La réponse sera donnée par le corollaire \ref{CorZeroCst} : si on en connaît une, on les connaît toutes. 

%+++++++++++++++++++++++++++++++++++++++++++++++++++++++++++++++++++++++++++++++++++++++++++++++++++++++++++++++++++++++++++
					\section{Règles de calcul}
%+++++++++++++++++++++++++++++++++++++++++++++++++++++++++++++++++++++++++++++++++++++++++++++++++++++++++++++++++++++++++++

Le calcul de dérivées de fonctions se ramène toujours à un calcul de limite. C'est pas toujours très facile, mais ce n'est jamais un cauchemar\footnote{Un cauchemar ? Pfff ! Ne me fais pas croire que tu rêves de ton cours de math en dormant.}. Faisons-en une ou deux juste pour le plaisir.

\begin{lemma}			\label{LemDeccCarr}
	Si $f(x)=x^2$, alors $f'(x)=2x$.
\end{lemma}

\begin{proof}
	Utilisons la définition, et remplaçons $f$ par sa valeur :
	\begin{subequations}
		\begin{align}
			f'(x)	&=\lim_{\epsilon\to 0}\frac{ f(x+\epsilon)-f(x) }{ \epsilon }\\
				&=\lim_{\epsilon\to 0}\frac{ (x+\epsilon)^2-x^2 }{ \epsilon }\\
				&=\lim_{\epsilon\to 0}\frac{ x^2+2x\epsilon+\epsilon^2-x^2 }{ \epsilon }\\
				&=\lim_{\epsilon\to 0}\frac{\epsilon(2x+\epsilon)}{ \epsilon }\\
				&=\lim_{\epsilon\to 0}(2x+\epsilon)\\
				&=2x,
		\end{align}
	\end{subequations}
	ce qu'il fallait prouver.
\end{proof}

Une facile, maintenant.
\begin{proposition}
	La dérivé de la fonction $x\mapsto x$ vaut $1$, en notations compactes : $(x)'=1$.
\end{proposition}

\begin{proof}
D'après la définition de la dérivée, si $f(x)=x$, nous avons
\begin{equation}
	f(x)=\lim_{\epsilon\to 0}\frac{ (x+\epsilon) -x }{\epsilon} =\lim_{\epsilon\to 0}\frac{ \epsilon }{\epsilon} =1,
\end{equation}
et c'est déjà fini.
\end{proof}

Pour continuer, nous allons en faire une un peu plus abstraite.
\begin{proposition}		\label{PropDerrLin}
	La dérivation est une opération linéaire, c'est à dire que
	\begin{enumerate}
		\item $(\lambda f)'=\lambda f'$ pour tout réel $\lambda$ où, pour rappel, la fonction $(\lambda f)$ est définie par $(\lambda f)(x)=\lambda\cdot f(x)$,
		\item $(f+g)'=f'+g'$.
	\end{enumerate}
\end{proposition}

\begin{proof}
Ces deux propriétés découlent des propriétés correspondantes de la limite. Nous allons faire la première, et laisser la seconde à titre d'exercice. Écrivons la définition de la dérivée avec $(\lambda f)$ au lieu de $f$, et calculons un petit peu :
\begin{equation}
	\begin{aligned}[]
		(\lambda f)'(x)	&=\lim_{\epsilon\to 0}\frac{ (\lambda f)(x+\epsilon)-(\lambda f)(x) }{ \epsilon }\\
				&=\lim_{\epsilon\to 0}\frac{ \lambda \big( f(x+\epsilon) \big)-\lambda f(x) }{ \epsilon }\\
				&=\lim_{\epsilon\to 0}\lambda \frac{ f(x+\epsilon) -f(x) }{ \epsilon }\\
				&=\lambda \lim_{\epsilon\to 0}\frac{ f(x+\epsilon) -f(x) }{ \epsilon }\\
				&=\lambda f'(x).
	\end{aligned}
\end{equation}
\end{proof}

Tu peux te persuader de ce résultat en regardant la tangente de la fonction, comme dessinée sur la figure \ref{fig_tg_parab}.
% Fichier généré automatiquement. Ne pas modifier à la main.
\begin{figure}[ht]
\centering
\begin{pspicture}(-3.0,-0.99)(3.0,9.0)
\psset{PointSymbol=none,PointName=none,algebraic=true}

\psplot[]{-3}{3}{(x)^2}
\pstGeonode[PointSymbol=none,PointName=none](1.0,1.0){apl}
\pstGeonode[PointSymbol=*](1.0,1.0){apl}
\rput(apl){\rput(0.3;135){$P$}}
\psaxes[]{->}(0.0,0.0)(-2.99,-0.99)(3.0,9.0)
\psgrid[gridlabels=0,subgriddiv=0,griddots=5](-2.99,-0.99)(3.0,9.0)
\end{pspicture}

\psset{xunit=1,yunit=1}

\caption{La dérivée à la fonction $x\mapsto x^2$}\label{fig_tg_parab}
\end{figure}
% This is part of Un soupçon de physique, sans être agressif pour autant
% Copyright (C) 2006-2009
%   Laurent Claessens
% See the file fdl-1.3.txt for copying conditions.




\begin{proposition}
	La dérivée d'un produit obéit à la \defe{règle de Leibnitz}{Règle de Leibnitz}\index{Leibnitz}:
	\begin{equation}
		(fg)'(x)=f'(x)g(x)+f(g)g'(x).
	\end{equation}
	Cette règle est souvent écrite sous la forme compacte $(fg)'=f'g+g'f$.
\end{proposition}

\begin{proof}
La définition de la dérivée dit que
\begin{equation}		\label{Eqfgrimeepsfgx}
	(fg)'(x)=\lim_{\epsilon\to 0}\frac{f(x+\epsilon)g(x+\epsilon)-f(x)g(x)}{\epsilon}.
\end{equation}
La subtilité est d'ajouter au numérateur la quantité $-f(x)g(x+\epsilon)+f(x)g(x+\epsilon)$, ce qui est permit parce que cette quantité est nulle\footnote{Le coup d'ajouter et enlever la même chose a déjà été fait durant la démonstration du théorème \ref{Tholimfgabab}. C'est une technique assez courante en analyse.}. Le numérateur de \eqref{Eqfgrimeepsfgx} devient donc
\begin{equation}
	\begin{aligned}[]
f(x+\epsilon)g(x+\epsilon)&-f(x)g(x+\epsilon)+f(x)g(x+\epsilon)-f(x)g(x) \\
			&= g(x+\epsilon)\big( f(x+\epsilon)-f(x) \big)+f(x)\big( g(x+\epsilon)-g(x) \big),
	\end{aligned}
\end{equation}
où nous avons effectué deux mises en évidence. Étant donné que nous avons deux termes, nous pouvons couper la limite en deux :
\begin{equation}
	\begin{aligned}[]
		(fg)'(x)	&=\lim_{\epsilon\to 0}g(x+\epsilon)\frac{ f(x+\epsilon)-f(x) }{\epsilon} 			&+\lim_{\epsilon\to 0}f(x)\frac{ g(x+\epsilon)-g(x) }{\epsilon}\\
				&=\lim_{\epsilon\to 0}g(x+\epsilon)\lim_{\epsilon\to 0}\frac{ f(x+\epsilon)-f(x) }{\epsilon}	&+f(x)\lim_{\epsilon\to 0}\frac{ g(x+\epsilon)-g(x) }{\epsilon},
	\end{aligned}
\end{equation}
où nous avons utilisé le théorème \ref{Tholimfgabab} pour scinder la première limite en deux, ainsi que la propriété \eqref{Eqbutmultlim} pour sortir le $f(x)$ de la limite dans le second terme. Maintenant, dans le premier terme, nous avons évidement\footnote{Pas tout à fait évidemment : selon le théorème \ref{ThoLimCont}, \emph{limite et continuité}, il faut que $g$ soit continue.} $\lim_{\epsilon\to 0}g(x+\epsilon)=g(x)$. Les limites qui restent sont les définitions classiques des dérivées de $f$ et $g$ au point~$x$ :
\begin{equation}
	(fg)'(x)=g(x)f'(x)-f(x)g'(x),
\end{equation}
ce qu'il fallait démontrer.
\end{proof}

%+++++++++++++++++++++++++++++++++++++++++++++++++++++++++++++++++++++++++++++++++++++++++++++++++++++++++++++++++++++++++++
					\section{Démystification du MRUA (première)}		\label{SecDemMRUAun}
%+++++++++++++++++++++++++++++++++++++++++++++++++++++++++++++++++++++++++++++++++++++++++++++++++++++++++++++++++++++++++++

Nous allons maintenant donner une preuve (presque) complète de la formule du MRUA en physique. Pour rappel, lorsqu'un mobile se déplace en ligne droite en partant au repos à l'instant $t=0$ avec une accélération $a$ parcours une distance
\begin{equation}
	x(t)=\frac{ at^2 }{ 2 }.
\end{equation}
Cela est l'équation \eqref{EqMouvAccatc}, expliquée en long et en large dans la section \ref{SecMRUA}. Nous avons donné, dans cette section, une longue et douloureuse preuve de cette formule. Notre but maintenant est d'en trouver une preuve beaucoup plus simple.

Avec la notation de la dérivée, la définition \eqref{Eqvinstlimite} de la vitesse devient simplement
\begin{equation}
	v(t)=x'(t).
\end{equation}
La vitesse est la dérivée de la position. D'autre part, selon la formule \eqref{EqDefAcclvlim}, l'accélération est la dérivée de la vitesse:
\begin{equation}
	a(t)=v'(t)=x''(t).
\end{equation}
Dans le cas d'un MRUA, la fonction $a(t)$ est constante : $a(t)=a$. Donc la vitesse en fonction du temps est une fonction dont la dérivée par rapport au temps est la constante $a$. Sans rentrer dans les détail, tu remarqueras que la fonction $v(t)=at$ remplit la condition\footnote{Maintenant,tu sais pourquoi nous avons annoncé une preuve \emph{presque} complète : rien ne prouve qu'il n'y a pas une autre fonction qui rempli la condition.}.

Maintenant, il nous faut une fonction $x(t)$ telle que $x'(t)=at$. Là encore, une telle fonction n'est pas très compliquée à trouver : $x(t)=at^2/2$.

Et voila, c'est terminé : tout ton cours de cinématique tient en trois lignes. Il suffit de faire deux dérivées inverses.

%+++++++++++++++++++++++++++++++++++++++++++++++++++++++++++++++++++++++++++++++++++++++++++++++++++++++++++++++++++++++++++
					\section{Dérivation et croissance}
%+++++++++++++++++++++++++++++++++++++++++++++++++++++++++++++++++++++++++++++++++++++++++++++++++++++++++++++++++++++++++++

La dérivée d'une fonction a une propriété étonnante : elle dit si la fonction est croissante ou décroissante. La figure \ref{fig_derr_enveloppe} montre une fonction et une série de segments tangents. Étant donné que la courbe est \og collée \fg{} à ses tangentes, tant que les tangentes montent, la fonction monte. Or, une tangente qui monte correspond à une dérivée positive, parce que la dérivée est le coefficient angulaire de la tangente.

\newcommand{\CaptionDerrEnveloppe}{Quelque tangentes à la fonction $x\mapsto x^2/3$. Comme tu le vois, la fonction est \og collée\fg{} à ses tangentes.}
\input{fig_derr_enveloppe.pstricks}

Ce résultat très intuitif peut être prouvé rigoureusement. C'est la tache à laquelle nous allons nous atteler maintenant.

\begin{proposition}
	Si $f$ et $f'$ sont des fonctions continues sur l'intervalle $[a,b]$ et si $f'(x)$ est strictement positive sur $[a,b]$, alors $f$ est croissante sur $[a,b]$.

	De la même manière, si $f'(x)$ est strictement négative sur $[a,b]$, alors $f$ est décroissante sur $[a,b]$.
\end{proposition}

\begin{proof}
	Nous n'allons prouver que la première partie. La seconde partie se prouve en considérant $-f$ et en invoquant alors la première\footnote{Je te laisse méditer cela.}.
	Prenons $x_1$ et $x_2$ dans $[a,b]$ tels que $x_1<x_2$. Par hypothèse, pour tout $x$ dans $[x_1,x_2]$, nous avons
	\begin{equation}
		f'(x)=\lim_{\epsilon\to 0}\frac{ f(x+\epsilon)-f(x) }{\epsilon} >0.
	\end{equation}
	Maintenant, la proposition \ref{PropoLimPosFPos} dit que quand une limite est positive, alors la fonction dans la limite est positive sur un voisinage. En appliquant cette proposition à la fonction
	\begin{equation}
		r(\epsilon)=\frac{ f(x+\epsilon)-f(x) }{ \epsilon },
	\end{equation}
	dont la limite en zéro est positive, nous trouvons que $r(\epsilon)>0$ pour tout $\epsilon$ pas trop éloigné de zéro. En particulier, il existe un $\delta>0$ tel que $\epsilon<\delta$ implique $r(\epsilon)>0$; pour un tel $\epsilon$, nous avons donc
	\begin{equation}
		r(\epsilon)=\frac{ f(x+\epsilon)-f(x) }{ \epsilon }>0.
	\end{equation}
	Étant donné que $\epsilon>0$, nous avons que $f(x+\epsilon)-f(x)>0$, c'est à dire que $f$ est strictement croissante entre $x$ et $x+\delta$.

	Jusqu'ici, nous avons prouvé que la fonction $f$ était strictement croissante dans un voisinage autour de chaque point de $[a,b]$. Cela n'est cependant pas encore tout à fait suffisant pour conclure. Ce que nous voudrions faire, c'est de dire, c'est prendre un voisinage $]a,m_1[$ autour de $a$ sur lequel $f$ est croissante. Donc, $f(m_1)>f(a)$. Ensuite, on prend un voisinage $]m_1,m_2[$ de $m_1$ sur lequel $f$ est croissante. De ce fait, $f(m_2)>f(m_1)>f(a)$. Et ainsi de suite, nous voulons construire des $m_3$, $m_4$,\ldots jusqu'à arriver en $b$. Hélas, rien ne dit que ce processus va fonctionner. Il faut trouver une subtilité. Le problème est que les voisinages sur lesquels la fonction est croissante sont peut-être de plus en plus petit, de telle sorte à ce qu'il faille une infinité d'étapes avant d'arriver à bon port (en $b$).

	Heureusement, nous pouvons drastiquement réduire le nombre d'étapes en nous souvenant du théorème de Borel-Lebesgue (numéro \ref{ThoBOrelLebesgue}). Nous notons par $\mO_x$, un ouvert autour de $x$ tel que $f$ soit strictement croissante sur $\mO_x$. Un tel voisinage existe. Cela fait une infinité d'ouverts tels que
	\begin{equation}
		[a,b]\subseteq\bigcup_{x\in[a,b]}\mO_x.
	\end{equation}
	Ce que le théorème dit, c'est qu'on peut en choisir un nombre fini qui recouvre encore $[a,b]$. Soient $\{ \mO_{x_1},\ldots,\mO_{x_n} \}$, les heureux élus, que nous supposons prit dans l'ordre : $x_1<x_2<\ldots<x_n$. Nous avons
	\begin{equation}
		[a,b]\subseteq\bigcup_{i=1}^n\mO_i.
	\end{equation}
	Quitte à les rajouter à la collection, nous supposons que $x_1=a$ et que $x_n=b$. Maintenant nous allons choisir encore un sous ensemble de cette collection d'ouverts. On pose $\mA_1=\mO_{x_1}$. Nous savons que $\mA_1$ intersecte au moins un des autres $\mO_{x_i}$. Cette affirmation vient du fait que $[a,b]$ est connexe (proposition \ref{PropInterssiConn}), et que si $\mO_{x_1}$ n'intersectait personne, alors 
	\begin{equation}
		\begin{aligned}[]
			\mO_{x_1}&&\text{et}&&\bigcup_{i=2}^n\mO_{x_i}
		\end{aligned}
	\end{equation}
	forment une partition de $[a,b]$ en deux ouverts disjoints, ce qui n'est pas possible parce que $[a,b]$ est connexe. Nous nommons $\mA_2$, un des ouverts $\mO_{x_i}$ qui intersecte $\mA_1$. Disons que c'est $\mO_k$. Notons que $\mA_1\cup\mA_2$ est un intervalle sur lequel $f$ est strictement croissante. En effet, si $y_{12}$ est dans l'intersection, $f(a)<f(y_{12})$ parce que $f$ est strictement croissante sur $\mA_1$, et pour tout $x>y_{12}$ dans $\mA_2$, $f(x)>f(y_{12})$ parce que $f$ est strictement croissante dans $\mA_2$. 

	Maintenant, nous éliminons de la liste des $\mO_{x_i}$ tous ceux qui sont inclus à $\mA_1\cup\mA_2$. Dans ce qu'il reste, il y en a automatiquement un qui intersecte $\mA_1\cup\mA_2$, pour la même raison de connexité que celle invoquée plus haut. Nous appelons cet ouvert $\mA_3$, et pour la même raison qu'avant, $f$ est strictement croissante sur $\mA_1\cup\mA_2\cup\mA_3$.

	En recommençant suffisamment de fois, nous finissons par devoir prendre un des $\mO_{x_i}$ qui contient $b$, parce qu'au moins un des $\mO_{x_i}$ contient $b$. À ce moment, nous avons finit la démonstration.

	Ouf !
\end{proof}

Il est intéressant de noter que ce théorème concerne la croissance d'une fonction sous l'hypothèse que la dérivée est positive. Il nous a fallu très peu de temps, en utilisant la positivité de la dérivée, pour conclure qu'autour de tout point, la fonction était strictement croissante. À partir de là, c'était pour ainsi dire gagné. Mais il a fallu un réel travail de topologie très fine et subtile\footnote{et je te rappelle que nous avons utilisé la proposition \ref{PropInterssiConn}, qui elle même était déjà un très gros boulot !} pour conclure. Étonnant qu'une telle quantité de topologie soit nécessaire pour démontrer un résultat essentiellement analytique dont l'hypothèse est qu'une limite est positive, n'est-ce pas ? 

À quoi à servi toute cette topologie ? La réponse est simple : l'hypothèse de dérivée positive est une hypothèse locale. C'est à dire que l'on a une hypothèse qui concerne une limite, qui, par essence, ne lient que des points très proches : dire \og $\lim_{x\to a}f(x)=b$\fg{}  ne donne des informations sur $f(x)$ que pour des $x$ très proches de $a$. Toute la topologie a servi à transporter un résultat local vrai partout (la fonction est croissante dans un voisinage de tout point) un un seul résultat global : la fonction est croissante sur tout l'intervalle $[a,b]$.

La topologie a servi à recoller des morceaux. Cela est une philosophie assez générale en analyse. Une hypothèse analytique permet de monter un résultat sur plein de tout petit intervalles, et puis de la topologie vient en déduire un résultat sur un intervalle plus grand.

Une petite facile, maintenant.

\begin{proposition}
	Si $f$ est croissante sur un intervalle, alors $f'\geq 0$ à l'intérieur cet intervalle, et si $f$ est décroissante sur l'intervalle, alors $f'\leq 0$ à l'intérieur de l'intervalle.
\end{proposition}

Note qu'ici, nous demandons juste la croissance de $f$, et non sa \emph{stricte} croissance.

\begin{proof}
	Soit $f$, une fonction croissante sur l'intervalle $I$, et $x$ un point intérieur de $I$. La dérivée de $f$ en $x$ vaut
	\begin{equation}
		f'(x)=\lim_{\epsilon\to 0}\frac{ f(x+\epsilon)-f(x) }{\epsilon},
	\end{equation}
	mais, comme $f$ est croissante sur $I$, nous avons toujours que $f(x+\epsilon)-f(x)\geq0$ quand $\epsilon>0$, et $f(x+\epsilon)-f(x)\leq0$ quand $\epsilon<0$, donc cette limite est une limite de nombre positifs ou nuls, qui est donc positive ou nulle. Cela prouve que $f'(x)\geq 0$.
\end{proof}

% http://fr.wikipedia.org/wiki/Théorème_de_Rolle
% http://gconnan.free.fr/les%20pdf/Deriv.pdf
Les deux prochains théorèmes sont très importants.
\begin{theorem}[\href{http://fr.wikipedia.org/wiki/Théorème_de_Rolle}{Théorème de Rolle}]		\label{ThoRolle}
	Soit $f$, une fonction continue sur $[a,b]$ et dérivable sur $]a,b[$. Si $f(a)=f(b)$, alors il existe un point $c\in]a,b[$ tel que $f'(c)=0$.
\end{theorem}
\newcommand{\CaptionRolle}{Illustration du théorème de Rolle. Nous avons $f(a)=f(b)$, et effectivement, au point $c$, la tangente est horizonale (dérivée nulle).}
\input{figure_Rolle_Rolle.pstricks}

\begin{proof}
	Une petite illustration est présentée sur la figure \ref{FigRolle}.
	Étant donné que $[a,b]$ est un intervalle compact, l'image de $[a,b]$ par $f$ est un intervalle compact, soit $[m,M]$ (théorème \ref{ThoImCompCotComp}). Si $m=M$, alors le théorème est évident : c'est que la fonction est constante, et la dérivée est par conséquent nulle. Supposons que $M> f(a)$ (il se peut que $M=f(a)$, mais alors si $f$ n'est pas constante, il faut avoir $m<f(a)$ et le reste de la preuve peut être adaptée).

	Comme $M$ est dans l'image de $[a,b]$ par $f$, il existe $c\in ]a,b[$ tel que $f(c)=M$. Considérons maintenant la fonction
	\begin{equation}
		\tau(x) =\frac{ f(c+x)-f(c) }{ x }.
	\end{equation}
	Par définition, $\lim_{x\to 0}\tau(x)=f'(c)$. Par hypothèse, si $u<c$,
	\begin{equation}
		\tau(u-c) = \frac{ f(u)-f(c) }{ u-c }>0
	\end{equation}
	parce que $u-c<0$ et $f(u)-f(c)<0$. Par conséquent, $\lim_{x\to 0}\tau(x)\geq 0$. Nous avons aussi, pour $v>c$,
	\begin{equation}
		\tau(v-c) = \frac{ f(v)-f(c) }{ v-c }<0
	\end{equation}
	parce que $v-c>0$ et $f(v)-f(c)<0$. Par conséquent, $\lim_{x\to 0}\tau(x)\leq 0$. Mettant les deux ensemble, nous avons $f'(c)=\lim_{x\to 0}\tau(x)=0$, et $c$ est le point que nous cherchions.
\end{proof}

Sur wikipédia, deux démonstrations complètement différentes sont proposées, celle qui est présentée ici est adaptée de celle qui est proposée par le célèste mathémator de \href{http://gconnan.free.fr/les\%20pdf/Deriv.pdf}{Téhessin le Rézéen}.

Le corollaire suivant est le théorème des \defe{accroissements finis}{Théorème!accroissement fini}.

\begin{theorem}[accroissements finis]		\label{ThoAccFinis}
	Si $f$ est une fonction continue sur $[a,b]$ et dérivable sur $]a,b[$, alors il existe au moins un réel $c\in]a,b[$ tel que $f(b)-f(a)=(b-a)f'(c)$.
\end{theorem}

\newcommand{\CaptionAccfinis}{Illustration du théorème des accroissements finis. Nous avons effectivement qu'au point $c$, la tangente est parallèle au segment qui joint $a$ et $b$.}
\input{figure_Rolle_AccFinis.pstricks}

\begin{proof}
	Regardes l'illustration sur la figure \ref{FigAccFinis}. Considérons la fonction
	\begin{equation}
		\tau(x)=f(x)-\big( \frac{ f(b)-f(a) }{ b-a }x + f(a) - a\frac{ f(b)-f(a) }{ b-a } \big),
	\end{equation}
	c'est à dire la fonction qui donne la distance entre $f$ et le segment de droite qui lie $(a,f(a))$ à $(b,f(b))$. Par construction (et tu peux le vérifier), $\tau(a)-\tau(b)-=0$, donc le théorème de Rolle s'appliqe à $\tau$ pour laquelle il existe donc un $c\in]a,b[$ tel que $\tau'(c)=0$.

	En utilisant les règles de dérivation, nous trouvons que la dérivée de $\tau$ vaut
	\begin{equation}
		\tau'(x)= f'(x)-\frac{ f(b)-f(a) }{ b-a },
	\end{equation}
	donc dire que $\tau'(c)=0$ revient à dire que $f(b)-f(a)=(b-a)f'(c)$, ce qu'il fallait démontrer.
\end{proof}

\begin{corollary}
Soit $f$ une fonction dérivable sur $[a,b]$ telle que $f'(x) = 0$ pour tout $x \in [a,b]$. Alors $f$ est constante sur $[a,b]$.
\end{corollary}

\begin{proof}
	Si $f$ n'était pas constante sur $[a,b]$, il existerait un $x_1\in ]a,b[$ tel que $f(a)\neq f(x_1)$, et dans ce cas, il existerait un $c\in]a,x_1[$ tel que 
	\begin{equation}
		f'(c)=\frac{ f(x_1)-f(a) }{ x_1-a }\neq 0,
	\end{equation}
	ce qui contredirait les hypothèses.
\end{proof}

\begin{corollary}
	Soit $f$ et $g$, deux fonctions dérivables sur $[a,b]$ telles que
	\begin{equation}
		f'(x) = g'(x)
	\end{equation}
	pour tout $x \in [a,b]$. Alors existe un réel $C$ tel que $f (x) = g (x) + C$ pour tout $x\in [a,b]$.
\end{corollary}

\begin{proof}
	Considérons la fonction $h(x)=f(x)-g(x)$, dont la dérivée est, par hypothèse, nulle. L'annulation de la dérivée entraine que $h$ est  constante. Si $h(x)=C$, alors $f(x)=g(x)+C$, ce qu'il fallait prouver.
\end{proof}
\addtocounter{numtho}{-1}

Exprimé en termes des primitives introduites dans la sous-section \ref{SubSecAirePrimInto}, ce corollaire signifie que
\begin{corollary}[bis]	\label{CorZeroCst}
	Si $F$ et $G$ sont deux primitives de la même fonction $f$, alors il existe une constante $C$ pour laquelle $F(x)=G(x)+C$.
\end{corollary}
Cela signifie qu'il n'y a, en réalité, pas des milliards de primitives différentes à une fonction. Il y en a essentiellement une seule, et puis les autres, ce sont juste les mêmes, mais décalées d'une constante.

%+++++++++++++++++++++++++++++++++++++++++++++++++++++++++++++++++++++++++++++++++++++++++++++++++++++++++++++++++++++++++++
					\section{Démystification du MRUA (seconde)}
%+++++++++++++++++++++++++++++++++++++++++++++++++++++++++++++++++++++++++++++++++++++++++++++++++++++++++++++++++++++++++++

%---------------------------------------------------------------------------------------------------------------------------
					\subsection{Preuve de la formule}
%---------------------------------------------------------------------------------------------------------------------------

Nous sommes maintenant en mesure de donner une démonstration complète de la formule du MRUA :
\begin{equation}	\label{EqMRUAINT}
	x(t) = \frac{ at^2 }{ 2 } + v_0t +x_0.
\end{equation}

Au niveau de la physique, nous considérons un mobile qui se déplace avec une accélération constante $a$. Nous notons par $v_0$ sa vitesse initiale et par $x_0$ sa position initiale.

Reprenons le cheminement suivit dans la section \ref{SecDemMRUAun}. Nous savons que, pour tout mouvement, si $x(t)$ est la position en fonction du temps, et si $v(t)$ et $a(t)$ représentent la vitesse et l'accélération en fonction du temps, alors
\begin{equation}
	\begin{aligned}[]
		v(t)&=x'(t)&\text{et}	&&a(t)=v'(t)=x''(t).
	\end{aligned}
\end{equation}
Affin de trouver $x(t)$ en connaissant $a(t)$, il \og suffit\fg{} donc de prendre deux fois la primitive. Essayons ça dans le cas facile du MRUA où $a(t)=a$ est constante.

La vitesse $v(t)$ doit être une primitive de la constante $a$. Il est facile de voir que $v(t)=at$ est une primitive de $a$. Par le corollaire \ref{CorZeroCst}(bis),
\begin{equation}	\label{EqvtatC}
	v(t)=at+C_1
\end{equation}
pour une certaine constante $C_1$. Affin de fixer $C_1$, il faut faire appel à la physique : d'après la formule \eqref{EqvtatC}, la vitesse initiale est $v(0)=C_1$. Donc il faut identifier $C_1$ à la vitesse initiale : $C_1=v_0$. Nous avons donc déjà obtenu que
\begin{equation}
	v(t)=at+v_0.
\end{equation}
Affin de trouver $x(t)$, il faut trouver une primitive de $v(t)$. Il n'est pas très difficile de voir que $at^2/2 + v_0t$ fonctionne, donc il existe une constante $C_2$ telle que
\begin{equation}
	x(t)=\frac{ at^2 }{ 2 }+v_0t+C_2.
\end{equation}
Encore une fois, regardons la condition initiale : la formule donne comme position initiale $x(0)=C_2$, et donc nous devons identifier $C_2$ avec la position initiale $x_0$. En définitive, nous avons bien
\begin{equation}
	x(t) = \frac{ at^2 }{ 2 } + v_0t +x_0.
\end{equation}

Cette formule est donc maintenant \emph{démontrée} à partir de la seule définition de la vitesse comme dérivée de la position et de l'accélération comme dérivée de la vitesse. 

Remarquons cependant que la preuve complète fut \emph{très} longue. En effet, nous avons utilisé les règles de dérivation 
\ref{PropDerrLin} et \ref{LemDeccCarr}, pour la démonstration desquels, les résultats \ref{ThoLimLin} et \ref{ThoLimLinMul} ont étés utiles. Mais nous avons surtout utilisé le corollaire \ref{CorZeroCst}(bis) qui repose sur le théorème de Rolle \ref{ThoRolle}, qui lui-même demande le théorème de Borel-Lebesgue \ref{ThoBOrelLebesgue} dans lequel la notion d'ensemble compact a été cruciale.

%---------------------------------------------------------------------------------------------------------------------------
					\subsection{Interprétation graphique}
%---------------------------------------------------------------------------------------------------------------------------

La distance parcourue $x(t)$ en un temps $t$ est la primitive de la vitesse. Nous avons, par ailleurs, vu au point \ref{SubSecAirePrimInto} que l'opération inverse de la dérivée donnait la surface. Pour reprendre les mêmes notations, nous notons $S_v(t)$ la surface contenue en-dessous de la fonction $v$ entre $0$ et $x$. Nous ne serions donc pas étonné que
\begin{equation}		\label{EqEncoreMRUASvt}
	S_v(t) = \frac{ at^2 }{ 2 }+v_0t+x_0
\end{equation}
\newcommand{\CaptionSurfMRUA}{La surface bleue est un triangle de base $t$ et de hauteur $at$, tandis que le rectangle rouge est de base $t$ et de hauteur $v_0$.}
\input{fig_surfMRUA.pstricks}
soit la surface en-dessous de la fonction $v(t)=at+v_0$. Regardons cela sur la figure~\ref{FigSurfMRUA}. Nous voyons que la surface totale sous la fonction $v(t)=at+v_0$ est exactement
\begin{equation}
	S_v(t)=\frac{ at^2 }{ 2 }+v_0t.
\end{equation}
Cela est un bon début, mais hélas nous ne retrouvons pas le terme \og $+x_0$\fg{} de la formule \eqref{EqEncoreMRUASvt}. Cela n'est pas tout à fait étonnant parce que au point \ref{SubSecAirePrimInto}, nous n'avons que montré que la surface sous une fonction était \emph{une} primitive de la fonction, mais nous n'avons pas dit \emph{laquelle}. D'après le fameux corollaire \ref{CorZeroCst}(bis), la primitive n'est définie qu'à une constante près. Ici, c'est la constante $x_0$ qu'on a perdue en chemin.

Nous parlerons plus en détail du lien entre les surfaces et les primitives dans la section dédié à l'intégration.


\label{LaFin}
