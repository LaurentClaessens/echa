% This is part of Un soupçon de physique, sans être agressif pour autant
% Copyright (C) 2006-2009
%   Laurent Claessens
% See the file fdl-1.3.txt for copying conditions.


\begin{corrige}{006}

D'abord, comme dans tout problème ne faisant entrer que la gravitation en ligne de compte, la réponse ne dépend pas de la masse. La pierre de \unit{5}{\kilogram} et celle de \unit{10}{\kilogram} prendront un temps identique pour parcourir leur trajet vertical en chute libre.

En chute libre, la distance parcourue en un temps $t$ est donné par la formule \eqref{EqMouvAccatc} dans laquelle il faut remplacer $a$ par $g$. Dans le cas de la pierre en chute libre, nous supposons que la vitesse initiale est nulle, de telle manière à ce qu'il faille résoudre l'équation (pour $t$)
\[ 
  d=\frac{ gt^2 }{ 2 }
\]
avec $d=\unit{52}{\kilo\meter}=\unit{52000}{\meter}$. La solution est donnée par
\[ 
  t=\sqrt{\frac{ 2d }{ g }}=\unit{103}{\second}.
\]
Soit environ une minute et 45 secondes. Inutile de préciser que ce temps bat de très loin les meilleurs sportifs !

\end{corrige}
