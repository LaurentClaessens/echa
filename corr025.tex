\begin{corrige}{025}

L'erreur à ne pas commettre est de dire \og j'ai une pierre de \unit{5}{\kilo\gram} et donc une force de \unit{50}{\newton}, ce qui est équivalent à l'exercice précédent\fg. Pourquoi est-ce faux ? Parce que la force de \unit{50}{\newton} s'applique cette fois-ci non seulement au chariot, mais {\bf également à la pierre elle-même}. La force tire donc $\unit{5+2.5=7.5}{\kilo\gram}$ et non seulement les deux kilos et demi du chariot !

Le système accéléra donc moins qu'avant. Cependant, le chercheur d'or se fatigue moins parce que maintenant c'est la gravitation qui travaille.

En ce qui concerne les calculs, la situation est tout de même similaire : une force de \unit{50}{\newton} qui se déplace de \unit{10}{\meter} fourni une énergie de \unit{500}{\joule}. Ces joules servent à faire avancer la pierre plus le chariot dont la masse totale est \unit{7.5}{\kilo\gram}. Donc
\[ 
  v=\sqrt{ \frac{ 2E_C }{ m } }=\sqrt{ \frac{ 500 }{ 7.5 } }=\unit{8.164}{\meter\per\second}.
\]


\end{corrige}
