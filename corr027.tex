\begin{corrige}{027}

La pierre subit deux forces : l'une est celle de la gravitation qui la tire vers le bas et le frottement de l'eau qui pousse vers le haut. Comme elle tombe à vitesse constante, c'est que ces deux forces sont égales en normes et opposées en sens. Quand on pèse \unit{5}{\kilo\gram}, la force de pesenteur qui s'applique est
\[ 
  P=mg= \unit{5\cdot 9.81=49}{\newton}.
\]
Les forces de frottements doivent donc valoir également \unit{49}{\newton}.
\end{corrige}
% This is part of Un soupçon de physique, sans être agressif pour autant
% Copyright (C) 2006-2009
%   Laurent Claessens
% See the file fdl-1.3.txt for copying conditions.


