% This is part of Un soupçon de physique, sans être agressif pour autant
% Copyright (C) 2006-2009
%   Laurent Claessens
% See the file fdl-1.3.txt for copying conditions.


\newcommand{\prefigzerodeuxzero}{%
   \pstGeonode(0,0){A}(4,0){C}(4,0){B}

\pstHomO[HomCoef=0.3]{A}{B}[Oa]			% Place de la première roue
\pstHomO[HomCoef=0.4]{A}{B}[Pa]			% Place de la seconde roue
\pstRotation[RotAngle=90]{Oa}{B}[Obu]
\pstHomO[HomCoef=0.07]{Oa}{Obu}[Ob] 		% Rayon des roues
\pstMiddleAB{Oa}{Ob}{Oc}

   \pstTranslation{Oa}{Pa}{Ob,Oc}[Pb,Pc]	

\pstHomO[HomCoef=1.5]{Pc}{Oc}[Cg]
\pstHomO[HomCoef=1.5]{Oc}{Pc}[Cd]		% Longueur du chariot
\pstTranslation{Oa}{Ob}{Cd}[pCdh]
\pstTranslation{Oa}{Ob}{Cg}[pCgh]
\pstHomO[HomCoef=1.5]{Cd}{pCdh}[Cdh]		% Hauteur du chariot
\pstHomO[HomCoef=1.5]{Cg}{pCgh}[Cgh]


\pstMiddleAB{Cg}{Cdh}{Cc}

\rput(Cc){\pstGeonode(0,-1.8){bG}}		% Position et plaçage de G

\pstTranslation{Oa}{Ob}{Cc}[pbR]
\pstHomO[HomCoef=6]{Cc}{pbR}[bR]		% Longueur de F_2

\pstTransHom{Oc}{Pc}{Cc}{4}{bF}
\pstSymO{Cc}{bF}[bFp]

\pstHomO[HomCoef=1.5]{Oa}{B}[prAB]
\pstRotation[RotAngle=20]{B}{prAB}[prABr]
\pstHomO[HomCoef=0.3]{B}{prABr}[bT]		% Position de la tige
\pstHomO[HomCoef=0.45]{B}{prABr}[cP]		% Centre de la poulie
\pstInterCC{Cdh}{cP}{cP}{bT}{Iu}{Id}		% Tangentes à la poulie...

\rput(cP){\pstGeonode(2,0){bdmtV}}		% Construire une horizontale au centre de la poulie
\pstInterLC{cP}{bdmtV}{cP}{bT}{IIu}{tV}
\pstTransHom{Cc}{bG}{tV}{0.6}{bC}		% Construire le bout de la corde par translation et homothetie de BC
\rput(bC){\pstGeonode(0,-0.7){bGd}}

}
\begin{figure}[h]
\centering
\begin{pspicture}(-0.5,-2)(4,2.5)
  \psset{PointSymbol=none, PointName=none}
\prefigzerodeuxzero				% La géométrie de la construction est contenue dans cette commande
   \psline(A)(C)
   \psline(C)(B)
   \psline(A)(B)
  
   \pstCircleAB{Oa}{Ob}
   \pstCircleAB{Pa}{Pb}

   \rput(Cdh){\psdot}				% Marquer l'endroit où la corde va venir.

   \psline(Cg)(Cgh)
   \psline(Cgh)(Cdh)
   \psline(Cd)(Cdh)
   \psline(Cg)(Cd)


   \psline[arrows=->](Cc)(bR)
   \rput(bR){\rput(0.4;30){$\fF_2$}}		% Position de la marque de F_2
   \pstMarqueForce{Cc}{bF}{0.6;60}{$\fF_4$}

   \pstMarqueForce{Cc}{bG}{0.4;0}{$\fG_1$}

   \pstMarqueForce{Cc}{bFp}{0.3;90}{$\fF_1$}

   \psline(B)(bT)
   \pstCircleOA{cP}{bT}				% Trace la poulie

   \psline[linestyle=dashed](Cdh)(Iu)

   \psline[linestyle=dashed](tV)(bC)		% Dessiner la corde

   \psellipse[fillstyle=solid,fillcolor=lightgray](bC)(0.5,0.2)


   \pstMarqueForce{bC}{bGd}{0.3,0}{$\fG_2$}

\end{pspicture}
\caption{Chariot tiré par une masse pour l'exercice \ref{exo:chariotplat}. La corde est dessinée en trait discontinu.}\label{fig:chariotplat}
\end{figure}

\begin{exercice}		\label{exo:chariotplat}\label{exo020}
 
Tu te souviens du chariot de la figure \ref{fig_valeurerg} ? Prenons le cas un peu plus général de la figure \ref{fig:chariotplat}. Après avoir relu ce qu'on a fait dans le premier cas, calcule le travail de toutes les forces dessinées lors d'un déplacement d'une longueur $l$ vers la droite.

Sans faire plus de calculs, donne les traveaux des mêmes forces si le chariot se déplace d'une distance $s$ vers la \emph{gauche}.
\end{exercice}
