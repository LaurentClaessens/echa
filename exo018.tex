% This is part of Un soupçon de physique, sans être agressif pour autant
% Copyright (C) 2006-2009
%   Laurent Claessens
% See the file fdl-1.3.txt for copying conditions.



\begin{figure}[h]
\centering
\begin{pspicture}(-0.5,0)(4,2)

   \psset{PointSymbol=none, PointName=none}
   \prefigzerounhuit
   \psline(A)(C)
   \psline(C)(B)
   \psline(A)(B)
   \pstCircleAB[fillstyle=crosshatch,fillcolor=black]{Oa}{Ob}
   \pstMarqueForce{Cc}{bF}{0.3;270}{$\fF$}
\end{pspicture}
\caption{Travail d'une force mal plac\'ee pour l'exercice \ref{exo:incliun}.}\label{fig:incliun}
\end{figure}



\begin{exercice} \label{exo:incliun}\label{exo018}
Comme montré sur la figure \ref{fig:incliun}, quelqu'un voudrais faire monter par sa boule de bowling la pente $AB$ en exerçant une force constante horizontale $\fF$. Que vaut le travail effectu\'e entre $A$ et $B$ ? 

La personne qui est en train de tirer cet objet ne le fait pas de la manière la plus int\'eligente du monde. Comment devrait-t-il s'y prendre ?

\corrref{018}
\end{exercice}
