% This is part of Un soupçon de physique, sans être agressif pour autant
% Copyright (C) 2006-2009
%   Laurent Claessens
% See the file fdl-1.3.txt for copying conditions.


\begin{corrige}{030}

Premier réflexe : convertir la vitesse en des unités du système international : \unit{180}{\kilo\meter\per\hour}=\unit{50}{ \meter\per\second}.

Étant donné que l'on connaît la vitesse et la puissance du moteur, la \og petite formule sympa\fg{} \eqref{eq_puissFv} permet de trouver la force du moteur :
\[ 
  F=\frac{ P }{ v }=\frac{ 110000 }{ 50 }=\unit{2200}{\newton}.
\]
Étant donné qu'on parle d'une vitesse constant, c'est que les forces de frottements compensent exactement la force du moteur. À partir de maintenant, l'idée d'égaliser les forces de frottements à la force d'un moteur quand on dit que la vitesse est constante devrait te venir automatiquement en tête. Donc les forces de frottements s'élèvent également à \unit{2200}{\newton}.

\end{corrige}
