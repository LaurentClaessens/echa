% This is part of Un soupçon de physique, sans être agressif pour autant
% Copyright (C) 2006-2009
%   Laurent Claessens
% See the file fdl-1.3.txt for copying conditions.


\begin{corrige}{211}

Si $P(x)$ est un polynôme de degré impair en $x$, alors $\lim_{x\to-\infty}P(x)=-\infty$, tandis que $\lim_{x\to\infty}P(x)=\infty$. Cela prouve que $P$ est négatif à un moment et positif à un autre moment, et donc le théorème des valeurs intermédiaires impose à $P$ d'être nul entre les deux parce que un polynôme est toujours continu.

\end{corrige}
