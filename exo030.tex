% This is part of Un soupçon de physique, sans être agressif pour autant
% Copyright (C) 2006-2009
%   Laurent Claessens
% See the file fdl-1.3.txt for copying conditions.


\begin{exercice}\label{exo030}

Un automobiliste peu respectueux du code de la route, peu soucieux de sa propre vie ou de celle des autres\footnote{et du risque d'amende : de 55 à 2750 euros (montant doublé si trois récidives dans l'année), déchéance du permis de conduite entre 8 jours et 5 ans. \url{www.jesuispour.be/}. Sans compter les conséquences écologiques d'un tel comportement.} avance à \unit{180}{\kilo\meter\per\hour} avec une voiture dont la masse scandaleusement élevée vis-à-vis de Kyoto vaut \unit{2000}{\kilogram} sur une route horizontale, tout en émettant \unit{225}{\gram} de $CO_2$ par kilomètre\footnote{Toute cette masse n'est pas à soustraire de la masse du véhicule parce qu'une bonne partie (exercice de chimie : combien ?) provient de l'air. En effet, l'essence fournit le carbone qui est accroché à des atomes d'oxygènes pris dans l'air. Voila pourquoi il n'est pas du tout exagéré de dire que les voitures brûlent l'oxygène que l'on respire.}. La puissance du moteur est de \unit{110}{\kilo\watt}. Quelle est la force du moteur ? Quelles sont les forces de frottements ?

Et je n'invente rien : ce sont les caractéristiques de la Chrysler Grand Voyager fin 2007. 

\corrref{030}
\end{exercice}
