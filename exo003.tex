\begin{exercice}  \label{exo:mouv2}\label{exo003}
Un train traverse une garre lorsque un voyageur laisse tomber son gsm.
\begin{enumerate}
\item Vu du voisin assis dans le train, quelle sera la forme de la trajectoire (droite, courbe, vers gauche, la droite ?).
\item Une personne assise sur la quai observe, lui aussi, la chute du GSM. Quelle est la forme de la trajectoire du GSM que cette personne va observer ?
\item Quand tu auras trouvé que la réponse est $y=\frac{ g }{ 2v_0 }x^2$, fais une étude de cette fonction (comme au cours de math : continuité, dérivée première et seconde et tout ça) et interprète physiquement les résultats.
\end{enumerate}
\corrref{003}
\end{exercice}
% This is part of Un soupçon de physique, sans être agressif pour autant
% Copyright (C) 2006-2009
%   Laurent Claessens
% See the file fdl-1.3.txt for copying conditions.


