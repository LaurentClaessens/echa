% This is part of Un soupçon de physique, sans être agressif pour autant
% Copyright (C) 2006-2009
%   Laurent Claessens
% See the file fdl-1.3.txt for copying conditions.


\begin{exercice}\label{exo207}

Trouve des exemples d'ensembles $A$ et de points $a$ pour les situations suivantes (ou précise quand ce n'est pas possible) :
\begin{center}
\begin{tabular}{l|c|c|c|r}
$a$ est un	&	majorant&	supremum	&	maximum	& de $A$\\\hline
		&	oui	&	oui		&	oui	\\ 
		&	oui	&	oui		&	non	\\ 
		&	oui	&	non		&	non	\\ 
		&	oui	&	non		&	oui	\\ 
		&	non	&	non		&	oui	\\ 
		&	non	&	non		&	non	\\ 
		&	non	&	oui		&	non	\\ 
		&	non	&	oui		&	oui	\\ 
\end{tabular}
\end{center}
Par exemple la seconde ligne te demande un exemple d'ensemble $A$ et de point $a$ tels que $a$ soit un majorant de $A$, qu'il soit un supremum, mais pas un maximum.
\corrref{207}
\end{exercice}
