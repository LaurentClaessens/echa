% This is part of Un soupçon de physique, sans être agressif pour autant
% Copyright (C) 2006-2009
%   Laurent Claessens
% See the file fdl-1.3.txt for copying conditions.


\begin{figure}[ht]
\centering
\begin{pspicture}(-1,-4)(1,1)
\psset{PointSymbol=none, PointName=none}
   \pstGeonode(-0.5,0){tCd}(0,1){A}(0,-1){B}		% A et B ne servent qu'à la sym orthogonale pour l'axe verticlal.
   \pstOrtSym{A}{B}{tCd}[tCu]
   \pstCircleAB{tCd}{tCu}
\rput(tCu){\pstGeonode(0,-1){hcB}}			% La position de la première masse
   \psline[linestyle=dashed](tCu)(hcB)
   \pstBoite{hcB}{0.5}{0.7}{cbu}{bbu}
\rput(bbu){\pstGeonode(0,-1){hcBd}}			% La position de la seconde masse
   \psline[linestyle=dashed](bbu)(hcBd)			% Relier la seconde masse
   \pstBoite{hcBd}{0.5}{0.7}{cbd}{bbd}

% La partie gauche

\rput(tCd){\pstGeonode(0,-1.5){hcBt}}			% La position de la première masse
   \psline[linestyle=dashed](tCd)(hcBt)
   \pstBoite{hcBt}{0.5}{0.7}{cbut}{bbut}
\end{pspicture}
\caption{Des masses pendues à une poulie pour l'exercice \ref{exo:poulie}. La corde est dessin\'ee en trait discontinu.}\label{fig:exo:poulie}
\end{figure}

\begin{exercice}\label{exo023}\label{exo:poulie}
Dessinez toutes les forces en présence dans la situaion de la figure \ref{fig:exo:poulie}. Quel est le travail de chacune de ces forces pour un déplacement $h$ ? Nous suppons que toutes les masses sont égales et vallent $m$.
\corrref{023}
\end{exercice}
