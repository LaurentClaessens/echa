% This is part of Un soupçon de physique, sans être agressif pour autant
% Copyright (C) 2006-2009
%   Laurent Claessens
% See the file fdl-1.3.txt for copying conditions.


\section{Généralité sur les courants électriques}
%++++++++++++++++++++++++++++++++++++++++++++++++

\section{Résistances}
%+++++++++++++++++++

\begin{exercice}
	Lorsqu'un informaticien développant des logiciels pour des GSM travaille, il optimise son code pour que le GSM consomme le moins possible\footnote{Le monde de l'optimisation de code informatique est passionnant. Il faut savoir que la façon de diminuer la \emph{consommation} n'est pas spécialement la même que celle pour diminuer les \emph{temps de calcul}.}, affin d'augmenter l'autonomie de la batterie.
	
	Un développeur travaille sur un GSM dont la batterie fait $\unit{13}{\milli\ampere\hour}$. Suite à une optimisation du code, le GSM passe d'une consommation de $\unit{3}{\watt}$ à $\unit{2.5}{\watt}$. Combien de temps le GSM peut resté allumé en plus ?
	
	Le GSM fonctionne avec une tension de $\unit{3.6}{\volt}$.
\end{exercice}



