% This is part of Un soupçon de physique, sans être agressif pour autant
% Copyright (C) 2006-2009
%   Laurent Claessens
% See the file fdl-1.3.txt for copying conditions.


\begin{corrige}{024}
Une force de \unit{50}{\newton} qui se déplace de \unit{10}{\meter} effectue un travail de 
\[
 W=Fd=\unit{50\cdot 10=500}{\joule}.
\]
 Ce travail fourni par le chercheur d'or  fait avancer le chariot et donc lui donne de l'énergie cinétique. Autrement dit, le chariot a gagné \unit{500}{\joule} d'énergie cinétique. Pour trouver à quelle vitesse cela correspond, on utilise la formule de l'énergie cinétique :
\[ 
  E_c=500=\frac{ mv^2 }{ 2 }.
\]
Ici, $m$ est la masse du chariot et $v$ la vitesse atteinte. Donc
\[ 
  v=\sqrt{ \frac{ 2E_C }{ m } }=\sqrt{ \frac{ 500 }{ 2.5 } }=\unit{14}{\meter\per\second}.
\]
\end{corrige}
