% This is part of Un soupçon de physique, sans être agressif pour autant
% Copyright (C) 2006-2009
%   Laurent Claessens
% See the file fdl-1.3.txt for copying conditions.


\begin{exercice}\label{exo025}

Affin de moins se fatiguer, le chercheur d'or de l'exercice \ref{exo:chariotor} attache le chariot à une corde qu'il fait passer par une poulie et à laquelle il suspend une pierre de \unit{5}{\kilogram} de ferrite rouill\'ee qu'il a trouv\'e sur son chemin dans la mine. Quelle vitess ateint-il après avoir parcouru les même \unit{10}{\meter} ? Est-ce que tu peux savoir sans calculs si il ira plus vite ou moins vite que dans le cas précédent ?

Remarquez que la situationd devient fort analogue à celle de l'exercice \ref{exo020}. Tu peux commencer par recopier le dessin en ne réécrivant que les éléments qu'il faut dans le cas présent.
\corrref{025}
\end{exercice}
