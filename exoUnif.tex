% This is part of Un soupçon de physique, sans être agressif pour autant
% Copyright (C) 2006-2009
%   Laurent Claessens
% See the file fdl-1.3.txt for copying conditions.


%---------------------------------------------------------------------------------------------------------------------------
\subsection{Corrigez vous vous-même}
%---------------------------------------------------------------------------------------------------------------------------

Si vous faites des exercices supplémentaires et que vous voulez des corrections, n'oubliez pas que vous avez un ordinateur à disposition. De nos jours, les ordinateurs sont capables de calculer à peu près tout ce qui se trouve dans ces notes. Le logiciel que je vous propose est \href{http://www.sagemath.org}{Sage}. Pour l'utiliser, il n'est même pas nécessaire de l'installer sur votre ordinateur~: il tourne en ligne, directement dans votre navigateur.

\begin{enumerate}

	\item
		Aller sur \href{http://www.sagenb.org}{http://www.sagenb.org}
	\item
		Créer un compte
	\item
		Créer des feuilles de calcul et amusez-vous !!

\end{enumerate}

Il y a beaucoup de \href{http://lmgtfy.com/?q=sage+documentation}{documentation} sur le \href{http://www.sagemath.org}{site officiel}\footnote{\href{http://www.sagemath.org}{http://www.sagemath.org}}.


Si vous comptez utiliser régulièrement ce logiciel, je vous recommande \emph{chaudement} de \href{http://mirror.switch.ch/mirror/sagemath/index.html}{l'installer} sur votre ordinateur\footnote{Le paquet sagemath.deb fourni par Ubuntu est très bogué, ne l'utilisez pas.}. Ce logiciel étant distribué sous licence GPL, vous ne devez ni payer ni vous procurer de codes.


Affin de vous éviter de trop longues recherches sur internet, j'indiquerai, dans certains exercices, comment il faut faire pour le résoudre avec Sage.  	 Voir les exercices \ref{exoINGE11140028}, \ref{exoINGE11140031}.

%---------------------------------------------------------------------------------------------------------------------------
\subsection{Les exercices}
%---------------------------------------------------------------------------------------------------------------------------

Ce chapitre contient une série d'exercices du type de ce qui est plus ou moins censé être connu à l'entrée de l'université dans diverses sections scientifiques. Ils font l'objet des premières séances d'exercices.

%\Exo{INGE1114-0001}
%\Exo{INGE1114-0002}
%\Exo{INGE1114-0003}
%\Exo{INGE1114-0004}
%\Exo{INGE1114-0005}
\Exo{INGE1114-0006}
%\Exo{INGE1114-0007}
\Exo{INGE1114-0008}
\Exo{INGE1114-0009}

\Exo{INGE1114-0010}
\Exo{INGE1114-0011}
\Exo{INGE1114-0012}
%\Exo{INGE1114-0013}
%\Exo{INGE1114-0014}
%\Exo{INGE1114-0015}
\Exo{INGE1114-0016}
\Exo{INGE1114-0017}
\Exo{INGE1114-0018}
%\Exo{INGE1114-0019}
%\Exo{INGE1114-0020}


%\Exo{INGE11140021}
%\Exo{INGE11140022}
\Exo{INGE11140023}
\Exo{INGE11140024}
\Exo{INGE11140025}
%\Exo{INGE11140026}
\Exo{INGE11140027}

%+++++++++++++++++++++++++++++++++++++++++++++++++++++++++++++++++++++++++++++++++++++++++++++++++++++++++++++++++++++++++++
\section{Limites et continuité}
%+++++++++++++++++++++++++++++++++++++++++++++++++++++++++++++++++++++++++++++++++++++++++++++++++++++++++++++++++++++++++++

\Exo{INGE11140028}
\Exo{INGE11140029}
\Exo{INGE11140030}
\Exo{INGE11140031}
\Exo{INGE11140032}

%+++++++++++++++++++++++++++++++++++++++++++++++++++++++++++++++++++++++++++++++++++++++++++++++++++++++++++++++++++++++++++
\section{Suites numériques}
%+++++++++++++++++++++++++++++++++++++++++++++++++++++++++++++++++++++++++++++++++++++++++++++++++++++++++++++++++++++++++++



\Exo{INGE11140033}
\Exo{INGE11140034}
\Exo{INGE11140035}
\Exo{INGE11140036}
\Exo{INGE11140037}
%\Exo{INGE11140038}
%\Exo{INGE11140039}
%\Exo{INGE11140040}
%\Exo{INGE11140041}
%\Exo{INGE11140042}
%\Exo{INGE11140043}
%\Exo{INGE11140044}
%\Exo{INGE11140045}
%\Exo{INGE11140046}
%\Exo{INGE11140047}
%\Exo{INGE11140048}
%\Exo{INGE11140049}
%\Exo{INGE11140050}
%\Exo{INGE11140051}
%\Exo{INGE11140052}
%\Exo{INGE11140053}
%\Exo{INGE11140054}
%\Exo{INGE11140055}
%\Exo{INGE11140056}
%\Exo{INGE11140057}
%\Exo{INGE11140058}
%\Exo{INGE11140059}
%\Exo{INGE11140060}
%\Exo{INGE11140061}
%\Exo{INGE11140062}
%\Exo{INGE11140063}
%\Exo{INGE11140064}
%\Exo{INGE11140065}
%\Exo{INGE11140066}
%\Exo{INGE11140067}
%\Exo{INGE11140068}
%\Exo{INGE11140069}
%\Exo{INGE11140070}
%\Exo{INGE11140071}
%\Exo{INGE11140072}
%\Exo{INGE11140073}
%\Exo{INGE11140074}
%\Exo{INGE11140075}
%\Exo{INGE11140076}
%\Exo{INGE11140077}
%\Exo{INGE11140078}
%\Exo{INGE11140079}
%\Exo{INGE11140080}
%\Exo{INGE11140081}
%\Exo{INGE11140082}
%\Exo{INGE11140083}
%\Exo{INGE11140084}
%\Exo{INGE11140085}
%\Exo{INGE11140086}
%\Exo{INGE11140087}
%\Exo{INGE11140088}
%\Exo{INGE11140089}
%\Exo{INGE11140090}
%\Exo{INGE11140091}
%\Exo{INGE11140092}
%\Exo{INGE11140093}
%\Exo{INGE11140094}
%\Exo{INGE11140095}
%\Exo{INGE11140096}
%\Exo{INGE11140097}
%\Exo{INGE11140098}
%\Exo{INGE11140099}
%\Exo{INGE11140100}
