% This is part of Un soupçon de physique, sans être agressif pour autant
% Copyright (C) 2006-2009
%   Laurent Claessens
% See the file fdl-1.3.txt for copying conditions.


\begin{corrige}{039}

\begin{enumerate}

\item Sans frottements, il s'agit encore une fois d'égaliser l'énergie cinétique de départ à l'énergie potentielle au sommet de la trajectoire, et encore une fois, la réponse ne va pas dépendre de la masse. On reprend  la formule $h=v^2/2g$ déduite dans la sous-section \ref{SubsecTirVecrtical} :
\[ 
  h=\unit{1.27}{\meter}.
\]
 \item   En comptant les frottements, le bilan d'énergie est plus subtil. Au départ, on a toujours l'énergie cinétique $mv^2/2$. Mais à la fin, bien que la pièce n'ait toujours que son énergie potentielle, il faut tenir compte de l'énergie perdue par le travail de la force de frottement :
\[ 
  E_C=mgh+Fh
\]
où $Fh$ est le travail de la force de frottement $F$ sur la distance $h$ que l'on cherche. Cette fois, la réponse dépend de la masse. Le calcul est d'isoler $h$ dans
\[ 
  \frac{ mv^2 }{ 2 }=(mg+F)h,
\]
donc (en termes d'unités, je te rappelle que \unit{2}{\gram}=\unit{0.002}{\kilo\gram}),	
\[ 
  h=\frac{ mv^2 }{ 2(mg+F) }=\frac{ 0.05 }{ 2\cdot(0.002\cdot 9.81+0.0012) }=\unit{1.2}{\meter}.
\]
\end{enumerate}

\end{corrige}



