% This is part of Un soupçon de physique, sans être agressif pour autant
% Copyright (C) 2006-2009
%   Laurent Claessens
% See the file fdl-1.3.txt for copying conditions.


\begin{corrige}{029}
Juste pour s'amuser, nous allons tout laisser en lettres dans les calculs jusqu'à la réponse finale. Les données sont :

\begin{itemize}
\item $m_f$ la masse de ferrite qui tombe,
\item $m_c$ celle du chariot,
\item $h$ le déplacement du système,
\item $BC$ et $AB$, la hauteur et l'hypothénuse du triangle rectangle.
\end{itemize}

En tombant d'une hauteur $h$, la masse fournit une énergie $m_fgh$ au système. Cette énergie se distribue en énergie cinétique pour les deux masses (parce qu'elles se mettent à bouger) et en énergie potentielle de la boule parce qu'elle monte. 

Commençons par analyser l'énergie cinétique des choses. On sait que les deux objets vont se déplacer à la même vitesse. Notons $v$ cette vitesse. Les énergies cinétiques sont donc
\[ 
  \frac{ m_fv^2 }{ 2 }\quad\text{et}\quad\frac{ m_cv^2 }{ 2 }.
\]
Ce qui est plus compliqué, c'est de voir comment se passe l'énergie potentielle de la boule. Si on note $\alpha$ l'angle de la pente (c'est à dire l'angle au point $A$), on a que la boule monte d'une hauteur $h\sin\alpha$. Mais $\sin\alpha$ peut s'exprimer en termes des données : $\sin\alpha=\frac{ BC }{ AB }$. Le gain d'énergie potentielle que la boule engrange dans l'affaire vaut donc
\[ 
  m_cgh\frac{ BC }{ AB }.
\]
Finalement, ce qu'on trouve comme bilan d'énergie, c'est que
\[ 
  m_fgh=\frac{ m_fv^2 }{ 2 }+\frac{ m_cv^2 }{ 2 }+m_cgh\frac{ BC }{ AB }.
\]
La seule inconnue de cette équation est $v$. En remplaçant tout ce qu'on peut par des nombres, on trouve $v=\unit{8.08}{\meter\per\second}$.
                   


\end{corrige}
