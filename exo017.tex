% This is part of Un soupçon de physique, sans être agressif pour autant
% Copyright (C) 2006-2009
%   Laurent Claessens
% See the file fdl-1.3.txt for copying conditions.


\begin{exercice}\label{exo017}
Une grue soulève en \unit{30}{\second} à vitesse constante un bloc de béton de \unit{50}{\kilogram} d'une hauteur de \unit{20}{\meter}. Quel est le travail effecté par la grue ? Quelle est sa puissance ? Avec la même puissance, combien de temps faudra-t-il à la grue pour soulever une masse de \unit{100}{\kilogram} de la même hauteur ?

\corrref{017}

\end{exercice}
