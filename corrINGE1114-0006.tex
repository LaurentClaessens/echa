% This is part of Un soupçon de physique, sans être agressif pour autant
% Copyright (C) 2006-2009
%   Laurent Claessens
% See the file fdl-1.3.txt for copying conditions.


\begin{corrige}{SerieUn0006}

	Une technique de base dans ces exercices est de savoir que
	\begin{equation}
		| \text{truc} |<\text{machin}
	\end{equation}
	implique la double inégalité
	\begin{equation}					\label{Eqmactrucmach}
		-\text{machin}<\text{truc}<\text{machin},
	\end{equation}
	lorsque $\text{machin}>0$.


	\begin{enumerate}

		\item
			$x\in\mathopen[ -2 , 2 \mathclose]$.
		\item
			Ici, nous appliquons la règle \eqref{Eqmactrucmach} :
			\begin{equation}
				-1<5-\frac{1}{ x }<1,
			\end{equation}
			et nous résolvons les deux inégalités séparément. Les solutions du problème seront l'\emph{intersection} des solutions. 

			La première donne $-5<-1/x$, et donc $6>1/x$. Cette inégalité est certainement vraie pour les $x$ négatifs. Pour les $x>0$, elle est vraie quand $x>1/6$. Les solutions de cette première inégalité sont donc $\mathopen] -\infty , 0 \mathclose[\cup\mathopen] \frac{1}{ 6 } , \infty \mathclose[$.

			La seconde inégalité donne $\frac{1}{ x }>4$. Cette inégalité n'est valable pour aucun négatif. Pour les positifs, les solutions sont $x\in\mathopen] 0 , \frac{1}{ 4 } \mathclose[$.

			La solution de l'exercice est donc donnée par
			\begin{equation}
				\Big(    \mathopen] -\infty , 0 \mathclose[\cup\mathopen] \frac{1}{ 6 } , \infty \mathclose[  \Big)\cap \mathopen] 0, \frac{1}{ 4 } \mathclose[.
			\end{equation}
			C'est à dire $x\in\mathopen] \frac{1}{ 6 } , \frac{1}{ 4 } \mathclose[$.


		\item
			Étant donné que les valeurs absolues sont positives, nous pouvons inverser les fractions (en changeant le sens de l'inégalité) et nous avons
			\begin{equation}
				| x+2 |<| x-1 |.
			\end{equation}
			Nous décomposons cette inégalité en quatre possibilités selon que $x+2$ et $x-1$ soient positifs ou négatifs~:
			\begin{enumerate}
				\item $x+2\geq0$ et $x-1\geq0$, conditions remplie lorsque $x\geq 1$
				\item $x+2\geq0$ et $x-1\leq0$, conditions remplie lorsque $x\in\mathopen[ -2 , 1 \mathclose]$
				\item $x+2\leq0$ et $x-1\geq0$, conditions jamais remplie.
				\item $x+2\leq0$ et $x-1\leq0$, conditions remplie lorsque $x\leq-2$
			\end{enumerate}
			Résolvons l'inéquation dans les quatre cas. Le premier cas donne $| x+2 |=x+2$ et $| x-1 |=x-1$, et donc
			\begin{equation}
				x+2<x-1
			\end{equation}
			Cette inéquation n'est jamais satisfaite et ne fournit donc aucune solutions.

			Le second cas donne $| x+2 |=x+2$ et $| x-1 |=-x+1$, et donc
			\begin{equation}
				x+2<-x+1,
			\end{equation}
			ce qui donne $x<-1/2$. Ces solutions ne sont des vraies solutions de l'exercice que lorsque $x\in\mathopen[ -2 , 1 \mathclose]$. Ce second cas fournit donc les solutions $x\in\mathopen[ -2 , -\frac{1}{ 2 } [$.

			Le troisième cas n'est pas à traiter.

			Le quatrième cas donne
			\begin{equation}
				-x-2<-x+1,
			\end{equation}
			qui est toujours satisfaite. Les solutions fournies par ce cas sont donc $x\in\mathopen] -\infty , -2 \mathclose]$.

			Au final, les solutions de l'exercice sont
			\begin{equation}
				x\in\mathopen] -\infty , -\frac{ 1 }{ 2 } \mathclose[\setminus\{ -2 \}.
			\end{equation}
			Pourquoi retirer $-2$ ? Parce qu'il est dans les conditions d'existence de l'énoncé de départ.


		\item
		\item
			Nous traduisons le problème en
			\begin{equation}
				-1\leq x^2-2\leq 1.
			\end{equation}
			La première inégalité est $x^2-1\geq 0$. (un petit tableau de signe pour ce binôme ?). Les solutions sont $x\in\mathopen] -\infty , -1 \mathclose]\cup\mathopen[ 1 , \infty [$.

			Les solutions de la seconde sont $x\in\mathopen[ -\sqrt{3} , \sqrt{3} \mathclose]$.

			L'intersection des solutions des deux inégalités est
			\begin{equation}
				x\in\mathopen[ -\sqrt{3} , -1 \mathclose]\cup\mathopen[ 1 , \sqrt{3} \mathclose].
			\end{equation}
			

		\item
		\item
			Ici encore, on coupe en deux inégalités
			\begin{equation}
				-1<3-2x<1.
			\end{equation}
			Les solutions sont respectivement $2>x$ et $x>1$ et donc $x\in\mathopen] 1 , 2 \mathclose[$.



	\end{enumerate}
	

\end{corrige}
