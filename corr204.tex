% This is part of Un soupçon de physique, sans être agressif pour autant
% Copyright (C) 2006-2009
%   Laurent Claessens
% See the file fdl-1.3.txt for copying conditions.


\begin{corrige}{204}

Il faut prendre un $\epsilon>0$, et puis trouver un $\delta$ tel que $(| x- a |)\leq\delta$ implique $ | 6x-6a |\leq\epsilon$. Par ce que l'on sait sur les inégalités, 
\[ 
  | 6x-6a |\leq\epsilon
\]
 est équivalent à
\[ 
  | x-a |\leq \frac{ \epsilon }{ 6 }.
\]
Donc prendre $\delta=\epsilon/6$ répond à la question.

\end{corrige}
