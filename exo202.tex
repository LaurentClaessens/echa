% This is part of Un soupçon de physique, sans être agressif pour autant
% Copyright (C) 2006-2009
%   Laurent Claessens
% See the file fdl-1.3.txt for copying conditions.


% Ce fichier est généré automatiquement par le script ran_exo.py
  \begin{exercice}\label{exo202}
 \corrref{202} Les trinômes suivants sont aléatoires; trouves les solutions quand il y en a. \begin{align*}
f_{1}(x)&=-10x^2+x+9&f_{2}(x)&=3x^2-7x-9\\
f_{3}(x)&=5x^2-5x-9&f_{4}(x)&=x^2-3x-3\\
f_{5}(x)&=-2x^2+2&f_{6}(x)&=-x^2+x-4\\
f_{7}(x)&=-8x^2+10x-6&f_{8}(x)&=x^2+x-9\\
f_{9}(x)&=-9x^2+2x+3&f_{10}(x)&=7x^2-4x+9
\end{align*}
\end{exercice}