% Ce fichier est généré automatiquement par le script ran_exo.py
  \begin{exercice}\label{exo201}
 \corrref{201}
 Les trinômes suivants ont deux racines rationnelles -10 et 10. Trouves les.\begin{align*}
f_{1}(x)&=\frac{17}{6}x^2&f_{2}(x)&=-\frac{9}{10}x^2+\frac{3}{7}x-\frac{3}{70}\\
f_{3}(x)&=-\frac{7}{3}x^2-\frac{91}{36}x-\frac{35}{72}&f_{4}(x)&=\frac{3}{2}x^2-\frac{92}{15}x+\frac{26}{5}\\
f_{5}(x)&=\frac{1}{8}x^2-\frac{19}{16}x+\frac{45}{16}&f_{6}(x)&=\frac{9}{8}x^2+\frac{55}{64}x-\frac{7}{64}\\
f_{7}(x)&=-\frac{21}{4}x^2-\frac{5}{2}x+4&f_{8}(x)&=\frac{6}{5}x^2-\frac{33}{10}x-\frac{9}{10}\\
f_{9}(x)&=-\frac{16}{5}x^2-18x-\frac{99}{5}&f_{10}(x)&=-\frac{5}{2}x^2+7x+\frac{55}{2}
\end{align*}
\end{exercice}% This is part of Un soupçon de physique, sans être agressif pour autant
% Copyright (C) 2006-2009
%   Laurent Claessens
% See the file fdl-1.3.txt for copying conditions.


