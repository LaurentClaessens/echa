
\documentclass[a4paper,12pt]{book}
%\documentclass[a4paper,12pt,draft]{book}

\usepackage{latexsym}
\usepackage{amsfonts}
\usepackage{amsmath}
\usepackage{amsthm}
\usepackage{amssymb}

\newtheorem{exercice}{Exercice}			% Les exercices ne se numérotent pas avec les autres, pour que les références soient plus faciles à suivre dans la partie corrigée.


\newenvironment{corrige}[1]{\par\bigskip\noindent \label{corr#1}\par\nopagebreak}{}
\newcommand{\corrref}[1]{\par Corrigé à la page \pageref{corr#1}.}

\begin{document}

\newwrite\test
\openout\test=corr			% Les choses écrites par \write\test arrivent dans le fichier corr.tex, et non dans corr; le .tex est ajouté automatiquement.


\subsection{Exercices}
%---------------------

Prenons maintenant quelque exemples plus mathématiques. Quels sont les cas où une fonction peut ne pas être définie ? Il y en a essentiellement deux pour l'instant (mais d'autres cas viendront) :
\begin{itemize}
\item dénominateur nul,
\item nombre strictement négatif sous la racine.
\end{itemize}
Donc dès que tu vois une fraction, tu exclu du domaine ce qui annule le dénominateur, et dès que tu vois une racine, tu exclu ce qui rends négative l'expression sous la racine.

 Les trinômes suivants ont deux racines entières entre -10 et 10. Trouves les. 
\begin{align*}
f_{1}(x)&=8x^2-96x+280&f_{2}(x)&=-7x^2+49x\\
f_{3}(x)&=x^2-6x-27&f_{4}(x)&=7x^2-21x\\
f_{5}(x)&=5x^2-25x-30&f_{6}(x)&=-9x^2-9x+270\\
f_{7}(x)&=7x^2-105x+350&f_{8}(x)&=7x^2+42x+56\\
f_{9}(x)&=-5x^2-15x+270&f_{10}(x)&=6x^2-18x-324
\end{align*}

\write\test{\string\input{ecmI}}

\begin{align*}
S_{1}&=\left\{-3,-\frac{16}{5}\right\}&S_{2}&=\left\{\frac{1}{9},-\frac{21}{5}\right\}\\
S_{3}&=\left\{1,-\frac{1}{9}\right\}&S_{4}&=\left\{-\frac{31}{10},-3\right\}\\
S_{5}&=\left\{-\frac{2}{3},-\frac{5}{4}\right\}&S_{6}&=\left\{-\frac{11}{5},-\frac{4}{5}\right\}\\
S_{7}&=\left\{-\frac{3}{2},-\frac{5}{4}\right\}&S_{8}&=\left\{-\frac{27}{7},\frac{35}{9}\right\}\\
S_{9}&=\left\{\frac{20}{7},\frac{25}{6}\right\}&S_{10}&=\left\{4,\frac{9}{4}\right\}
\end{align*}

\write\test{\string\input{ecmI}}

\begin{align*}
S_{1}&=\emptyset&S_{2}&=\left\{\frac{4-\sqrt{11}}{5},\frac{4+\sqrt{11}}{5}\right\}\\
S_{3}&=\left\{\frac{-3-\sqrt{29}}{2},\frac{-3+\sqrt{29}}{2}\right\}&S_{4}&=\left\{-1,-1\right\}\\
S_{5}&=\left\{\frac{3}{7},1\right\}&S_{6}&=\emptyset\\
S_{7}&=\emptyset&S_{8}&=\emptyset\\
S_{9}&=\emptyset&S_{10}&=\left\{-\frac{1}{2}+\frac{1}{6}\sqrt{93},-\frac{1}{2}-\frac{1}{6}\sqrt{93}\right\}
\end{align*}

\write\test{\string\input{ecmI}}

\section{Correction}

% This is part of Un soupçon de physique, sans être agressif pour autant
% Copyright (C) 2006-2009
%   Laurent Claessens
% See the file fdl-1.3.txt for copying conditions.



% Ce fichier est généré automatiquement par le script ran_exo.py
  \begin{corrige}{200}

\begin{align*}
S_{1}&=\left\{-3,-5\right\}&S_{2}&=\left\{0,-2\right\}\\
S_{3}&=\left\{10,-6\right\}&S_{4}&=\left\{0,9\right\}\\
S_{5}&=\left\{-3,1\right\}&S_{6}&=\left\{-7,-3\right\}\\
S_{7}&=\left\{3,10\right\}&S_{8}&=\left\{-3,8\right\}\\
S_{9}&=\left\{-6,6\right\}&S_{10}&=\left\{-1,7\right\}
\end{align*}
 Affin d'avoir deux solutions entières, un trinome doit s'écrire sous la forme $a(x-x_1)(x-x_2)$ où $x_1$ et $x_2$ sont les deux racines entières. Le trinôme aura donc toujours la forme $ax^2-a(x_1+x_2)+ax_1x_2$. C'est pour cela que \emph{tous} les trinômes de cet exercice peuvent commencer par simplifier le coefficient de $x^2$.
\end{corrige}
% This is part of Un soupçon de physique, sans être agressif pour autant
% Copyright (C) 2006-2009
%   Laurent Claessens
% See the file fdl-1.3.txt for copying conditions.



% Ce fichier est généré automatiquement par le script ran_exo.py
  \begin{corrige}{201}

\begin{align*}
S_{1}&=\left\{0,0\right\}&S_{2}&=\left\{\frac{1}{7},\frac{1}{3}\right\}\\
S_{3}&=\left\{-\frac{5}{6},-\frac{1}{4}\right\}&S_{4}&=\left\{\frac{26}{9},\frac{6}{5}\right\}\\
S_{5}&=\left\{5,\frac{9}{2}\right\}&S_{6}&=\left\{\frac{1}{9},-\frac{7}{8}\right\}\\
S_{7}&=\left\{-\frac{8}{7},\frac{2}{3}\right\}&S_{8}&=\left\{3,-\frac{1}{4}\right\}\\
S_{9}&=\left\{-\frac{33}{8},-\frac{3}{2}\right\}&S_{10}&=\left\{-\frac{11}{5},5\right\}
\end{align*}
 \end{corrige}
% This is part of Un soupçon de physique, sans être agressif pour autant
% Copyright (C) 2006-2009
%   Laurent Claessens
% See the file fdl-1.3.txt for copying conditions.



% Ce fichier est généré automatiquement par le script ran_exo.py
  \begin{corrige}{202}

\begin{align*}
S_{1}&=\left\{-\frac{9}{10},1\right\}&S_{2}&=\left\{\frac{7+\sqrt{157}}{6},\frac{7-\sqrt{157}}{6}\right\}\\
S_{3}&=\left\{\frac{1}{2}+\frac{1}{10}\sqrt{205},\frac{1}{2}-\frac{1}{10}\sqrt{205}\right\}&S_{4}&=\left\{\frac{3+\sqrt{21}}{2},\frac{3-\sqrt{21}}{2}\right\}\\
S_{5}&=\left\{-1,1\right\}&S_{6}&=\emptyset\\
S_{7}&=\emptyset&S_{8}&=\left\{\frac{-1+\sqrt{37}}{2},\frac{-1-\sqrt{37}}{2}\right\}\\
S_{9}&=\left\{\frac{1-2\sqrt{7}}{9},\frac{1+2\sqrt{7}}{9}\right\}&S_{10}&=\emptyset
\end{align*}
 \end{corrige}
% This is part of Un soupçon de physique, sans être agressif pour autant
% Copyright (C) 2010
%   Laurent Claessens
% See the file fdl-1.3.txt for copying conditions.

\begin{corrige}{213}
Le polynôme $l^2+3l+2$ se factorise en $(l+1)(l+2)$. Lorsque $l$ est entier, il s'agit du produit de deux entiers successifs. L'un des deux est forcément pair.
\end{corrige}

% This is part of Un soupçon de physique, sans être agressif pour autant
% Copyright (C) 2006-2009
%   Laurent Claessens
% See the file fdl-1.3.txt for copying conditions.


\begin{corrige}{203}
\begin{enumerate}
\item $\forall M\in\eR,\exists x_0\in\eR\tq (x>x_0)\Rightarrow f(x)\leq M.$
\item $\forall M\in\eR,\exists x_0\in\eR\tq (x<x_0)\Rightarrow f(x)\geq M.$
\end{enumerate}
\end{corrige}

% This is part of Un soupçon de physique, sans être agressif pour autant
% Copyright (C) 2006-2009
%   Laurent Claessens
% See the file fdl-1.3.txt for copying conditions.


\begin{corrige}{206}
Le minimum d'un sous-ensemble $A$ de $\eR$ est un élément qui est plus petit que tous les éléments de $A$. En formule, on dit que $m$ est un \defe{minorant}{Minorant} de $A$ si 
\[ 
  \forall x\in A,\,x\geq m.
\]
L'\defe{infimum}{Infimum} de $A$ est le plus grand minorant. Si $m$ est l'infimum, alors $\forall x>m$, il existe $y\in A$ avec $y<x$. Et enfin, un \defe{minimum}{Minimum} de $A$ est un infimum qui appartient à $A$.
\end{corrige}

\begin{corrige}{207}


\begin{center}
\begin{tabular}{l|c|c|c|l}
$a$ est un	&	majorant&	supremum	&	maximum	& de $A$\\\hline
		&	oui	&	oui		&	oui	&	$1$ pour $[0,1]$	\\ 
		&	oui	&	oui		&	non	&	$1$ pour $[0,1[$	\\ 
		&	oui	&	non		&	non	&	$2$ pour $[0,1]$	\\ 
		&	oui	&	non		&	oui	&	maximum implique supremum	\\ 
		&	non	&	non		&	oui	&	idem	\\ 
		&	non	&	non		&	non	&	$-1$ pour $[0,1]$	\\ 
		&	non	&	oui		&	non	&	supremum implique majorant	\\ 
		&	non	&	oui		&	oui	&	idem	\\ 
\end{tabular}
\end{center}
Maintenant je t'encourage fortement à recommencer la même chose en remplaçant \emph{majorant}, \emph{supremum} et \emph{maximum} par \emph{minorant}, \emph{infimum} et \emph{minimum}.
\end{corrige}
% This is part of Un soupçon de physique, sans être agressif pour autant
% Copyright (C) 2006-2009
%   Laurent Claessens
% See the file fdl-1.3.txt for copying conditions.



% This is part of Un soupçon de physique, sans être agressif pour autant
% Copyright (C) 2006-2009
%   Laurent Claessens
% See the file fdl-1.3.txt for copying conditions.


\begin{corrige}{210}

Prend par exemple $t=36$. La position du train sera à ce moment 
\[ 
  x(100)=100-36\cdot\sqrt{36}=-116.
\]
Comme $x(t)$ est une fonction continue qui vaut $100$ en $t=0$ et $-116$ en $t=25$, c'est certain qu'elle passe par $13$ à un certain moment.

\end{corrige}

% This is part of Un soupçon de physique, sans être agressif pour autant
% Copyright (C) 2006-2009
%   Laurent Claessens
% See the file fdl-1.3.txt for copying conditions.


\begin{corrige}{211}

Si $P(x)$ est un polynôme de degré impair en $x$, alors $\lim_{x\to-\infty}P(x)=-\infty$, tandis que $\lim_{x\to\infty}P(x)=\infty$. Cela prouve que $P$ est négatif à un moment et positif à un autre moment, et donc le théorème des valeurs intermédiaires impose à $P$ d'être nul entre les deux parce que un polynôme est toujours continu.

\end{corrige}

% This is part of Un soupçon de physique, sans être agressif pour autant
% Copyright (C) 2006-2009
%   Laurent Claessens
% See the file fdl-1.3.txt for copying conditions.


\begin{corrige}{209}

Si $f$ est une telle fontion, il faut prouver que la fonction $g(x)=f(x)-x$ passe par zéro. Que peut valoir $g(0)$ ? Par définition, $g(0)=f(0)\in[0,1]$. Première remarque : si $g(0)=0$, c'est que $f(0)=0$, et donc $0$ est un point fixe de $f$. Supposons donc que $g(0)\neq 0$. Dans ce cas, nous avons $g(0)>0$ (j'insiste sur le strict de cette inégalité).

En faisans le même raisonement (fais-le !), tu trouves que $g(1)<0$. Le théorème de la valeur intermédiaire conclut parce que $g$ est continue (pourquoi ?).

\end{corrige}

\begin{corrige}{208}

Il est naturel de regarder la fonction $d(t)$ qui indique la distance parcourue en un temps $t$. Si $t$ se compte en heures et les distances en kilomètres, cette fonction vérifie évidement
\begin{align*}
d(0)&=0&\text{et}&&d(2)&=10.
\end{align*}
Cela dit, ce qui nous intéresse vraiment, c'est la distance qu'il parcours en une heure, c'est à dire la fonction
\[ 
  f(t)=d(t+1)-d(t).
\]
Cette fonction satisfait 
\begin{align*}
f(0)&=d(1)&\text{et}&&f(1)=10-d(1).
\end{align*}
Ce que tu voudrais prouver, c'est que cette fonction passe par la valeur $5$ entre $t=0$ et $t=1$. Eh bien oui, le nombr $5$ est quelque part entre $d(1)$ et $10-d(1)$, quelle que soit la valeur exacte de $d(1)$. En effet, $5$ est la moyenne arithmétique :
\[ 
  \frac{ d(1)+\big(10-d(1)\big) }{ 2 }=5.
\]

\end{corrige}
% This is part of Un soupçon de physique, sans être agressif pour autant
% Copyright (C) 2006-2009
%   Laurent Claessens
% See the file fdl-1.3.txt for copying conditions.



% This is part of Un soupçon de physique, sans être agressif pour autant
% Copyright (C) 2006-2009
%   Laurent Claessens
% See the file fdl-1.3.txt for copying conditions.


\begin{corrige}{204}

Il faut prendre un $\epsilon>0$, et puis trouver un $\delta$ tel que $(| x- a |)\leq\delta$ implique $ | 6x-6a |\leq\epsilon$. Par ce que l'on sait sur les inégalités, 
\[ 
  | 6x-6a |\leq\epsilon
\]
 est équivalent à
\[ 
  | x-a |\leq \frac{ \epsilon }{ 6 }.
\]
Donc prendre $\delta=\epsilon/6$ répond à la question.

\end{corrige}

% This is part of Un soupçon de physique, sans être agressif pour autant
% Copyright (C) 2006-2009
%   Laurent Claessens
% See the file fdl-1.3.txt for copying conditions.


\begin{corrige}{212}

En vertu du théorème \ref{ThoLimCont}  (limite et continuité), il est suffisant de montrer que pour tout $a\in\eR$,
\[ 
  \lim_{x\to a}\cos(x)=\cos(a).
\]
C'est bien pour ça que ce théorème est génial : il permet de prouver des continuités en calculant des limites. Nous avons par la proposition \ref{PropChmVarLim} :
\begin{equation}
\lim_{x\to a}\cos(x)=\lim_{\epsilon\to 0}\cos(a+\epsilon)=\lim_{\epsilon\to0}\Big( \cos(a)\cos(\epsilon)-\sin(a)\sin(\epsilon) \Big),
\end{equation}
en utilisant la formule \eqref{EqLimLinRes}, nous trouvons à calculer
\[ 
  \cos(a)\Big(\lim_{\epsilon\to 0}\cos(\epsilon)\Big)-\sin(a)\Big( \lim_{\epsilon\to 0}\sin(\epsilon)\Big).
\]
\end{corrige}


%Copyright (c) 2006 Claessens Laurent. Permission is granted to copy, distribute and/or modify this document under the terms of the  GNU Free Documentation License, Version 1.2 or any later version published by the Free Software Foundation; with no Invariant Sections, no Front-Cover Texts, and no Back-Cover Texts. A  copy of the license is included in the section entitled "GNU Free Documentation License".
\begin{corrige}{001}
\begin{enumerate}
\item C'est VRAI.
\item C'est FAUX.
\item C'est FAUX.
\item C'est VRAI.
\end{enumerate}
\end{corrige}
% This is part of Un soupçon de physique, sans être agressif pour autant
% Copyright (C) 2006-2009
%   Laurent Claessens
% See the file fdl-1.3.txt for copying conditions.



\begin{corrige}{003}
\begin{enumerate}
\item Il suffit de prendre un train pour se rendre compte que la trajectoire observée par le voisin est verticale.
\item La seconde question est plus subile. En effet, le GSM continue le mouvement horizontal que le train lui a communiqué, c'est à dire un mouvement uniforme. Mais en même temps, il effectue un mouvement en chutte libre dans la direction verticlale. En équation paramétriques on a donc $x(t)=v_0t$ et $y(t)=\frac{gt^2}{2}$. La conversion en coordonées cartésiennes donne une parabole, c'est à dire une équation $y=f(x)$ du second degré.
\end{enumerate}
\end{corrige}
% This is part of Un soupçon de physique, sans être agressif pour autant
% Copyright (C) 2006-2009
%   Laurent Claessens
% See the file fdl-1.3.txt for copying conditions.



% This is part of Un soupçon de physique, sans être agressif pour autant
% Copyright (C) 2006-2009
%   Laurent Claessens
% See the file fdl-1.3.txt for copying conditions.


% This is part of Un soupçon de physique, sans être agressif pour autant			[1]
% Copyright (C) 2006-2009
%   Laurent Claessens
% See the file fdl-1.3.txt for copying conditions.

\begin{corrige}{MRU0001}

	Nous calculons séparément le temps qu'il faut pour parcourir $\unit{400}{\meter}$ à $\unit{50}{\kilo\meter\per\hour}$ et à $30$. La formule qui donne le temps qu'il faut pour parcourir la distance $\Delta x$ à la vitesse $v$ est donnée par
	\begin{equation}
		t = \frac{ \Delta x }{ v }.
	\end{equation}
	Le calcul à faire est donc
	\begin{equation}
		\frac{ \Delta x }{ v_1 }-\frac{ \Delta x }{ v_2 }=\Delta x\left( \frac{ v_2-v_1 }{ v_1v_2 } \right)
	\end{equation}
	où $\Delta x=\unit{400}{\meter}$, $v_1=\unit{13.9}{\meter\per\second}$ et $v_2=\unit{8.3}{\meter\per\second}$. La réponse est $\unit{19}{\second}$.

	En supposant qu'il ait de la chance avec les feux au bas de la descente du \href{http://www.openstreetmap.org/?lat=50.82012&lon=4.38813&zoom=15&layers=B000FTF}{boulevard de la Plaine}, et en supposant qu'il reste à du $50$ sur toute la longueur (càd sans compter le fait qu'il doive faire un virage à $\unit{90}{\degree}$ pour rentrer sur le boulevard général Jacques), cet automobiliste aurait donc pu gagner vingt secondes.

\end{corrige}

\begin{remark}
	L'auteur de cet exercice précise qu'il est parfaitement conscient qu'il y a une piste cyclable sur toute la longueur de la descente du boulevard de la Plaine. Il y a plusieurs raisons pour lesquelles cette piste ne doit pas être utilisée. L'une d'entre elles est qu'il y a régulièrement des bouteilles en verre cassées. Quant aux autres raisons \ldots je les laisse à la sagacité du lecteur et à son sens de l'observation.

	Dans le sens de la montée, je conseille de prendre le trottoir du côté des banques.
\end{remark}

% [1] D'ac, ici c'est un peu agressif, mais cet abruti méritait d'être insulté et déconstruit.


% This is part of Un soupçon de physique, sans être agressif pour autant
% Copyright (C) 2006-2009
%   Laurent Claessens
% See the file fdl-1.3.txt for copying conditions.


\begin{corrige}{006}

D'abord, comme dans tout problème ne faisant entrer que la gravitation en ligne de compte, la réponse ne dépend pas de la masse. La pierre de \unit{5}{\kilogram} et celle de \unit{10}{\kilogram} prendront un temps identique pour parcourir leur trajet vertical en chute libre.

En chute libre, la distance parcourue en un temps $t$ est donné par la formule \eqref{EqMouvAccatc} dans laquelle il faut remplacer $a$ par $g$. Dans le cas de la pierre en chute libre, nous supposons que la vitesse initiale est nulle, de telle manière à ce qu'il faille résoudre l'équation (pour $t$)
\[ 
  d=\frac{ gt^2 }{ 2 }
\]
avec $d=\unit{52}{\kilo\meter}=\unit{52000}{\meter}$. La solution est donnée par
\[ 
  t=\sqrt{\frac{ 2d }{ g }}=\unit{103}{\second}.
\]
Soit environ une minute et 45 secondes. Inutile de préciser que ce temps bat de très loin les meilleurs sportifs !

\end{corrige}

\begin{corrige}{007}

Le premier cas est très simple parce que l'énergie cinétique que la pierre acquière est l'énergie potentielle perdue durant sa chute de $\Delta h=\unit{15-1.8=13.2}{\meter}$, soit $mg\Delta h=\unit{388.5}{\joule}$.

Dans le second cas, il faut juste ajouter l'énergie cinétique initiale : $mv^2/2$ avec $m=\unit{3}{\kilo\gram}$ et $v=\unit{1}{\meter\per\second}$. L'énergie cinétique initiale est donc \unit{1.5}{\joule}, ce qui fait que l'ennemi prendra $390$ joules sur la tête.

\end{corrige}
% This is part of Un soupçon de physique, sans être agressif pour autant
% Copyright (C) 2006-2009
%   Laurent Claessens
% See the file fdl-1.3.txt for copying conditions.



% This is part of Un soupçon de physique, sans être agressif pour autant
% Copyright (C) 2006-2009
%   Laurent Claessens
% See the file fdl-1.3.txt for copying conditions.


\begin{corrige}{024}
Une force de \unit{50}{\newton} qui se déplace de \unit{10}{\meter} effectue un travail de 
\[
 W=Fd=\unit{50\cdot 10=500}{\joule}.
\]
 Ce travail fourni par le chercheur d'or  fait avancer le chariot et donc lui donne de l'énergie cinétique. Autrement dit, le chariot a gagné \unit{500}{\joule} d'énergie cinétique. Pour trouver à quelle vitesse cela correspond, on utilise la formule de l'énergie cinétique :
\[ 
  E_c=500=\frac{ mv^2 }{ 2 }.
\]
Ici, $m$ est la masse du chariot et $v$ la vitesse atteinte. Donc
\[ 
  v=\sqrt{ \frac{ 2E_C }{ m } }=\sqrt{ \frac{ 500 }{ 2.5 } }=\unit{14}{\meter\per\second}.
\]
\end{corrige}

\begin{corrige}{025}

L'erreur à ne pas commettre est de dire \og j'ai une pierre de \unit{5}{\kilo\gram} et donc une force de \unit{50}{\newton}, ce qui est équivalent à l'exercice précédent\fg. Pourquoi est-ce faux ? Parce que la force de \unit{50}{\newton} s'applique cette fois-ci non seulement au chariot, mais {\bf également à la pierre elle-même}. La force tire donc $\unit{5+2.5=7.5}{\kilo\gram}$ et non seulement les deux kilos et demi du chariot !

Le système accéléra donc moins qu'avant. Cependant, le chercheur d'or se fatigue moins parce que maintenant c'est la gravitation qui travaille.

En ce qui concerne les calculs, la situation est tout de même similaire : une force de \unit{50}{\newton} qui se déplace de \unit{10}{\meter} fourni une énergie de \unit{500}{\joule}. Ces joules servent à faire avancer la pierre plus le chariot dont la masse totale est \unit{7.5}{\kilo\gram}. Donc
\[ 
  v=\sqrt{ \frac{ 2E_C }{ m } }=\sqrt{ \frac{ 500 }{ 7.5 } }=\unit{8.164}{\meter\per\second}.
\]


\end{corrige}

% This is part of Un soupçon de physique, sans être agressif pour autant
% Copyright (C) 2006-2009
%   Laurent Claessens
% See the file fdl-1.3.txt for copying conditions.


\begin{corrige}{029}
Juste pour s'amuser, nous allons tout laisser en lettres dans les calculs jusqu'à la réponse finale. Les données sont :

\begin{itemize}
\item $m_f$ la masse de ferrite qui tombe,
\item $m_c$ celle du chariot,
\item $h$ le déplacement du système,
\item $BC$ et $AB$, la hauteur et l'hypothénuse du triangle rectangle.
\end{itemize}

En tombant d'une hauteur $h$, la masse fournit une énergie $m_fgh$ au système. Cette énergie se distribue en énergie cinétique pour les deux masses (parce qu'elles se mettent à bouger) et en énergie potentielle de la boule parce qu'elle monte. 

Commençons par analyser l'énergie cinétique des choses. On sait que les deux objets vont se déplacer à la même vitesse. Notons $v$ cette vitesse. Les énergies cinétiques sont donc
\[ 
  \frac{ m_fv^2 }{ 2 }\quad\text{et}\quad\frac{ m_cv^2 }{ 2 }.
\]
Ce qui est plus compliqué, c'est de voir comment se passe l'énergie potentielle de la boule. Si on note $\alpha$ l'angle de la pente (c'est à dire l'angle au point $A$), on a que la boule monte d'une hauteur $h\sin\alpha$. Mais $\sin\alpha$ peut s'exprimer en termes des données : $\sin\alpha=\frac{ BC }{ AB }$. Le gain d'énergie potentielle que la boule engrange dans l'affaire vaut donc
\[ 
  m_cgh\frac{ BC }{ AB }.
\]
Finalement, ce qu'on trouve comme bilan d'énergie, c'est que
\[ 
  m_fgh=\frac{ m_fv^2 }{ 2 }+\frac{ m_cv^2 }{ 2 }+m_cgh\frac{ BC }{ AB }.
\]
La seule inconnue de cette équation est $v$. En remplaçant tout ce qu'on peut par des nombres, on trouve $v=\unit{8.08}{\meter\per\second}$.
                   


\end{corrige}

% This is part of Un soupçon de physique, sans être agressif pour autant
% Copyright (C) 2006-2009
%   Laurent Claessens
% See the file fdl-1.3.txt for copying conditions.


%Copyright (c) 2006 Claessens Laurent. Permission is granted to copy, distribute and/or modify this document under the terms of the  GNU Free Documentation License, Version 1.2 or any later version published by the Free Software Foundation; with no Invariant Sections, no Front-Cover Texts, and no Back-Cover Texts. A  copy of the license is included in the section entitled "GNU Free Documentation License".
\begin{corrige}{011}

Le second dynamomètre indique également \unit{2}{\newton} parce quele mur agit de la même façon sur le premier que la masse sur le second. En effet, la masse qui pend tire sur le dynamomètre, et pour que celui-ci reste en équilibre, il faut qu'une force égale en grandeur et opposée tire le dynamomètre vers la gauche. Peu importe que cette force soit une force de réaction du mur ou une force de pesenteur d'une autre masse, le résultat est le même : le dynamomètre est tiré dans les deux sens et reste en équilibre.
\end{corrige}

% This is part of Un soupçon de physique, sans être agressif pour autant
% Copyright (C) 2006-2009
%   Laurent Claessens
% See the file fdl-1.3.txt for copying conditions.


%Copyright (c) 2006 Claessens Laurent. Permission is granted to copy, distribute and/or modify this document under the terms of the  GNU Free Documentation License, Version 1.2 or any later version published by the Free Software Foundation; with no Invariant Sections, no Front-Cover Texts, and no Back-Cover Texts. A  copy of the license is included in the section entitled "GNU Free Documentation License".


\begin{figure}[ht]
\centering

%   Première sous-figure

\subfigure[Première addition]{%
\begin{pspicture}(-0.5,-2)(4,2)
   \psset{PointSymbol=none, PointName=none}
   \prefigzerotreize

\pstMarqueForce{O}{A}{0.5;30}{$\fF_1$}
\pstMarqueForce{O}{B}{0.5;0.3}{$\fF_2$}
\pstMarqueForce{O}{C}{0.5;0.3}{$\fF_3$}
   \pstTranslation{O}{B}{A}[Fq]

{\psset{linecolor=blue} \pstMarqueForce{O}{Fq}{0.3;0}{$\overrightarrow{F}_4$} }
\end{pspicture}
}						% Fin de la première sous-figure
%  Seconde sous-figure 
\subfigure[Seconde addition]{%
\begin{pspicture}(-0.5,-2)(4,2)
   \psset{PointSymbol=none, PointName=none}
   \prefigzerotreize
   \pstTranslation{O}{B}{A}[Fq]
\pstMarqueForce{O}{C}{0.5;0.3}{$\fF_3$}
\pstMarqueForce{O}{Fq}{0.5;0.3}{$\fF_4$}
   \pstTranslation{O}{C}{Fq}[Fc]
{\psset{linecolor=blue} \pstMarqueForce{O}{Fc}{0.3;0}{$\overrightarrow{F}_5$} }
\end{pspicture}
}						% La fin de la seconde sous-figure
\caption{Exercice de composition de forces}\label{fig:exo:comp:corr}
\end{figure}


\begin{corrige}{013}
Regardez la figure \ref{fig:exo:comp:corr}. D'abord on construit $\overrightarrow{F}_4=\overrightarrow{F}_1+\overrightarrow{F}_2$ en ne regardant pas $\overrightarrow{F}_3$. Pour cela, on met juste $\fF_1$ et $\fF_2$ bout à bout. Peu importe si on met $\fF_1$ au bout de $\fF_2$ ou le contraire : le résultat sera le même.


Ensuite on oublie $\fF_1$ et $\fF_2$ et on additionne $\fF_4$ avec $\fF_3$.

Montrons à présent comment décomposer $\fF_1$ en $\fF_2$ et $\fF_3$ sur la figure \ref{fig_decom013}. La première chose à faire est de prolonger les axes \emph{dans les deux sens} ainsi que dessiner des parallèles à ces axes passant par le bout de la force à décomposer. Ensuite, les points d'intersection donnent la décomposition.

Un autre exemple de décomposition de forces est donné à la figure \ref{fig:Force_decomp} se trouvant à la page \pageref{fig:Force_decomp}.

\begin{figure}[ht]
\centering
   \psset{PointSymbol=none, PointName=none}
\subfigure[Prolonger les axes et trouver les points d'intersection]{%
\begin{pspicture}(-1,-2.3)(4,2.5)
   \prefigzerotreize

\pstMarqueForce{O}{A}{0.5;30}{$\fF_1$}
\pstMarqueForce{O}{B}{0.5;-50}{$\fF_2$}
\pstMarqueForce{O}{C}{0.5;0}{$\fF_3$}

  \pstDecompForce{O}{A}{O}{B}{O}{C}{dFu}{dFd}
  \pstSymO{O}{dFu,dFd}[sdFu,sdFd]		% Comme ça, ça va prolonger dans les deux sens
   {%
\psset{linecolor=red,linestyle=dotted}
\pstLineAB[nodesepA=-0.5,nodesepB=1.2]{dFu}{sdFu}
\pstLineAB[nodesepA=-0.5,nodesepB=-0.4]{dFd}{sdFd}
   }

   {%
\psset{linecolor=green,linestyle=dotted}
\pstLineAB[nodesepA=-1.2,nodesepB=-1.2]{dFu}{A}
\pstLineAB[nodesepA=-1.2,nodesepB=-1.2]{dFd}{A}
   }
\end{pspicture}
                              }			% Fin de la première sous-fugure
\subfigure[Décomposer la force]{%
\begin{pspicture}(-0.5,-2)(3.5,2)
   \prefigzerotreize


\pstMarqueForce{O}{A}{0.5;30}{$\fF_1$}
\pstMarqueForce{O}{B}{0.5;-50}{$\fF_2$}
\pstMarqueForce{O}{C}{0.5;0}{$\fF_3$}
  \pstDecompForce{O}{A}{O}{B}{O}{C}{dFu}{dFd}

   {%
\psset{linecolor=blue}
\pstMarqueForce{O}{dFu}{0.5;0.3}{}
\pstMarqueForce{O}{dFd}{0.5;0.3}{} 
   }
\end{pspicture}
                       }		% Fin de la deuxième sous-figure
\caption{Décomposition de forces}\label{fig_decom013}
\end{figure}

\end{corrige}

% This is part of Un soupçon de physique, sans être agressif pour autant
% Copyright (C) 2006-2009
%   Laurent Claessens
% See the file fdl-1.3.txt for copying conditions.


\begin{corrige}{014}
Le poids de Titania n'a aucun sens; on ne peut rien en dire. Ou alors il faut dire le poids par rapport à quoi; le Soleil ? Uranus ? En effet, le poids est la force de gravitation qui s'exerce sur un objet. Ça a donc un sens de parler du poids d'un objet sur Terre parce qu'on sous-entend qu'on regarde la force de gravitation que la Terre exerce sur l'objet. Mais on ne peut pas vraiment parler du poids d'une planète. Par contre, on peut parler de sa masse.
\end{corrige}

% This is part of Un soupçon de physique, sans être agressif pour autant
% Copyright (C) 2006-2009
%   Laurent Claessens
% See the file fdl-1.3.txt for copying conditions.


\begin{corrige}{016}

La caisse se déplace à vitesse constante. Pourquoi ? Parce qu'il y a \emph{deux} forces qui s'y appliquent : la première est celle de frottement, tandis que la seconde la force qui tire dont on parle. Il est dit dans l'énoncé que ces deux forces se compensent exactement. Donc il n'y a globalement aucune force qui s'appliquent à la caisse, ce qui fait qu'il avance à vitesse consante.

Remarque qu'au début, pour faire démarrer la caisse, il a fallu pendant un certain temps appliquer une force suppérieure à celle de frottement.

Le travail d'une force $\fF$ qui se déplace d'une distance $d$ est par définition $W=\| \fF \|f$ quand la force est parallèle au déplacement, ce qui est le cas ici parce qu'on parle d'un plan horizontal et d'une force horizontale. On a donc
\[ 
  W=\unit{10}{\newton}\cdot\unit{2.3}{\meter}=\unit{23}{\joule}
\]



\end{corrige}

% This is part of Un soupçon de physique, sans être agressif pour autant
% Copyright (C) 2006-2009
%   Laurent Claessens
% See the file fdl-1.3.txt for copying conditions.


\begin{corrige}{017}
Le travail effectué par la grue est l'énergie qu'il faut donner au béton pour lui permettre d'avancer de \unit{20}{\meter} dans le champ de gravitation. C'est à dire $W=P\cdot h$ où $P$ est le poids du béton; donc
\[ 
  W=mgh=50\cdot 9.81\cdot 20=\unit{9810}{\joule}.
\]
Si tu n'es pas sûre du signe, il y a au moins deux manières de savoir que c'est bien positif. La première est de remarquer que le déplacement se fait dans le même sens que la force (c'est pour ça qu'on a noté $50\cdot 9.81\cdot 20$ et non des choses du genre $(-50)\cdot 9.81\cdot 20$ ou $50\cdot9.81\cdot (-20)$).

La seconde est de remarquer que la grue transforme de l'énergie potentielle (par exemple ses batteries ou son pétrole) en de l'énergie cinétique (de la charge qui monte). Cependant, la force de gravitation se charge de transformer immédiatement cette énergie cinétique en énergie potentielle (le travail de la gravitation est négatif, ce qui est cohérent avec le fait que son sens est le contraire de celui du déplacement), donc on ne \og voit\fg{} pas bien ce passage par du cinétique.

La puissance d'une machine est la quantité de joules que la grue dévesre par seconde dans l'objet sur lequel elle travaille. Ici, c'est donc le nombre de joules par secondes que reçoit le bloc de béton. En l'occurence, le béton reçoit $9810$ joules en $30$ secondes, soit $\unit{\frac{9810 }{ 30 }}{\joule\per\second}=\unit{327}{\watt}$.

Il s'agit maintenant de soulever \unit{100}{\kilogram}, c'est à dire deux fois plus de masse. Étant donné que le travail de la gravitation est proportionel à la masse, il faudra un travail deux fois plus grand. Autrement dit, il faut deux fois plus de joules à un objet de \unit{100}{\kilogram} pour monter de la même hauteur qu'un objet de \unit{50}{\kilogram}. Comme la puissance est la même, il reçoit autant de joules par seconde que le premier objet. Pour en avoir deux fois plus il faut donc deux fois plus de temps, soit une minute.

\end{corrige}

% This is part of Un soupçon de physique, sans être agressif pour autant
% Copyright (C) 2006-2009
%   Laurent Claessens
% See the file fdl-1.3.txt for copying conditions.


%Copyright (c) 2006 Claessens Laurent. Permission is granted to copy, distribute and/or modify this document under the terms of the  GNU Free Documentation License, Version 1.2 or any later version published by the Free Software Foundation; with no Invariant Sections, no Front-Cover Texts, and no Back-Cover Texts. A  copy of the license is included in the section entitled "GNU Free Documentation License".
\begin{corrige}{018}
Le mobile effectue à la fois le déplacement $AB$ et le déplacement $AC$. Mais on sait que dans le travail, seule compte la composante du déplacement parallèle à la force. Donc le problème serait exactement le même si on faisait $\| BC \|=0$, c'est à dire si le mobile se déplaçait sur un plan horizontal. Dans ce cas, le déplacement est parallèle à la force et donc le travail vaut $F\| AC \|$.


Une autre façon de répondre est de considérer la formule du travail dans le cas non parallèle, c'est à dire
  $W=F\cdot d\cdot\cos(\fF,\overrightarrow{d})$ où $\cos(\fF,\overrightarrow{d})$ désigne le cosinus de l'angle entre $\fF$ et le déplacement $\overrightarrow{d}$, c'est à dire l'angle $\widehat{BAC}$. Donc
\[ 
  W=F\cdot | AB |\cdot \cos(\widehat{ABC}),
\]
mais $| AB |\cos(\widehat{BAC})=| AC |$, et donc
\[ 
  W=F\cdot | AC |.
\]

La façon la plus efficace de tirer cet objet est de le tirer parallèlement au déplacement (de façon à ce que le cosinus soit $1$), c'est à dire avec une force parallèle au plan incliné. En effet, la composante perpendiculaire est annulée par une réaction du sol.

\begin{figure}[h]
\centering
\begin{pspicture}(-0.5,-1.5)(5.5,3)

   \psset{PointSymbol=none, PointName=none}
   \prefigzerounhuit
   \psline(A)(C)
   \psline(C)(B)
   \psline(A)(B)
   \pstCircleAB[fillstyle=crosshatch,fillcolor=black]{Oa}{Ob}
   \pstMarqueForce{Cc}{bF}{0.3;270}{$\fF$}
   \pstDecompForce{Cc}{bF}{A}{B}{Oa}{Ob}{Fu}{Fd}
{%
\psset{linecolor=blue}
   \pstMarqueForce{Cc}{Fu}{0.3;0}{$F_{\parallel}$}
   \pstMarqueForce{Cc}{Fd}{0.3;0}{$F_{\perp}$}
}
\end{pspicture}
\caption{Décomposition de la force de la correction \ref{corr018}.}
\end{figure}
 

\end{corrige}

% This is part of Un soupçon de physique, sans être agressif pour autant
% Copyright (C) 2006-2009
%   Laurent Claessens
% See the file fdl-1.3.txt for copying conditions.


%Copyright (c) 2006 Claessens Laurent. Permission is granted to copy, distribute and/or modify this document under the terms of the  GNU Free Documentation License, Version 1.2 or any later version published by the Free Software Foundation; with no Invariant Sections, no Front-Cover Texts, and no Back-Cover Texts. A  copy of the license is included in the section entitled "GNU Free Documentation License".

\begin{figure}[h]
\centering
\begin{pspicture}(-0.5,-0.5)(4,2.5)
  \psset{PointSymbol=none, PointName=none}
\prefigzerounneuf
   \psline(A)(C)
   \psline(C)(B)
   \psline(A)(B)
  
   \pstMarqueForce{Cc}{bG}{0.4;45}{$\fG$}

\pstInterLL{Cc}{bG}{A}{B}{perc}
\pstInterLL{Cc}{bG}{A}{C}{percd}

\pstTransHom{perc}{B}{Cc}{1}{cAB}
\pstTransHom{perc}{B}{Cc}{-1}{cBA}

   \pstMarqueForce{Cc}{cAB}{0.3;90}{$A\to B$}
   \pstMarqueForce{Cc}{cBA}{0.5;90}{$B\to A$}

\pstMarkAngle{C}{A}{B}{$\alpha$}
\pstMarkAngle[linecolor=blue]{A}{perc}{percd}{$\beta$}
\pstMarkAngle[linecolor=red]{percd}{perc}{B}{$\gamma$}

\end{pspicture}
\caption{Angles entre la gravitation et les déplacements}\label{fig_corrchar}
\end{figure}


\begin{corrige}{019}


Tu dois aller voir la figure \ref{fig:Force_decomp} pour voir comment il faut traiter avec les problèmes de décomposition de forces sur un plan incliné.

La force $\fR$ est perpendiculaire au déplacement (que ce soit pour $A$ vers $B$ ou $B$ vers $A$), donc son travail est toujours nul.

La force $\fF$ est toujours parallèle au déplacement. Durant le déplacement de $A$ vers $B$, elle est dans le même sens et donc $W_F^{A\to B}=F| AB |$; durant le déplacement de $B$ vers $A$, elle est dans le sens inverse et donc $W_F^{B\to A}=-F| AB |$.

Sur la figure \ref{fig_corrchar}, l'angle $\beta$ entre $\fG$ et le déplacement $B\to A$ vaut $90-\alpha$ tandis que l'angle $\gamma$ entre $\fG$ et le déplacement $A\to B$ vaut $180-(90-\alpha)=90+\alpha$.
Durant le déplacement $A\to B$, le travail de $\fG$ vaut donc $G\cdot| AB |\cos\gamma=-G\cdot| AB |\sin(\alpha)=-G| BC |$ parce que $\cos(90+\alpha)=-\sin(\alpha)$.
Durant le déplacement $B\to A$, le travail de $\fG$ vaut 
\[
G\cdot| AB |\cos(90-\alpha)=G\cdot| AB |\sin(\alpha)=G\cdot| BC |.
\]

Lorsque $\| \fF \|=\| \fG \|$, le chariot, l'objet monte. En effet, seule le composante parallèle de $\fG$ n'est utile pour faire descendre le chariot, mais chaque composante d'une force est toujours plus petite que la force elle-mêle, c'est à dire que $\| \fG_{\parallel} \|<\| \fG \|$.

\end{corrige}

% This is part of Un soupçon de physique, sans être agressif pour autant
% Copyright (C) 2006-2009
%   Laurent Claessens
% See the file fdl-1.3.txt for copying conditions.


%Copyright (c) 2006 Claessens Laurent. Permission is granted to copy, distribute and/or modify this document under the terms of the  GNU Free Documentation License, Version 1.2 or any later version published by the Free Software Foundation; with no Invariant Sections, no Front-Cover Texts, and no Back-Cover Texts. A  copy of the license is included in the section entitled "GNU Free Documentation License".
\begin{corrige}{020}
Les forces $F_2$ et $\fG_1$ sont perpendiculaires au déplacement, donc leur travail est nul. Les forces $\fF_4$ et $\fG_G$ sont parallèles au déplacement, dans le même sens; donc leur travail vaut juste le produit de leurs normes par la longueur du déplacement : $W_{\fF_4}=\| \fF_4 \|\cdot d$ et $W_{\fG_2}=\| \fG_2 \|\cdot d$. La force $\fF_1$ va dans le sens inverse de son déplacement, ce qui fait venir un signe moins (penser au cosinus de $180$) : $W_{\fF_1}=-\| \fF_1 \|\cdot d$.

Si le tout se déplace d'une distance $d$ vers la gauche, il faut juste changer tous les signes.
\end{corrige}

\begin{corrige}{023}
La masse  de gauche va monter tandis que celles de droite vont descendre. Donc le travail de la force de pesenteur sera négatif sur la masse de gauche et positif à droite. Le travail est donné par $W=G\Delta h=mg\Delta h$. En l'occurence :
\[ 
 W_{\text{gauche}}=-mgh,
\]
et de la même manière : $W_{\text{droite}}=mgh$, avec un signe positif cette fois. Ce travail est effectué sur \emph{chacune} des deux masses. 

Nous pouvons aussi considérer que les deux masses de droite n'en font qu'une seule de $2m$, et alors dire qu'il n'y a qu'un seul travail 
\[
W_{\text{droite}}=2mgh.
\]
\end{corrige}

% This is part of Un soupçon de physique, sans être agressif pour autant
% Copyright (C) 2006-2009
%   Laurent Claessens
% See the file fdl-1.3.txt for copying conditions.


\begin{corrige}{026}
Le frein doit appliquer une force parallèle à la pente, dirigée vers le haut, sinon la viture la dévalerait.  Une telle force est donc indispensable pour le faire tenir en équilibre. Regardons la figure \ref{fig:chariotdecomp}. Le fait que la pente soit de $5$ \%  signifie que $AB=\unit{100}{\meter}$ et $BC=\unit{4}{\meter}$, c'est à dire que $\sin(\widehat{BAC})=5/100$. 

Disons que $g=\unit{10}{\meter\per\second\squared}$ pour avoir des nombres ronds.

Nous savons que la partie de $\fG$ perpendiculaire au plan incliné sera compensée par la réaction du plan. On ne s'y intéresse donc pas. En ce qui concerne la partie parallèle au plan incliné de la force de gravitation, la formule \eqref{eq_Stevinpapa} nous enseigne que
\[ 
  G_{\parallel}=G\sin(\widehat{BAC})=\unit{5000\cdot 5/100=250}{\newton},
\]
et est dirigée vers le bas. La force que l'on cherche, vaut la même chose en norme, mais est dirigée vers le haut.
 
 
\end{corrige}

\begin{corrige}{027}

La pierre subit deux forces : l'une est celle de la gravitation qui la tire vers le bas et le frottement de l'eau qui pousse vers le haut. Comme elle tombe à vitesse constante, c'est que ces deux forces sont égales en normes et opposées en sens. Quand on pèse \unit{5}{\kilo\gram}, la force de pesenteur qui s'applique est
\[ 
  P=mg= \unit{5\cdot 9.81=49}{\newton}.
\]
Les forces de frottements doivent donc valoir également \unit{49}{\newton}.
\end{corrige}
% This is part of Un soupçon de physique, sans être agressif pour autant
% Copyright (C) 2006-2009
%   Laurent Claessens
% See the file fdl-1.3.txt for copying conditions.



% This is part of Un soupçon de physique, sans être agressif pour autant
% Copyright (C) 2006-2009
%   Laurent Claessens
% See the file fdl-1.3.txt for copying conditions.


\begin{figure}[h]
\centering
\begin{pspicture}(-0.7,-1)(4.7,2.5)
  \psset{PointSymbol=none, PointName=none}
\prefigplincl						% La position des points est contenue dans cette macro
							%  qui se trouve en principe juste au-dessus.
   \psline(A)(C)
   \psline(C)(B)
   \psline(A)(B)
  
   \pstCircleAB{Oa}{Ob}
   \pstCircleAB{Pa}{Pb}

%   \psline(Cg)(Cgh)
%   \psline(Cgh)(Cdh)
%   \psline(Cd)(Cdh)
%   \psline(Cg)(Cd)


\pstDecompForce{Cc}{bG}{Cc}{bR}{A}{B}{Gpe}{Gpa}
\pstSymO{Cc}{Gpa}[Fr]

   \pstMarqueForce{Cc}{bG}{0.3;0}{$\fG$}

{\psset{linecolor=red}
   \pstMarqueForce{Cc}{Gpa}{0.5;110}{$\fG_{\parallel}$}
}

{\psset{linecolor=green}
   \pstMarqueForce{Cc}{Fr}{0.3;90}{$\fF$}
}
\end{pspicture}
\caption{Certaines forces qui s'appliquent au cycliste.} \label{fig_cylccorr}
\end{figure}

\begin{corrige}{028}
Le cycliste et le vélo sont dans la même situation qu'une voiture\footnote{Sauf que le cycliste n'aura pas de comptes à rendre aux générations futures pour ses émissions de $CO_2$.} dont le moteur fournit la force $F$ cherchée pour gravir à vitesse constante le plan incliné $AB$, voir figure \ref{fig_cylccorr}. 

Comme le cycliste avec son vélo font $\unit{80+3=83}{\kilo\gram}$, on a que
\[
 \| \fG \|=\unit{9.81\cdot 83=814}{\newton}.
\]
 D'autre part, $\sin(\widehat{BAC})=7/100$. Maintenant, il faut se souvenir de ce qu'on avait dit à propos de la décomposition de la force de pesenteur $\fG$ sur un plan incliné :
\[ 
  \| \fG_{\parallel}\|=\| \fG \|\cdot\sin(\widehat{BAC}) =\unit{57}{\newton}. 
\]
C'est cette force-là qui doit être compensée par les muscles du courageux cycliste. Il devra donc fournir une force de \unit{57}{\newton}.


Il existe une {\bf autre méthode} pour résoudre cet exercice. Le principe est que le cycliste produit une force qui travaille. Ce travai sert à d'une part à l'accélérer (gain d'énergie cinétique) et d'autre part à le monter le long du plan incliné (gain d'énergie potentielle). Ici, on a supposé que la vitesse était constante, c'est à dire que tout le travail passe à faire gagner de l'énergie potentielle.

Étudions ce qui se passe lorsque le cycliste parcours le chemin $AB$. Si $AB=\unit{100}{\meter}$, alors $BC=\unit{7}{\meter}$ et il monte de $7\meter$, c'est à dire qu'il gagne $E_p=mgh=\unit{83\cdot 9.81\cdot 7=5700}{\joule}$. 

Ce faisant, le sportif à deux roues aura fait un traval $W=F\cdot d=100F$. Quand on égalise ce travail au gain d'énergie potentielle, on trouve $100F=5700$, ce qui permet de conclure $F=\unit{57}{\newton}$.

En ce qui concerne la puissance, il suffit d'utilier la formule sympa \eqref{eq_puissFv}. Nous avons une force de \unit{57}{\newton} qui se déplace à la vitesse constante de \unit{3}{\meter\per\second}. La puissance est donc de $P=\unit{57\cdot 3=171}{\watt}$.


\end{corrige}

% This is part of Un soupçon de physique, sans être agressif pour autant
% Copyright (C) 2006-2009
%   Laurent Claessens
% See the file fdl-1.3.txt for copying conditions.


\begin{corrige}{030}

Premier réflexe : convertir la vitesse en des unités du système international : \unit{180}{\kilo\meter\per\hour}=\unit{50}{ \meter\per\second}.

Étant donné que l'on connaît la vitesse et la puissance du moteur, la \og petite formule sympa\fg{} \eqref{eq_puissFv} permet de trouver la force du moteur :
\[ 
  F=\frac{ P }{ v }=\frac{ 110000 }{ 50 }=\unit{2200}{\newton}.
\]
Étant donné qu'on parle d'une vitesse constant, c'est que les forces de frottements compensent exactement la force du moteur. À partir de maintenant, l'idée d'égaliser les forces de frottements à la force d'un moteur quand on dit que la vitesse est constante devrait te venir automatiquement en tête. Donc les forces de frottements s'élèvent également à \unit{2200}{\newton}.

\end{corrige}

% This is part of Un soupçon de physique, sans être agressif pour autant
% Copyright (C) 2006-2009
%   Laurent Claessens
% See the file fdl-1.3.txt for copying conditions.


\begin{corrige}{038}
\begin{enumerate}

\item Au départ l'énergie du cahier est potentielle, et à la fin elle est cinétique. L'énergie potentielle de départ est celle d'une masse de \unit{0.2}{\kilo\gram} à une hauteur de \unit{5}{\meter}, c'est à dire $E_P=mgh=\unit{9.81}{\joule}$. 

Pour trouver la vitesse $v$ à laquelle la pierre va toucher le sol, il faut égaliser cette énergie à $mv^2/2$ et en déduire $v$ :
\[ 
  v=\sqrt{ \frac{ 2E_P }{ m } }=\sqrt{\frac{ 2mgh }{ m }}=\sqrt{2gh}=\unit{9.904}{\meter\per\second}.
\]
Une fois de plus, {\bf la réponse ne dépend  pas de la masse !}


\item  Lorsque l'on tient compte des frottement, l'énergie potentielle de départ est convertie en énergie cinétique plus de l'énergie perdue. Cette énergie perdue vaut le travail des forces de frottement : $Fh=0.2\cdot 40=\unit{8}{\joule}$. Donc,
\[ 
  \frac{ mv^2 }{ 2 }-Fh=mgh,
\]
et donc
\[ 
  v=\sqrt{  \frac{ 2(mgh+Fh) }{ m }  }=\unit{7.94}{ \meter\per\second}.
\]

\end{enumerate}


\end{corrige}

% This is part of Un soupçon de physique, sans être agressif pour autant
% Copyright (C) 2006-2009
%   Laurent Claessens
% See the file fdl-1.3.txt for copying conditions.


\begin{corrige}{039}

\begin{enumerate}

\item Sans frottements, il s'agit encore une fois d'égaliser l'énergie cinétique de départ à l'énergie potentielle au sommet de la trajectoire, et encore une fois, la réponse ne va pas dépendre de la masse. On reprend  la formule $h=v^2/2g$ déduite dans la sous-section \ref{SubsecTirVecrtical} :
\[ 
  h=\unit{1.27}{\meter}.
\]
 \item   En comptant les frottements, le bilan d'énergie est plus subtil. Au départ, on a toujours l'énergie cinétique $mv^2/2$. Mais à la fin, bien que la pièce n'ait toujours que son énergie potentielle, il faut tenir compte de l'énergie perdue par le travail de la force de frottement :
\[ 
  E_C=mgh+Fh
\]
où $Fh$ est le travail de la force de frottement $F$ sur la distance $h$ que l'on cherche. Cette fois, la réponse dépend de la masse. Le calcul est d'isoler $h$ dans
\[ 
  \frac{ mv^2 }{ 2 }=(mg+F)h,
\]
donc (en termes d'unités, je te rappelle que \unit{2}{\gram}=\unit{0.002}{\kilo\gram}),	
\[ 
  h=\frac{ mv^2 }{ 2(mg+F) }=\frac{ 0.05 }{ 2\cdot(0.002\cdot 9.81+0.0012) }=\unit{1.2}{\meter}.
\]
\end{enumerate}

\end{corrige}







\end{document}
% This is part of Un soupçon de physique, sans être agressif pour autant
% Copyright (C) 2006-2009
%   Laurent Claessens
% See the file fdl-1.3.txt for copying conditions.


