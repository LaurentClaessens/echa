% This is part of Un soupçon de physique, sans être agressif pour autant
% Copyright (C) 2006-2009
%   Laurent Claessens
% See the file fdl-1.3.txt for copying conditions.


\section{Travail}
%++++++++++++++++


\subsection{De la notion de travail}
%--------------------------------------

Un système isolé conserve son énergie. Mais, \emph{a contrario}, un système non isolé a une énergie qui a de fortes chances de varier. C'est à dire que quand une force s'applique à un objet, l'objet gagne de l'énergie. Il n'est pas compliqué de comprendre pourquoi : une force provoque une accélération, et donc une variation de la vitesse. Or qui dit vitesse dit énergie cinétique. L'énergie qu'un objet gagne sous l'action d'une force s'appelle le \defe{travail}{} de la force. Comment la calculer ? 

Considérons une force constante $F$ qui tire horizontalement un objet de masse $m$ initialement au repos. L'accélération que l'objet acquiert est $a=F/m$, et sa vitesse après une durée $t$ est $v=at$. Selon la formule de l'énergie cinétique,
\begin{equation}
\begin{aligned}
   E_c(t)&=\frac{ mv(t)^2 }{ 2 }\\
	&=\frac{ ma^2t^2 }{ 2 }&\text{$v=at$}\\
	&=ma\frac{ at^2 }{ 2 }&\text{réarangement}\\
	&=Fd&\text{$F=ma$ et $d=\frac{ at^2 }{ 2 }$}.
\end{aligned}
\end{equation}
Cette énergie n'est pas conservée du fait de l'action d'une force extérieure sur l'objet.

Lorsqu'une force déplace son point d'application, on dit qu'elle \defe{travaille}{Travail}. Ce travail est l'énergie que la force fait gagner à l'objet sur lequel elle s'applique. Nous venons de traiter le cas où la force était parallèle au déplacement. Tu te souviens que la force et le déplacement sont des \emph{vecteurs}; que faire lorsque ces deux vecteurs ne sont pas parallèles ?

Le travail d'une force est le produit scalaire entre la force et le déplacement. En formules, si on a une force $fF$ qui se déplace d'un vecteur $\overrightarrow{\Delta x}$, son travail noté $W$ vaut
\[
W=   \overrightarrow{F}\cdot\overrightarrow{\Delta x}.
\]
où le point signifie un produit scalaire. Au cas où tu ne serais pas familier avec les produits scalaires entre vecteurs, la figure \ref{fig:travail} montre comment ça marche.

\begin{figure}[ht]
\begin{center}
\begin{pspicture}(-0.5,-1)(6,2)
%  \psaxes[dotsep=1pt]{->}(0,0)(-2,-2)(2,2)
%  \psframe(-0.5,-1)(6,2)
  \pstGeonode[PosAngle=-90](0,0){A}
  \pstGeonode[PosAngle=-90](3,0){B}
   \pstGeonode[PointSymbol=none, PointName=none](5,0){C}
  \pstGeonode[PtNameMath=false, PointName=none](2,1){f}
   \rput(f){\rput(0.3,0.3){$\fF$}}
  \psline{->}(A)(B)
   \psline[linestyle=dotted]{-}(B)(C)
  \psline{->}(A)(f)
  \pstMarkAngle[arrows=->]{B}{A}{f}{$\alpha$}
  \pstTranslation[PointName=none]{A}{B}{f}[fp]
  \psline{->}(B)(fp)
   \rput(fp){\rput(0.3,0.3){$\fF$}}
   \pstMarkAngle[arrows=->]{C}{B}{fp}{$\alpha$}
\end{pspicture}
\end{center}
\caption{Travail d'une force}\label{fig:travail}
\end{figure}
\noindent Si la force $\fF$ de norme $|\fF|$ se translate le long du vecteur $\overrightarrow{AB}$, le \defe{travail}{Travail} de $\fF$ est 
\begin{equation}
  W_{ \overrightarrow{AB} }(\fF)=\fF\cdot\overrightarrow{AB}:=|\fF|\cdot|\overrightarrow{AB}|\cdot\cos(\alpha).
\end{equation}
On comprend facilement pourquoi le produit des normes de $\fF$ et de $\fAB$ arrivent dans la formule : plus une force est grande plus elle va pouvoir faire de travail, et plus le chemin sur lequel elle se déplace est grand, plus il a fallu de travail pour la faire bouger : soulever une pierre de \unit{20}{\kilogram} de \unit{10}{\meter}, c'est plus compliqué qu'une plume de \unit{20}{\gram}    de \unit{10}{\centi\meter}    !

Ce qui peut troubler, c'est le cosinus. Ce que le travail d'une force mesure, c'est l'énergie effectivement donnée à l'objet par la force. Prenons par exemple le cas du du bateau de la figure \ref{fig_bat_W} dans un canal étroit qu'on remorque en tirant depuis la berge entre le point $A$ et le point $B$. Comme le bateau n'arrête pas de se cogner contre le bord (parce qu'on le tire de travers), en fait la composante de la force perpendiculaire à la berge est perdue : seule la composante parallèle $\fF_{\parallel}$ au canal est utilisée pour faire avancer le bateau. 

Toute l'accélération que la composante $\fF_{\perp}$ essaye de donner au bateau est perdue à cause de la force de réaction $\fR$ de la berge.

\newcommand{\prefigbatW}{%
\psset{PointSymbol=none,PointName=none}
	\pstGeonode(0,-1){B}
	\pstGeonode(2,0){C}
	\rput(B|C){\pnode{A}}
	\rput(C|B){\pnode{D}}
	\pstMiddleAB{C}{D}{mCD}
	\pstTransHom{A}{C}{mCD}{0.4}{E}
	\pstTranslation{A}{C}{C}[bFpz]
	\pstRotation[RotAngle=60]{C}{bFpz}[bFz]		% La force arrive avec un angle de 60 degrés
	\pstHomO[HomCoef=1]{C}{bFz}[bF]			% Je place la force qui sera de 1,5 fois la longueur du bateau
	\pstDecompForce{C}{bF}{A}{C}{C}{D}{bFp}{bFn}	% Décomposition de la force
	\pstHomO[HomCoef=-1]{C}{bFn}[bR]		% Placer la force de réaction

	\rput(A){\pstGeonode(0,0.1){Ba}}
	\rput(C){\pstGeonode(0,0.1){Bc}}
	\pstHomO[HomCoef=2]{Bc}{Ba}[dB]
	\pstHomO[HomCoef=2]{Ba}{Bc}[fB]
}


\begin{figure}[ht]
\begin{center}
\begin{pspicture}(-0.5,-1)(6,2)
\prefigbatW
	\psline(C)(A)
	\psline(A)(B)
	\psline(B)(D)
	\psline(D)(E)
	\psline(E)(C)
	\pstMarqueForce{C}{bF}{0.3;0}{$\fF$}
	{\psset{linecolor=red} \pstMarqueForce{C}{bFn}{0.3;90}{$\fF_{\perp}$}}
	{\psset{linecolor=red} \pstMarqueForce{C}{bR}{0.3;180}{$\fR$}}
	{\psset{linecolor=green} \pstMarqueForce{C}{bFp}{0.5;300}{$\fF_{\parallel}$}}
	
	\pstLineAB{dB}{fB}		% Dessiner la berge
	\pstMarquePoint{dB}{0.3;90}{$A$}
	\pstMarquePoint{fB}{0.3;90}{$B$}
\end{pspicture}
\end{center}
\caption{Bateau remorqué par une force non parallèle au canal}\label{fig_bat_W}
\end{figure}

Or, ce que le travail mesure, c'est bien la capacité à faire bouger (plus précisément : accélérer) l'objet sur lequel la force agit. Voyons ça...

Que se passe-t-il si une force $\fF$ agit sur une masse $m$ pendant un temps $t$ ? L'accélération que l'objet sera $a=F/m$, mais $\Delta x=\frac{at^2}{2}$, donc
\[
  \Delta x=\frac{at^2}{2}=\frac{Ft^2}{2m}.
\]
Avec ça, on peut calculer le travail de la force :
\[
  W=F\Delta x=(ma)(\frac{mat^2}{2m})=\frac{ma^2t^2}{2}.
\]
Question énergie cinétique, on sait que la vitesse atteinte est $v=at$, et donc l'énergie cinétique est
\[
  E_c=\frac{mv^2}{2}=\frac{ma^2t^2}{2}=W.
\]
Tout est cohérent; pas mal hein !

On peut être encore plus précis en disant que le travail d'une force est l'énergie que la force donne à un objet, et pas seulement son énergie cinétique. Afin de voir ça, nous allons étudier plus en détail comment fonctionne le travail de la gravitation. Mais avant ça, donnons deux petits bonus. Le travail qu'on a accompli permet déjà de comprendre un objet mystérieux que tu as peut-être étudié l'année passée : le levier hydraulique. Il permet aussi de calculer plus simplement le mouvement sur un plan incliné.

\subsection{Bonus : mouvement sur un plan incliné}
%-------------------------------------------------

Nous avons étudié le mouvement sur un plan incliné en considérant les forces en jeu au point \ref{sss_inclineF}. Une autre méthode consiste à considérer les énergies en jeu. Lorsque la masse avance de $A$ à $B$, il gagne une énergie potentielle 
\[ 
 E_p=mg\cdot BC. 
\]
Cette énergie provient du travail de la force $F$. Comme par ailleurs il n'y a pas d'accélération, le travail de $\fF$ ne fournit pas d'énergie cinétique, et donc \emph{tout} le travail de $\fF$ est converti en énergie potentielle. Ceci fait que $E_p=W_{\fF}$, mais le travail de $\fF$, il est facile à exprimer : c'est $W_{\fF}=F\cdot AB$. Donc
\[ 
  E_p=F\cdot AB.
\]
En égalisant les deux expressions pour $E_p$, on trouve $mg\cdot BC=F\cdot AB$, et donc
\begin{equation}  \label{eq_expdeuxF}
  F=mg\frac{ BC}{ AB }.
\end{equation}
On peut noter la miraculeuse coïncidence de cette formule avec la formule \eqref{eq_expunF} déduite par un autre chemin. 

\subsection{Bonus : le levier hydraulique}
%----------------------------------------

\subsubsection{Comment on faisait avant}
%///////////////////////////////////////

Remémorons nous comment on décrivait le levier hydraulique avec des questions de pressions et de principe de Pascal.


\begin{figure}[ht]
\centering
\begin{pspicture}(-1,-2.3)(5,2)

\psset{PointSymbol=none, PointName=none}
\pstGeonode(0,0){O}(1,0){Ha}  (1.5,0){Hb}   (2,0){Hc} (0,-2){Va}(0,-0.75){Vb}(0,0.2){Vc}
\pstGeonode(0,0){O}(-1,0){mHa}(-1.5,0){mHb}(-2,0){mHc}(0,2){mVa}(0,0.75){mVb}
\pstGeonode(0,0.3){Vd}(0,1){lFu}(0,1.7){lFd}

\pstGeonode(0,0){TorPos}(0,0){A}

\pstTortue{Ha}{B}\pstTortue{Vb}{C}\pstTortue{Hb}{D}
\pstTortue{mVb}{E}\pstTortue{Hc}{F}\pstTortue{Va}{G}
\pstTortue{mHc}{Hii}\pstTortue{mHb}{Hi}\pstTortue{mHa}{H}
   \pspolygon[fillstyle=vlines,fillcolor=red](A)(B)(C)(D)(E)(F)(G)(H)

\pstTranslation{O}{Vc}{A,B,E,F}[RA,RB,RE,RF]
   \psline(A)(RA)
   \psline(B)(RB)
   \psline(E)(RE)
   \psline(F)(RF)

\pstMiddleAB{A}{B}{miAB}
\pstTranslation{O}{Vd}{miAB}[bFu]
\pstTranslation{O}{lFu}{bFu}[oFu]
   \pstMarqueForce{oFu}{bFu}{0.6;140}{$\fF_1$}



\pstMiddleAB{E}{F}{miEF}
\pstTranslation{O}{Vd}{miEF}[bFd]
\pstTranslation{O}{lFd}{bFd}[oFd]
\pstMarqueForce{bFd}{oFd}{0.6;0}{$\fF_2$}

   \rput(RB){\rput(0.5,0){$S_1$}}
   \rput(RF){\rput(0.5,0){$S_2$}}



\end{pspicture}
\caption{Le levier hydraulique}
\end{figure}

Une force $F_1$ est appliquée sur la surface $S_1$. Quelle force est-ce que cela provoque sur la surface $S_2$ ? Le principe de Pascal stipule que toutes les pressions sont égales\footnote{Si la pression était plus élevée en un point qu'au point d'à côté, il y aurait un courant qui se produirait jusqu'à égalisation. Comme le vent entre une haute et une basse pression.}. En particulier, la pression sur la surface $S_1$ est la même que celle sur la surface $S_2$.

La pression sur la surface $S_1$ est imposée par la force qu'on y applique :
\[
   P_1=\frac{F_1}{S_1},
\]
tandis que l'eau s'arrange pour créer la même pression partout. Sur la surface $S_2$, elle doit appliquer la force qu'il faut pour que $P_2=P_2$, c'est à dire
\begin{equation}  \label{eq:hydroun}
F_2=P_2S_2=P_1S_2=F_1\frac{S_1}{S_2}.
\end{equation}



\subsubsection{La notion de travail simplifie les choses !}
%//////////////////////////////////////////////////////////

Il faut remarquer une chose : quand on pousse sur le premier piston, on l'enfonce d'une hauteur $\Delta h_1$, tandis que le second piston monte d'une hauteur $\Delta h_2$ qui n'est pas spécialement la même. Nous allons tenir compte de deux choses :

\begin{enumerate}
\item la conservation du volume d'eau : l'eau est incompressible,
\item la conservation de l'énergie : le travail de la seconde force doit être le même que celui de la première. 
\end{enumerate}
Le volume d'eau déplacé par la première force est le cylindre de base $S_1$ et de hauteur $\Delta h_1$, c'est à dire
\[
   V_1=S_1\Delta h_1.
\]
Évidement, le volume déplacé de l'autre côté est le même :
\[
   V_2=S_2\Delta h_2\stackrel{!}{=}V_1.
\] 
Première conclusion : on sait de combien le second piston va se lever si on pousse le premier. Les égalités $V_1=S_1\Delta h_1=V_2=S_2\Delta h_2$ donnent 
\[
\Delta h_2=\Delta h_1\frac{S_1}{S_2},
\]
ou encore
\begin{equation}  \label{eq:DhuSDhd}
\frac{\Delta h_2}{\Delta h_1}=\frac{S_1}{S_2}.
\end{equation}
Ça, c'était la subtilité : expliciter la conservation du volume d'eau. Maintenant, on fait la conservation de l'énergie comme d'habitude : le travail $F_1\Delta h_1$ que l'on exécute au niveau du premier piston doit être le même que celui que l'eau produit au niveau du second : $F_2\Delta h_1$. Donc on a 
\[
    \frac{F_1}{F_2}=\frac{\Delta h_2}{\Delta h_1}.
\]
En y reportant l'équation \eqref{eq:DhuSDhd}, on trouve
\begin{equation}
\frac{F_1}{F_2}=\frac{S_1}{S_2},
\end{equation}
ce qui est exactement la même chose que \eqref{eq:hydroun}. 

Cette nouvelle méthode pour décrire le piston hydraulique est beaucoup plus intéressante que la première parce qu'elle se base sur deux lois de conservation : le volume d'eau et l'énergie. Il est important de remarquer qu'on a déduit un résultat pas évident seulement en disant que telle et telle quantité sont conservées. La physique moderne attache une grande importance aux quantités conservées.





\subsection{Récapitulons}
%////////////////////////////

Il existe plusieurs types d'énergies dont les plus courants sont les énergies {\bf cinétiques} et {\bf potentielles}. Quand un objet ou un système est livré à lui même (quand il est \emph{isolé}) son énergie est conservée. Le type d'énergie le plus facile à trouver est l'énergie cinétique : quand un objet de masse $m$ se déplace avec une vitesse $v$, son énergie cinétique vaut
\begin{equation}
E_c=\frac{mv^2}{2}.
\end{equation}
L'énergie potentielle d'un objet se calcule en général en se demandant par quel processus on peut transformer l'énergie potentielle en énergie cinétique : quelle vitesse peut acquérir l'objet grâce à son potentiel ? Quand on a répondu à cette question, l'énergie \emph{potentielle} de l'objet est l'énergie cinétique que l'objet \emph{pourrait} acquérir. Jusqu'ici, la seule énergie potentielle qu'on ait vue est celle liée à la gravitation.

Les forces servent à transformer les énergies : soit elles font passer de l'énergie d'un objet à un autre, soit elles transforment de l'énergie potentielle en énergie cinétique. La capacité qu'une force a à apporter de l'énergie à l'objet sur lequel elle s'applique est la \emph{travail} de la force. 

Quand un travail est positif, la force transforme de l'énergie potentielle en énergie cinétique. Quand le travail est négatif, ça transforme de l'énergie cinétique en énergie potentielle.

La somme des énergies cinétiques et potentielle est appelée \defe{l'énergie mécanique}{Énergie!mécanique}. L'énergie mécanique d'un système isolé est conservée. Elle varie quand une force extérieur travaille dessus.
\begin{equation}
E_t=E_c+E_p.
\end{equation}
Le \defe{théorème de l'énergie cinétique}{Théorème!de l'énergie cinétique} dit  :

\begin{theoreme}
Le travail effectué pendant une certaine durée par la résultante des forces appliquées à un corps est égal à la variation d'énergie cinétique du corps durant cette même durée.
\end{theoreme}

\subsection{Exemples}
%//////////////////////

Reprenons les situations vues plus haut à la lumière de ces explications. Quand je tire un objet qui se met en mouvement, je fais un travail positif : le mouvement va dans le sens de la force. La transformation d'énergie est la suivante : le potentiel de la source de la force (c'est à dire l'énergie contenue dans les cellules de mes muscles) est transformée en énergie cinétique pour l'objet tiré.

Quand un objet tombe, la gravitation fait un travail positif. L'énergie potentielle de l'objet est transformée en énergie cinétique : au fur et à mesure que l'objet tombe, il gagne en vitesse.

Quand on monte un objet, il y a deux forces : celle musculaire qui fait un travail positif et qui transforme l'énergie potentielle des muscles en énergie cinétique pour l'objet. Mais la force de gravitation fait un travail négatif : elle transforme l'énergie cinétique en énergie potentielle. Voilà pourquoi plus un objet est lourd, plus il est difficile à soulever : le travail qu'on fourni pour communiquer une accélération est constamment pompé par la gravitation qui en fait de l'énergie potentielle ! Si la force qui tire vers le haut est plus grande que la gravitation, son travail sera plus grand, et l'un dans l'autre, l'objet gagnera en vitesse parce que la gravitation ne pompera pas l'énergie aussi vite qu'on la communique.

\paragraph{Mouvement horizontal}
%*******************************

Lorsqu'un objet se déplace horizontalement, la force de gravité ne travaille pas. C'est facile à voir dans le calcul : la force de gravité est verticale et donc perpendiculaire au déplacement; le cosinus est donc nul. Pourtant, plus un objet est lourd, plus il est difficile à pousser; on a bien l'impression que la gravité a quelque chose à voir là-dedans. D'abord remarquons qu'en mettant des roulettes sous l'objet ça va beaucoup mieux, tandis que pour le soulever, l'ajout de roulettes n'aide pas; le phénomène n'est donc pas vraiment identique. En fait quand on pousse un objet, il faut travailler contre la force de frottement entre l'objet et le sol.

\begin{exercice}
Demandes-toi le travail effectué par la gravitation sur un satellite qui orbite autour de la Terre. Quand tu auras compris pourquoi ce travail est nul, tu pourras te demander comment ça se fait que certains satellites retombent quand même sur Terre.
\end{exercice}

\Exo{024}
\Exo{025}
\Exo{029}

\subsection{Travail et intégration}
%----------------------------------

Tu as déjà vu les intégrales au cours de mathématique ? Une fois de plus, tu avais envie de dire que le prof de math est une peau de vache qui te fait étudier des trucs inutiles et compliqués dans le but de t'avoir en fin d'année; et une fois de plus, tu te trompes doublement : d'une part en fait les intégrales c'est facile, et d'autre part, c'est très utile en physique, c'est à dire pour comprendre le monde, la nature, les objets qui t'entourent ainsi le sens de la vie.

Nous avons vu comment se passe le calcul d'un travail pour une force constante, éventuellement non parallèle au déplacement. Voyons maintenant comment il faut faire pour calculer le travail d'une force non constante. Prenons comme exemple un train qui démarre alors que son moteur est froid. Au fur et à mesure que moteur tourne, il s'échauffe et peut donc fournir une force de plus en plus grande. L'accélération du train n'est donc pas constante.

Mettons que la force que la locomotive déploie soit la fonction $F(t)$, et que la masse du train soit $m$. L'accélération au temps $t$ est donc $a(t)=F(t)/m$. Bien sûr, la locomotive chauffe lentement, c'est à dire qu'entre le temps $10$ secondes et $11$ secondes, la force n'a pas beaucoup changée. En approximation, on peut dire qu'elle a été constante :
\[ 
  F_{moyenne}(10\to 11)=\frac{ F(10)+F(11) }{ 2 }.
\]
Pendant cette seconde, le train a avancé d'une distance\footnote{Je ne te cache pas qu'en pratique, c'est très difficile de savoir $d$, vu que l'accélération varie avec le temps. Mais pour l'instant, on fait de la théorie; on n'est pas dans les calculs. Disons donc $d$, en gardant en tête que ce n'est en réalité pas facile à calculer.} $d$. Durant cette seconde, le travail de la force est donc 
\[ 
  W(10\to11)=F_{m}(10\to 11)d.
\]
Si on refait cette approximation à chaque seconde, le travail total pour $10$ secondes de trajet vaut approximativement
\[ 
\begin{split}
  W\simeq&F_{m}(1\to 2)d(1)+F_{m}(2\to 3)d(2)+F_{m}(3\to 4)d(3)+F_{m}(4\to 5)d(4)\\
		&+F_{m}(5\to 6)d(5)+F_{m}(6\to 7)d(6)+F_{m}(7\to 8)d(7)+F_{m}(8\to 9)d(8)\\
		&+F_{m}(9\to 10)d(9).
\end{split}
\]
Bien entendu, si on veut plus de précision, on peut découper les dix secondes en petits intervalles d'un millième de secondes. Ça ne te rappelle rien, ce genre de raisonnement ? On peut toujours remplacer $d(t)$ par sa valeur $v_m(t)\Delta t$ en fonction de la vitesse. Ici, $\Delta t$ est l'intervalle de temps choisit (c'est à dire une seconde dans l'exemple). Disons qu'on découpe les $10$ secondes en $1000$ parties, c'est à dire $\Delta t=\unit{10/1000}{\second}$. Dans ce cas, le travail sera approximé par
\[ 
  W\simeq=\sum_{i=1}^{1000}F_m(i\to i+1)v_m(t\to i+1)\Delta t.
\]
Là, tu l'as reconnue hein ? Quand on découpe de plus en plus, ce qu'on obtient à la limite du découpage de plus en plus précis, c'est
\begin{equation}	\label{EqTravailt}
W=\int_0^{10} F(t)v(t)dt.
\end{equation}
Montons d'un cran. $l(t)=v(t)dt$, c'est le déplacement effectué pendant le temps \og infinitésimal\fg{} $dt$. Jusqu'à présent, nous avons parlé d'un train et supposé que le déplacement se faisait sur des rails droites, de telle manière à ce que le travail soit la force multipliée par le déplacement sans problèmes de cosinus et de produits scalaires. Tu te souviens que quand la force n'est pas parallèle au déplacement, on doit faire $W=\overrightarrow{ F }\cdot\overrightarrow{ l }$ si $\overrightarrow{ l }$ est le déplacement. Maintenant, c'est facile de voir comment modifier la formule \eqref{EqTravailt} pour tenir compte de cet effet :
\begin{equation}
	W=\int_A^B \overrightarrow{ F }\cdot \overrightarrow{ dl },
\end{equation}
où l'intégrale est prise sur le chemin qui va du point $A$ au point $B$. C'est à dire que si on veut savoir le travail que déploie une force pour déplacer un objet d'un point $A$ à un point $B$, il faut intégrer sur la courbe qui représente le trajet.


\section{Puissance}
%++++++++++++++++++

La \defe{puissance}{Puissance} d'une machine (qui effectue un travail) est la quantité d'énergie dégagée par unité de temps. Qu'est-ce à dire ? Pour soulever une pierre de \unit{3}{\kilogram} à une hauteur de \unit{5}{\meter}, il faut un \emph{travail} de $mgh=9.81\cdot 3\cdot 5=147 J$. Que je soulève cette pierre en vingt secondes ou deux heures, ça ne change rien au résultat de travail effectué. Par contre, ça change quelque chose à la puissance développée par mes muscles. Dans le premier cas, j'ai dépensé \unit{147}{\joule}    en vingt secondes, c'est à dire \unit{7.3}{\joule\per\second}. Dans le second cas, j'ai dépensé les mêmes $147$ joules en $7200$ secondes, ce qui fait la puissance plus modeste de \unit{0.02041}{\joule\per\second}.

Dans le premier cas, on dira que j'ai une puissance de $7.31W$ (watt) et dans le second, on dira $0.02041W$. On écrit :
\[
 \text{puissance}=\frac{\text{énergie}}{\text{durée}}=\frac{\text{travail}}{\text{durée}}.
\]
Ne pas confondre :
\begin{description}
\item[Le joule] qui est l'unité d'énergie qui correspond au déplacement d'un mètre du point d'application d'une force de un newton.
\item[Le watt] qui est l'unité de puissance qui correspond à une énergie de un joule dépensé en une seconde. La puissance est vitesse à laquelle un moteur déverse des joules dans l'objet sur lequel elle travaille.
\end{description}

Il existe une petite formule sympa qui permet de trouver la puissance d'une force qui sert à déplacer un objet. Par exemple la puissance du moteur d'un train qui avance à une certaine vitesse. Mettons qu'on ait une force $\fF$ qui avance à une vitesse $v$. Le travail que la force effectue durant un temps $\Delta t$ vaut 
\[ 
 W=F\cdot d 
\]
où $d$ est la distance parcourue par le moteur durant le temps $\Delta t$, c'est à dire $v\Delta t$. Ça c'est pour le travail : $W=Fv\Delta t$. En ce qui concerne la puissance, il faut diviser le travail par la durée que le moteur a mise pour effectuer le travail, c'est à dire $\Delta t$. Finalement on trouve :
\begin{equation} \label{eq_puissFv}
  P=Fv.
\end{equation}
