% This is part of Un soupçon de physique, sans être agressif pour autant
% Copyright (C) 2006-2009
%   Laurent Claessens
% See the file fdl-1.3.txt for copying conditions.


\begin{corrige}{212}

En vertu du théorème \ref{ThoLimCont}  (limite et continuité), il est suffisant de montrer que pour tout $a\in\eR$,
\[ 
  \lim_{x\to a}\cos(x)=\cos(a).
\]
C'est bien pour ça que ce théorème est génial : il permet de prouver des continuités en calculant des limites. Nous avons par la proposition \ref{PropChmVarLim} :
\begin{equation}
\lim_{x\to a}\cos(x)=\lim_{\epsilon\to 0}\cos(a+\epsilon)=\lim_{\epsilon\to0}\Big( \cos(a)\cos(\epsilon)-\sin(a)\sin(\epsilon) \Big),
\end{equation}
en utilisant la formule \eqref{EqLimLinRes}, nous trouvons à calculer
\[ 
  \cos(a)\Big(\lim_{\epsilon\to 0}\cos(\epsilon)\Big)-\sin(a)\Big( \lim_{\epsilon\to 0}\sin(\epsilon)\Big).
\]
\end{corrige}

