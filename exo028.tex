% This is part of Un soupçon de physique, sans être agressif pour autant
% Copyright (C) 2006-2009
%   Laurent Claessens
% See the file fdl-1.3.txt for copying conditions.


\begin{exercice}\label{exo028}

Un étudiant en mathématique rentre de Louvain-la-Neuve vers Bruxelles à vélo. Son poids\footnote{Est-ce bien un poids ? Comment il faudrait corriger l'énoncé pour ne pas commettre de faute ?} est de \unit{80}{\kilo\gram}, y compris le vélo. Il grimpe l'ultime et légendaire côte de Overijse à $7$\%. Il maintient une vitesse constante de \unit{3}{\meter\per\second} sur son vélo dont la masse est de \unit{5}{\kilogram}. Calcule la force et la puissance développée.

Il est maintenant sur la petite pente vers Jezus-Eik et en profite pour prendre un repos bien mérité : il se laisse aller. Quel sera son mouvement ?

\corrref{028}
\end{exercice}
