\begin{corrige}{003}
\begin{enumerate}
\item Il suffit de prendre un train pour se rendre compte que la trajectoire observée par le voisin est verticale.
\item La seconde question est plus subile. En effet, le GSM continue le mouvement horizontal que le train lui a communiqué, c'est à dire un mouvement uniforme. Mais en même temps, il effectue un mouvement en chutte libre dans la direction verticlale. En équation paramétriques on a donc $x(t)=v_0t$ et $y(t)=\frac{gt^2}{2}$. La conversion en coordonées cartésiennes donne une parabole, c'est à dire une équation $y=f(x)$ du second degré.
\end{enumerate}
\end{corrige}
% This is part of Un soupçon de physique, sans être agressif pour autant
% Copyright (C) 2006-2009
%   Laurent Claessens
% See the file fdl-1.3.txt for copying conditions.


