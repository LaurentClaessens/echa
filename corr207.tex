\begin{corrige}{207}


\begin{center}
\begin{tabular}{l|c|c|c|l}
$a$ est un	&	majorant&	supremum	&	maximum	& de $A$\\\hline
		&	oui	&	oui		&	oui	&	$1$ pour $[0,1]$	\\ 
		&	oui	&	oui		&	non	&	$1$ pour $[0,1[$	\\ 
		&	oui	&	non		&	non	&	$2$ pour $[0,1]$	\\ 
		&	oui	&	non		&	oui	&	maximum implique supremum	\\ 
		&	non	&	non		&	oui	&	idem	\\ 
		&	non	&	non		&	non	&	$-1$ pour $[0,1]$	\\ 
		&	non	&	oui		&	non	&	supremum implique majorant	\\ 
		&	non	&	oui		&	oui	&	idem	\\ 
\end{tabular}
\end{center}
Maintenant je t'encourage fortement à recommencer la même chose en remplaçant \emph{majorant}, \emph{supremum} et \emph{maximum} par \emph{minorant}, \emph{infimum} et \emph{minimum}.
\end{corrige}
% This is part of Un soupçon de physique, sans être agressif pour autant
% Copyright (C) 2006-2009
%   Laurent Claessens
% See the file fdl-1.3.txt for copying conditions.


