% This is part of Un soupçon de physique, sans être agressif pour autant
% Copyright (C) 2006-2009
%   Laurent Claessens
% See the file fdl-1.3.txt for copying conditions.


\begin{corrige}{017}
Le travail effectué par la grue est l'énergie qu'il faut donner au béton pour lui permettre d'avancer de \unit{20}{\meter} dans le champ de gravitation. C'est à dire $W=P\cdot h$ où $P$ est le poids du béton; donc
\[ 
  W=mgh=50\cdot 9.81\cdot 20=\unit{9810}{\joule}.
\]
Si tu n'es pas sûre du signe, il y a au moins deux manières de savoir que c'est bien positif. La première est de remarquer que le déplacement se fait dans le même sens que la force (c'est pour ça qu'on a noté $50\cdot 9.81\cdot 20$ et non des choses du genre $(-50)\cdot 9.81\cdot 20$ ou $50\cdot9.81\cdot (-20)$).

La seconde est de remarquer que la grue transforme de l'énergie potentielle (par exemple ses batteries ou son pétrole) en de l'énergie cinétique (de la charge qui monte). Cependant, la force de gravitation se charge de transformer immédiatement cette énergie cinétique en énergie potentielle (le travail de la gravitation est négatif, ce qui est cohérent avec le fait que son sens est le contraire de celui du déplacement), donc on ne \og voit\fg{} pas bien ce passage par du cinétique.

La puissance d'une machine est la quantité de joules que la grue dévesre par seconde dans l'objet sur lequel elle travaille. Ici, c'est donc le nombre de joules par secondes que reçoit le bloc de béton. En l'occurence, le béton reçoit $9810$ joules en $30$ secondes, soit $\unit{\frac{9810 }{ 30 }}{\joule\per\second}=\unit{327}{\watt}$.

Il s'agit maintenant de soulever \unit{100}{\kilogram}, c'est à dire deux fois plus de masse. Étant donné que le travail de la gravitation est proportionel à la masse, il faudra un travail deux fois plus grand. Autrement dit, il faut deux fois plus de joules à un objet de \unit{100}{\kilogram} pour monter de la même hauteur qu'un objet de \unit{50}{\kilogram}. Comme la puissance est la même, il reçoit autant de joules par seconde que le premier objet. Pour en avoir deux fois plus il faut donc deux fois plus de temps, soit une minute.

\end{corrige}
