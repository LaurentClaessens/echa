\begin{corrige}{208}

Il est naturel de regarder la fonction $d(t)$ qui indique la distance parcourue en un temps $t$. Si $t$ se compte en heures et les distances en kilomètres, cette fonction vérifie évidement
\begin{align*}
d(0)&=0&\text{et}&&d(2)&=10.
\end{align*}
Cela dit, ce qui nous intéresse vraiment, c'est la distance qu'il parcours en une heure, c'est à dire la fonction
\[ 
  f(t)=d(t+1)-d(t).
\]
Cette fonction satisfait 
\begin{align*}
f(0)&=d(1)&\text{et}&&f(1)=10-d(1).
\end{align*}
Ce que tu voudrais prouver, c'est que cette fonction passe par la valeur $5$ entre $t=0$ et $t=1$. Eh bien oui, le nombr $5$ est quelque part entre $d(1)$ et $10-d(1)$, quelle que soit la valeur exacte de $d(1)$. En effet, $5$ est la moyenne arithmétique :
\[ 
  \frac{ d(1)+\big(10-d(1)\big) }{ 2 }=5.
\]

\end{corrige}
% This is part of Un soupçon de physique, sans être agressif pour autant
% Copyright (C) 2006-2009
%   Laurent Claessens
% See the file fdl-1.3.txt for copying conditions.


