\begin{corrige}{007}

Le premier cas est très simple parce que l'énergie cinétique que la pierre acquière est l'énergie potentielle perdue durant sa chute de $\Delta h=\unit{15-1.8=13.2}{\meter}$, soit $mg\Delta h=\unit{388.5}{\joule}$.

Dans le second cas, il faut juste ajouter l'énergie cinétique initiale : $mv^2/2$ avec $m=\unit{3}{\kilo\gram}$ et $v=\unit{1}{\meter\per\second}$. L'énergie cinétique initiale est donc \unit{1.5}{\joule}, ce qui fait que l'ennemi prendra $390$ joules sur la tête.

\end{corrige}
% This is part of Un soupçon de physique, sans être agressif pour autant
% Copyright (C) 2006-2009
%   Laurent Claessens
% See the file fdl-1.3.txt for copying conditions.


