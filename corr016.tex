% This is part of Un soupçon de physique, sans être agressif pour autant
% Copyright (C) 2006-2009
%   Laurent Claessens
% See the file fdl-1.3.txt for copying conditions.


\begin{corrige}{016}

La caisse se déplace à vitesse constante. Pourquoi ? Parce qu'il y a \emph{deux} forces qui s'y appliquent : la première est celle de frottement, tandis que la seconde la force qui tire dont on parle. Il est dit dans l'énoncé que ces deux forces se compensent exactement. Donc il n'y a globalement aucune force qui s'appliquent à la caisse, ce qui fait qu'il avance à vitesse consante.

Remarque qu'au début, pour faire démarrer la caisse, il a fallu pendant un certain temps appliquer une force suppérieure à celle de frottement.

Le travail d'une force $\fF$ qui se déplace d'une distance $d$ est par définition $W=\| \fF \|f$ quand la force est parallèle au déplacement, ce qui est le cas ici parce qu'on parle d'un plan horizontal et d'une force horizontale. On a donc
\[ 
  W=\unit{10}{\newton}\cdot\unit{2.3}{\meter}=\unit{23}{\joule}
\]



\end{corrige}
