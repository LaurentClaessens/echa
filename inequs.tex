% This is part of Un soupçon de physique, sans être agressif pour autant
% Copyright (C) 2006-2009
%   Laurent Claessens
% See the file fdl-1.3.txt for copying conditions.


%%%%%%%%%%%%%%%%%%%%%%%%%%
%
   \section{Inéquations à une inconnue}
%
%%%%%%%%%%%%%%%%%%%%%%%%

\subsection{Rappel théorique}
%-----------------------------

\subsubsection{Addition et soustraction}


La première règle est que si Laurent est plus grand que Xavier, et si ils se mettent tout les deux sur des échasses de \unit{1}{\meter} de haut, ça ne changera rien : Laurent restera plus grand que Xavier. 

En termes plus formels, on dit que quand on a une inégalité, si on ajoute la même chose aux deux membres, l'inégalité reste. En formule :
\[ 
  a\leq b\Longleftrightarrow\, a+c\leq b+ c.
\]
Exemple : tu es d'accord que $4 > 3$. Eh bien sans surprises, on a que $4+10 > 3+10$.

La bonne nouvelle c'est que ça marche aussi quand on retranche au lieu d'ajouter. En effet, si Gaston a fait une meilleur interrogation que Gustave, mais que le prof décide de leur enlever deux point à chacun, Gaston aura toujours de meilleurs points que Gustave. En formule :
\[ 
  a\leq b\Longleftrightarrow\, a-c\leq b- c.
\]
En résumé, on peut dire que
\emph{Quand on ajoute ou quand on retranche un même réel aux deux membres d'une inéquation, on obtient une inéquation de même sens}
Cette s'exprime en formule sous la forme suivante :
\[ 
  a\leq b\Longleftrightarrow\, a\pm c\leq b\pm c.
\]

\subsubsection{Multiplication et division}

Maintenant les choses se compliquent. Si Arthur a plus d'argent que Léon, et qu'ils multiplient tout les deux par trois leur argent, évidement Arthur reste plus riche : $a > b$ implique que $3a>3b$.

Petit exemple de cinématique. Louis et Michel font du vélo dans le même sens et la même direction. Louis avance à \unit{15}{\kilo\meter\per\hour}, tandis que Michel pédale un peu moins bien et avance à seulement \unit{10}{\kilo\meter\per\hour}. On choisit bien entendu de placer l'axe de notre repère dans le sens de leur mouvement. Ainsi ces vitesses sont positives.

À un moment donné, les deux enfants entendent qu'une camionnette de glace s'est arrêtée un peu derrière eux. Immédiatement, ils font demi tour et pédalent deux fois plus vite qu'avant. Louis avance maintenant à \unit{30}{\kilo\meter\per\hour} et Michel à \unit{20}{\kilo\meter\per\hour}. Oui, mais ils vont maintenant \emph{dans le sens négatif} par rapport aux axes !

Donc en bon physicien qui respecte ses choix d'axes, Louis avance en réalité à \unit{-30}{\kilo\meter\per\hour} et Michel à \unit{-20}{\kilo\meter\per\hour}. C'est donc Michel qui avance le plus vite parce que $-20 > -30$.


Un exemple d'arithmétique. Tu sais que $8 > 5$. Maintenant multiplie ces deux chiffre par $-3$. Tu obtiens $-24$ et $-15$. Or, $-15>-24$. On a donc trouvé que $a>b$ implique que $-3a<-3b$.

\begin{quote}
Quand on multiplie deux membres d'une inéquation par un même réel strictement négatif (c'est à dire négatif et non nul), on obtient une équation \emph{de sens contraire}.
\end{quote}
En formule :
\[
\left\{ 
\begin{matrix}
a>b\\
c<0
\end{matrix}
\right.\,
\Longrightarrow
 a\cdot c>b\cdot c
\]







