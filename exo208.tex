% This is part of Un soupçon de physique, sans être agressif pour autant
% Copyright (C) 2006-2009
%   Laurent Claessens
% See the file fdl-1.3.txt for copying conditions.


\begin{exercice}\label{exo208}
Un poète fait sa ballade quotidienne de \unit{10}{\kilo\meter} en deux heures, entre cinq heure (du matin) et sept heures. Comme il est un peu physicien aussi en plus d'être poète, il remarque que sa vitesse moyenne était de \unit{5}{\kilo\meter\per\hour}. Notre bonhomme étant de surcroît fana de hi-tech, il a son \href{http://fr.wikipedia.org/wiki/Podomètre}{podomètre}, et il remarque qu'entre cinq heure quart et six heure quart, il a parcouru exactement 5 kilomètres.

Il refait la même ballade le lendemain, toujours en deux heures; et il remarque que cette fois, il a parcouru 5 kilomètres exactement entre cinq heures vingt sept et six heures vingt sept.

Intrigué, il refait la même ballade tous les jours de la semaine, et remarque que à tous les coups, il y a un intervalle de une heure durant lequel il a parcouru exactement 5 kilomètres. Hélas pour lui, être poète, physicien et amateur de hi-tech ne suffit pas pour être bon en math, et il est incapable de prouver ce qu'il croit, à savoir que si il fait une ballade de dix kilomètres en deux heures, il y aura \emph{toujours} un intervalle de une heure durant lequel il aura parcouru exactement 5 kilomètres. Prouve-le.
\corrref{208}
\end{exercice}
