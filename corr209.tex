% This is part of Un soupçon de physique, sans être agressif pour autant
% Copyright (C) 2006-2009
%   Laurent Claessens
% See the file fdl-1.3.txt for copying conditions.


\begin{corrige}{209}

Si $f$ est une telle fontion, il faut prouver que la fonction $g(x)=f(x)-x$ passe par zéro. Que peut valoir $g(0)$ ? Par définition, $g(0)=f(0)\in[0,1]$. Première remarque : si $g(0)=0$, c'est que $f(0)=0$, et donc $0$ est un point fixe de $f$. Supposons donc que $g(0)\neq 0$. Dans ce cas, nous avons $g(0)>0$ (j'insiste sur le strict de cette inégalité).

En faisans le même raisonement (fais-le !), tu trouves que $g(1)<0$. Le théorème de la valeur intermédiaire conclut parce que $g$ est continue (pourquoi ?).

\end{corrige}
