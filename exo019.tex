\begin{figure}[ht]
\centering
\begin{pspicture}(-0.5,-0.5)(4,2.5)
  \psset{PointSymbol=none, PointName=none}
\prefigzerounneuf
   \psline(A)(C)
   \psline(C)(B)
   \psline(A)(B)
  
   \pstCircleAB{Oa}{Ob}
   \pstCircleAB{Pa}{Pb}

   \psline(Cg)(Cgh)
   \psline(Cgh)(Cdh)
   \psline(Cd)(Cdh)
   \psline(Cg)(Cd)

\pstMarqueForce{Cc}{bR}{0.4;30}{$\fR$}

   \pstMarqueForce{Cc}{bF}{0.3;90}{$\fF$}

   \pstMarqueForce{Cc}{bG}{0.4;45}{$\fG$}
\end{pspicture}
\caption{Chariot sur un plan incliné pour l'exercice \ref{exo:chariotincli}.}\label{fig:chariotincli}
\end{figure}

\begin{exercice} \label{exo:chariotincli}\label{exo019}
Maintenant nous avons affaire à quelqu'un de plus malin que le joueur de bowling de l'exercice \ref{exo:incliun}. Le chariot de la figure \ref{fig:chariotincli} est tiré avec une force constante $\fF$ parallèle à la pente. Il est également soumis à la gravitation $\fG$ et à une force de réaction $\fR$ perpendiculaire au plan incliné.
\begin{enumerate}
\item Trouvez le travail de toutes forces pour un déplacement entre $A$ et $B$, et puis pour une déplacement de $B$ vers $A$.
\item Si les forces $\fF$ et $\fG$ sont telles que $\|\fF\|=\|\fG\|$, est-ce que le chariot va monter, rester en place ou descendre ?
\end{enumerate}

\corrref{019}

\end{exercice}
% This is part of Un soupçon de physique, sans être agressif pour autant
% Copyright (C) 2006-2009
%   Laurent Claessens
% See the file fdl-1.3.txt for copying conditions.


