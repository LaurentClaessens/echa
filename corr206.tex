% This is part of Un soupçon de physique, sans être agressif pour autant
% Copyright (C) 2006-2009
%   Laurent Claessens
% See the file fdl-1.3.txt for copying conditions.


\begin{corrige}{206}
Le minimum d'un sous-ensemble $A$ de $\eR$ est un élément qui est plus petit que tous les éléments de $A$. En formule, on dit que $m$ est un \defe{minorant}{Minorant} de $A$ si 
\[ 
  \forall x\in A,\,x\geq m.
\]
L'\defe{infimum}{Infimum} de $A$ est le plus grand minorant. Si $m$ est l'infimum, alors $\forall x>m$, il existe $y\in A$ avec $y<x$. Et enfin, un \defe{minimum}{Minimum} de $A$ est un infimum qui appartient à $A$.
\end{corrige}
