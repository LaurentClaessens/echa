% This is part of Un soupçon de physique, sans être agressif pour autant
% Copyright (C) 2006-2009
%   Laurent Claessens
% See the file fdl-1.3.txt for copying conditions.


%Copyright (c) 2006 Claessens Laurent. Permission is granted to copy, distribute and/or modify this document under the terms of the  GNU Free Documentation License, Version 1.2 or any later version published by the Free Software Foundation; with no Invariant Sections, no Front-Cover Texts, and no Back-Cover Texts. A  copy of the license is included in the section entitled "GNU Free Documentation License".
\begin{corrige}{011}

Le second dynamomètre indique également \unit{2}{\newton} parce quele mur agit de la même façon sur le premier que la masse sur le second. En effet, la masse qui pend tire sur le dynamomètre, et pour que celui-ci reste en équilibre, il faut qu'une force égale en grandeur et opposée tire le dynamomètre vers la gauche. Peu importe que cette force soit une force de réaction du mur ou une force de pesenteur d'une autre masse, le résultat est le même : le dynamomètre est tiré dans les deux sens et reste en équilibre.
\end{corrige}
