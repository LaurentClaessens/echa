%+++++++++++++++++++++++++++++++++++++++++++++++++++++++++++++++++++++++++++++++++++++++++++++++++++++++++++++++++++++++++++
\section{Vecteurs}

Il y a plusieurs façons de voir un vecteur. Un vecteur, c'est une flèche que l'on peut placer un peu partout dans le plan. Un vecteur c'est la donnée d'un déplacement dans le plan : la donnée de combien on s'est déplacé vers le haut et vers la droite.

Bref, un vecteur c'est la donnée de deux point du plan : le point de départ et celui d'arrivée. On peut distinguer deux types de vecteurs.
\begin{description}

\item[Les vecteurs \href{http://fr.wikipedia.org/wiki/Vente_liée}{liés}] sont des vecteurs accrochés à un point du plan. Si la figure \ref{FigVectLiesoupas} représente des vecteurs liés, alors les vecteurs $\overrightarrow{OA}$ et $\overrightarrow{XY}$ sont différents. Ce type de vecteurs se pense comme des forces : ils ont un point d'application. Deux forces de même intensité et même direction qui s'appliquent à des points différents sont différentes.

\item[Les vecteurs \href{http://fr.wikipedia.org/wiki/Portail:Logiciels_libres}{libres}] sont, comme leur nom l'indique, libre de se déplacer dans le plan. Si la même flèche est placée à différents endroits, elle représente le même vecteur libre. Donc sur la figure \ref{FigVectLiesoupas}, $\overrightarrow{OA}$ et $\overrightarrow{ XY }$ représentent le même vecteur libre.
\end{description}

\begin{figure}
\caption{Figure à faire}  \label{FigVectLiesoupas}
\end{figure}

%+++++++++++++++++++++++++++++++++++++++++++++++++++++++++++++++++++++++++++++++++++++++++++++++++++++++++++++++++++++++++++
\section{Équations de droites}

% Quand je serai ici, il faudra faire un lien vers les exercices ref{Exoeqsssollindegun} et le suivant pour dire qu'une équation du premier degré a soit zéro soit une infinité de solutions.

\section{Système de cordonnée polaire}
%----------------------------------------

Tu sais comment il est possible de repérer un point dans le plan à partir de ses coordonnées cartésiennes $(x,y)$. Si tu as fait un peu de scoutisme, tu connais sans doute déjà le système de coordonnées polaires (même si tu n'en est pas conscient).

