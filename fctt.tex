% This is part of Un soupçon de physique, sans être agressif pour autant
% Copyright (C) 2006-2009
%   Laurent Claessens
% See the file fdl-1.3.txt for copying conditions.


\documentclass[a4paper,12pt]{book}


\usepackage{pstricks,pst-node,pst-abspos,pst-plot,pstricks-add}
\usepackage{multido}
\usepackage{fp}
\usepackage{ifthen}
\usepackage{pst-eucl}
\usepackage{subfigure}
\usepackage{graphicx}


\usepackage[ps2pdf]{hyperref} 		%Doit êre appelé en dernier.

\usepackage{ucs}
\usepackage[utf8x]{inputenc}
\usepackage[english,frenchb]{babel}



\newcommand{\pstRayon}[2]{%
  \psset{PointSymbol=none, PointName=none}
\pstMiddleAB{#1}{#2}{inter}
{\psset{linecolor=blue}
\psline[arrows=->](#1)(inter)  % Ce qui serait bien c'est d'avoir la longeur du segment, de la diviser par deux et de mettre la flèche bien au milieu.
\psline(#1)(#2)
} % fin du psset
}

\begin{document}

% #1 est le point où le rayon va traverser l'interface
% #2 est un autre point de l'interface
% #3 est la distance à laquelle le rayon part
% #4 est l'angle avec lequel il arrive (angle en degré avec la normale)
% #5 est le temps depuis lequel il est parti
% #6 est la vitesse de la lumière dans le second milieu; celle dans le premier est 1.
% #7 est le nom du point pstricks de départ; c'est cette macro qui la place
% #8 est le nom du point pstricks d'arrivée.
\newcommand{\parclum}[8]{%
\FPeval{AngleNorRad}{\FPpi*#4/180}
\FPeval{AngleTgRad}{\FPpi/2-\AngleNorRad}
\FPeval{AngleTgDeg}{#4+90}

\FPround{\AngleTgDeg}{\AngleTgDeg}{3}
\FPround{\AngleTgRad}{\AngleTgRad}{3}
\FPround{\AngleNorRad}{\AngleNorRad}{3}

   \pstRotation[RotAngle=\AngleTgDeg]{#1}{#2}[lsept]
   \rput(E){\pstGeonode(#3;0){inter}}					% Place un point à une distance #3 de E pour ensuite construire le cercle correspondant
   \pstInterLC[Radius=\pstDistAB{E}{inter}]{#1}{lsept}{#1}{}{#7}{interF}


\FPiflt{#5}{#3}{%		Le cas où le temps est plus petit que la distance est facile et est traité ici
   \rput(P){\pstGeonode(#5;0){inter}}	
   \pstInterLC[Radius=\pstDistAB{#7}{inter}]{#7}{#1}{#7}{}{F}{#8}
  
		}		% fin de la possibilité où le temps est plus court que le trajet
\else{%
   \FPeval{trest}{(#3-#5)/#6}	% Le temps qu'il reste à voyager au moment où le rayon arrive à l'interface, divisé en l'indice de réfraction pour normaliser à un.
   \rput(P){\pstGeonode(\trest;0){inter}}	
   \pstInterLC[Radius=\pstDistAB{#7}{inter}]{#7}{#1}{#7}{}{interF}{#8}
		}\fi		% fin de la possibilité où le temps est plus grand que le trajet, et fin du FPiflt par la même occasion.

}		% fin de \parclum

\begin{figure}[h]
\centering
\begin{pspicture}(-1,-1)(2,1)
% \psframe[linecolor=blue](-1,-1)(2,1)
	\psset{PointSymbol=none, PointName=none}

   \pstGeonode(0,0){E}(-1,0){A}

   \pstLineAB[nodesepA=-1,nodesepB=-2]{A}{E}
\multido{\r=0+1}{20}{%
	\parclum{E}{A}{5}{25}{\r}{0.3}{P}{B}
   \psdot(B)
			}  % Fin du multido
\end{pspicture}
\caption{Mon \oe uvre d'art actuelle}\label{fig_refr_pm}
\end{figure}

\end{document}


>    Je terminerai enfin par une mise en garde générale. J'ai traité 
> beaucoup de questions relatives à pstricks et, assez souvent, le souhait 
> de vouloir manipuler soi-même les coordonnées pour calculer des 
> longueurs, des angles ou que sais-je encore était fondamentalement une 
> mauvaise idée. Je ne dis pas que ce soit le cas pour le posteur initial 
> mais je demande à voir le vrai problème : le cas particulier où le 
> calcul de la longueur d'un segment (ou autre) est vital. Par exemple, si 
> c'est pour afficher la longueur d'un segment à côté de celui-ci, c'est 
> sûr que le calcul est légitime, sinon...


   Le posteur initial a toujours en tête son histoire de réfraction. En fait je voudrais maintenant dire que j'ai un certain point (au sens pstricks) au dessus d'une interface entre deux milieux.
  Cette interface est une droite donnée par les points A et B. Le point est lumineux et émet un rayon lumineux avec un certain angle.
   Ensuite je demande à pstrikcs et à FPeval de me dire où sera la lumière après un temps t et de l'y metre un point.
  
    Le calcul à faire est le suivant : tant que le lumière est au-dessus de l'interface, elle avance à une vitesse de 1, et en-dessous, à une vitesse 0.5. Je dois donc d'abord voir si le temps écoulé est suffisant pour dépasser l'interface, et si oui, voir de combien la lumière l'a dépassée.

Pour l'instant j'ai tapé l'exemple suivant qui fonctionne bien sauf que je suis obligé de définir mon point de départ par un angle et une distance au lieu de simplement un point.


\documentclass[a4paper,12pt]{book}


\usepackage{pstricks,pst-node,pst-abspos,pst-plot,pstricks-add}
\usepackage{multido}
\usepackage{fp}
\usepackage{ifthen}
\usepackage{pst-eucl}
\usepackage{subfigure}
\usepackage{graphicx}


\usepackage[ps2pdf]{hyperref} 		%Doit êre appelé en dernier.

\usepackage{ucs}
\usepackage[utf8x]{inputenc}
\usepackage[english,frenchb]{babel}



\newcommand{\pstRayon}[2]{%
  \psset{PointSymbol=none, PointName=none}
\pstMiddleAB{#1}{#2}{inter}
{\psset{linecolor=blue}
\psline[arrows=->](#1)(inter)  % Ce qui serait bien c'est d'avoir la longeur du segment, de la diviser par deux et de mettre la flèche bien au milieu.
\psline(#1)(#2)
} % fin du psset
}

\begin{document}

% #1 est le point où le rayon va traverser l'interface
% #2 est un autre point de l'interface
% #3 est la distance à laquelle le rayon part
% #4 est l'angle avec lequel il arrive (angle en degré avec la normale)
% #5 est le temps depuis lequel il est parti
% #6 est la vitesse de la lumière dans le second milieu; celle dans le premier est 1.
% #7 est le nom du point pstricks de départ; c'est cette macro qui la place
% #8 est le nom du point pstricks d'arrivée.
\newcommand{\parclum}[8]{%
\FPeval{AngleNorRad}{\FPpi*#4/180}
\FPeval{AngleTgRad}{\FPpi/2-\AngleNorRad}
\FPeval{AngleTgDeg}{#4+90}

\FPround{\AngleTgDeg}{\AngleTgDeg}{3}
\FPround{\AngleTgRad}{\AngleTgRad}{3}
\FPround{\AngleNorRad}{\AngleNorRad}{3}

   \pstRotation[RotAngle=\AngleTgDeg]{#1}{#2}[lsept]
   \rput(E){\pstGeonode(#3;0){inter}}					% Place un point à une distance #3 de E pour ensuite construire le cercle correspondant
   \pstInterLC[Radius=\pstDistAB{E}{inter}]{#1}{lsept}{#1}{}{#7}{interF}


\FPiflt{#5}{#3}{%		Le cas où le temps est plus petit que la distance est facile et est traité ici
   \rput(P){\pstGeonode(#5;0){inter}}	
   \pstInterLC[Radius=\pstDistAB{#7}{inter}]{#7}{#1}{#7}{}{F}{#8}
  
		}		% fin de la possibilité où le temps est plus court que le trajet
\else{%
   \FPeval{trest}{(#3-#5)/#6}	% Le temps qu'il reste à voyager au moment où le rayon arrive à l'interface, divisé en l'indice de réfraction pour normaliser à un.
   \rput(P){\pstGeonode(\trest;0){inter}}	
   \pstInterLC[Radius=\pstDistAB{#7}{inter}]{#7}{#1}{#7}{}{interF}{#8}
		}\fi		% fin de la possibilité où le temps est plus grand que le trajet, et fin du FPiflt par la même occasion.

}		% fin de \parclum

\begin{figure}[h]
\centering
\begin{pspicture}(-1,-1)(2,1)
% \psframe[linecolor=blue](-1,-1)(2,1)
	\psset{PointSymbol=none, PointName=none}

   \pstGeonode(0,0){E}(-1,0){A}

   \pstLineAB[nodesepA=-1,nodesepB=-2]{A}{E}
\multido{\r=0+1}{20}{%
	\parclum{E}{A}{5}{25}{\r}{0.3}{P}{B}
   \psdot(B)
			}  % Fin du multido
\end{pspicture}
\caption{Mon \oe uvre d'art actuelle}\label{fig_refr_pm}
\end{figure}

\end{document}


 Je pars demain pour une semaine, alors je ne pourai pas discuter plus longement de mes projets dans l'immédiat. Mais vous voyez un peu le genre de préocupations : genre savoir si deux points définits par \pstGeonode sont oui ou non du même côté de la droite définie par deux autres points.
    Tant que je pars de coordonées, ça va parce que je peux faire les calculs avec FPeval en parallèle du dessin en pstricks. Mais si je pars de la simple donnée d'un point en pstricks, ben je ne peux rien faire.

Bonne semaine à toutes et à tous
Laurent



