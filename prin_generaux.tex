% This is part of Un soupçon de physique, sans être agressif pour autant
% Copyright (C) 2006-2009
%   Laurent Claessens
% See the file fdl-1.3.txt for copying conditions.


Ce chapitre a pour objectif de parler de quelque principes généraux de la physique qui transcendent l'une ou l'autre matière précise.

\section{Homogénéité et isotropie}
%+++++++++++++++++++++++++++++++++

\subsection{Homogénéité de l'espace}
%-----------------------------------

En physique, nous supposons que l'espace est homogène, c'est à dire qu'il n'y a pas un lieu privilégié. Voici deux façons de se représenter ce principe :
\begin{itemize}
\item Il n'y a pas un endroit dans l'univers où un panneau est planté pour dire \og ici c'est l'endroit de référence\fg,
\item si on fait la même expérience à deux endroits différents, on doit obtenir le même résultat.
\end{itemize}


Ce principe n'est évidement correct que si \emph{tous} les paramètres de l'expérience sont reproduits. Si mon expérience est de voir à quelle température l'eau entre en ébullition, j'aurai un résultat différent ici ou au sommet de l'Éverest (pour des questions de pression). Mais si je me mets ici dans un caisson à la pression du sommet de l'Éverest, je trouverai le bon résultat.

Une importante conséquence de ce principe est le fait que si une expérience dépend de la position de deux points, en fait le résultat ne peut dépendre que de la \emph{différence} entre les deux points, et non de leur position exacte dans l'espace. L'exemple le plus simple est de mettre par exemple deux aimants en présence. Ils vont s'attirer avec une certaine force. Je ne sais pas cette force, mais je sais que si le premier aimant est à la position $\overrightarrow{ r }$ et le second à la position $\overrightarrow{ r }+ \overrightarrow{ a }$, cette force ne dépendra que de $\overrightarrow{ a }$, c'est à dire de la position relative de l'un par rapport à l'autre, mais pas de leur position absolue dans l'espace.

\subsection{Homogénéité du temps}
%--------------------------------

Prenons un exemple. Tu veux étudier un pendule. Tu lance ton chrono quand tu rentre dans le labo, tu marche à ton aise vers le pendule, tu prends l'extrémité du pendule, tu la déplace un peu. Au moment de lâcher, tu regardes où en es ton chrono. Tu notes le résultat : une minute et vingt cinq secondes. Le pendule fait un aller-retour. Au moment où il a fini, tu regardes à nouveau ton chrono et tu notes le résultat : une minute et vingt sept secondes.

Le principe d'homogénéité te dit que toutes les caractéristiques mesurables du pendules sont contenues dans les \emph{différences} de temps. En l'occurrence, la période du pendule sera la différence entre une minute vingt sept et une minute vingt cinq, soit deux secondes.

En particulier, l'opération de \emph{somme} entre tes deux mesures n'a pas de sens : la grandeur de deux minutes et cinquante deux secondes n'a aucun rapport avec le pendule !

Si tu refais l'expérience le lendemain, tu lâches le pendule à vingt six heures, trente sept minutes et quatorze secondes; et une période plus tard, on en est à vingt six heures, trente sept minutes et seize secondes. À nouveau c'est la différence de deux secondes entre les deux mesures qui a un sens physique.

\subsection{Conclusion des homogénéités}		\label{SecConcHomo}
%------------------------------------

Si dans un système de coordonnées spatio-temporel, tu étudies un phénomène en mesurant deux événements intervenant dans le phénomène aux endroits et instants $(t_0,x_0)$ et $(t_1,x_1)$, les seules grandeurs qui caractérisent réellement le phénomène sont la durée $t_1-t_0$ et la distance $x_1-x_0$.

\subsection{Isotropie de l'espace}
%---------------------------------


\section{Le principe de relativité}
%+++++++++++++++++++++++++++++++++++

%http://fr.wikipedia.org/wiki/Théorie_de_la_relativité
Le \href{http://fr.wikipedia.org/wiki/Théorie_de_la_relativité}{principe de relativité} est un des plus vieux principes physique : il remonte à \href{http://fr.wikipedia.org/wiki/Relativité_galiléenne}{Galilée}. Son contenu est assez simple.
\begin{loiphyz}
Les lois de la nature sont les mêmes dans tous les référentiels d'inertie. En d'autres termes, si vous êtes dans une pièce fermée, il n'existe aucune expérience qui vous permet de savoir si vous êtes au repos ou en mouvement rectiligne uniforme.
\end{loiphyz}

Une conséquence remarquable de ce principe est qu'il est possible de jongler dans le métro entre deux stations. En effet, tant que le métro avance à une vitesse constante, tout se passe exactement comme si il était à l'arrêt, et les balles se comporteront exactement comme dans ton salon.

Autre conséquence. Un principe général en science\footnote{En sciences, j'entends en \emph{vraies} sciences; exit l'astrologie, la philosophie, la graphologie, la psychologie, les sciences politiques, la sorcellerie et autres charlatanismes.} est que si il n'existe pas d'expériences pour distinguer deux choses alors les deux choses doivent être considérées comme équivalentes. Par conséquent, on ne peut pas définir de \emph{repos absolu} parce qu'il serait impossible à distinguer du mouvement rectiligne uniforme. Toute personne se déplaçant en MRU a le droit de se considérer au repos et de dire que c'est le reste de l'univers qui se déplace.


