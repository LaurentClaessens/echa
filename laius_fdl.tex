\vspace{1cm}

Si vous n'avez pas envie de lire toute la licence (ce que je comprends), en voici un mini résumé :
\begin{itemize}
	\item 
		Si vous donnez un pdf du document ou d'une partie à quelqu'un, vous \emph{devez} lui dire en même temps où il peut télécharger les sources \LaTeX. En pratique, le site sur lequel j'ai mit les sources actuelles est déjà indiqué sur la page de garde. Il n'y a donc rien à faire.
	\item
		Si vous modifiez le document, les nouvelles sources \LaTeX\ \emph{doivent} être publiées sur internet à une adresse publiquement accessible\footnote{Les zones de icampus protégées par un mot de passe ne sont pas valables; la dropbox non plus.}, et le site où les sources se trouvent doit être indiqué dans le document.
	\item
		En cas de modification, vous pouvez ajouter votre nom à la liste des auteurs des fichiers modifiés, mais vous ne pouvez pas retirer les noms déjà existants, de plus les modifications elles-mêmes doivent être publiés sous la même licence, et les sources \LaTeX\ des modifications doivent également être publiquement accessibles.
\end{itemize}

\vspace{1cm}

Le but principal de ces conditions est d'éviter que le document deviennent inutilisable parce que les sources \LaTeX\ seront perdues dans $5$ ans. Cela arrive trop souvent. Il s'agit d'un échange : je donne gratuitement le droit de copier, modifier et redistribuer le document, mais en échange, vous devez prendre soin des sources, et transmettre ces droits aux personnes à qui vous distribuez des versions modifiées.

Vous n'êtes par contre absolument pas obligés de me tenir au courant des modifications que vous apportez, bien que cela soit souhaitable pour que chacun puisse profiter des améliorations de tous les autres.
