% Fichier généré automatiquement. Ne pas modifier à la main.
\begin{figure}[ht]
\centering
\begin{pspicture}(-3.0,-0.99)(3.0,9.0)
\psset{PointSymbol=none,PointName=none,algebraic=true}

\psplot[]{-3}{3}{(x)^2}
\pstGeonode[PointSymbol=none,PointName=none](1.0,1.0){apl}
\pstGeonode[PointSymbol=*](1.0,1.0){apl}
\rput(apl){\rput(0.3;135){$P$}}
\psaxes[]{->}(0.0,0.0)(-2.99,-0.99)(3.0,9.0)
\psgrid[gridlabels=0,subgriddiv=0,griddots=5](-2.99,-0.99)(3.0,9.0)
\end{pspicture}

\psset{xunit=1,yunit=1}

\caption{La dérivée à la fonction $x\mapsto x^2$}\label{fig_tg_parab}
\end{figure}
% This is part of Un soupçon de physique, sans être agressif pour autant
% Copyright (C) 2006-2009
%   Laurent Claessens
% See the file fdl-1.3.txt for copying conditions.


